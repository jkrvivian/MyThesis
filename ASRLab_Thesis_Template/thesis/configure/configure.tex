%
% This file is part of the project of
% National Cheng Kung University (NCKU) Thesis/Dissertation Template in LaTex.
% This project is hold at
%     <https://github.com/wengan-li/ncku-thesis-template-latex>
% by Wen-Gan Li.
%
% This project is distributed in the hope of usefuling to someone,
% you can redistribute it and/or modify it under the terms of the
% Attribution-NonCommercial-ShareAlike 4.0 International.
%
% You should have received a copy of the
% Attribution-NonCommercial-ShareAlike 4.0 International
% along with this project.
% If not, see <http://creativecommons.org/licenses/by-nc-sa/4.0/legalcode.txt>.
%
% Please feel free to fork it, modify it, and try it.
% Have fun !!!
%

% ------------------------------------------------

\documentclass[12pt, a4paper, onecolumn, english]{report}

% ------------------------------------------------

% XeLaTex檢查點, 以要求必須使用XeLaTex來處理模版
\usepackage{ifxetex}
\ifxetex\else\errmessage{模版: 請使用XeLaTex來產生論文.}\stop\fi

% ------------------------------------------------

% 引用字體的基本設定
%
% No longer need \usepackage[T1]{fontenc} and
% \usepackage[utf8]{inputenc} when using XeLaTeX and LuaLaTeX as the engine.
%

% 引用fontspec以提供控制英文字型
\usepackage{fontspec}
\defaultfontfeatures{Ligatures=TeX} % To support LaTeX quoting style

% 引用xeCJK以提供控制中文字型
\usepackage{xeCJK}

% ------------------------------------------------

% 引用需要的LaTex packages

% Some base packages
\usepackage{geometry}
\usepackage{fp}
\usepackage{ifthen}
\usepackage{pgfkeys}
\usepackage{xparse}
\usepackage{amsmath}
\usepackage[framemethod=tikz]{mdframed}
\usepackage{url}
\usepackage{color}
\usepackage{etoolbox}

% For floats
% flafter package will make sure that the floats are
% not placed before their definition
\usepackage{flafter}

% For list
% Ref: <http://ftp.yzu.edu.tw/CTAN/macros/latex/contrib/enumitem/enumitem.pdf>
%\usepackage{enumitem}
%\setlist{noitemsep, nosep}

% For paragraphs
\usepackage{parskip}

% For line spacing
% Ref: <https://en.wikibooks.org/wiki/LaTeX/Paragraph_Formatting>
\usepackage{setspace}

% For PDF
\usepackage{hyperref}
\usepackage{pdfpages}

% For figure
\usepackage{graphicx}
\usepackage{caption}
\usepackage{subcaption}

% For table
\usepackage{array}
\usepackage{multirow}
\usepackage{booktabs}
\usepackage{diagbox}

% For comment
\usepackage{comment}

% For 目錄
\usepackage[tocgraduated]{tocstyle}
\usetocstyle{standard}
\setcounter{tocdepth}{4} % 目錄會顯示subsubsection

% For appendix
\usepackage[titletoc]{appendix}

% For chinese number in title
% http://ftp.yzu.edu.tw/CTAN/macros/latex/contrib/zhnumber/zhnumber.pdf
\usepackage{zhnumber}

% For pseudocode
\usepackage{algorithm}
\usepackage[noend]{algpseudocode}
\algnewcommand\algorithmicswitch{\textbf{switch}}
\algnewcommand\algorithmiccase{\textbf{case}}
\algnewcommand\algorithmicdefault{\textbf{default}}
\algnewcommand\algorithmicbreak{\textbf{break}}
\algdef{SE}[SWITCH]{Switch}{EndSwitch}[1]{\algorithmicswitch\ #1\ }{\algorithmicend\ \algorithmicswitch}%
\algdef{SE}[CASE]{Case}{EndCase}[1]{\algorithmiccase\ #1:}{\algorithmicend\ \algorithmiccase}%
\algdef{SE}[CASE]{Default}{EndDefault}[0]{\algorithmicdefault:}{\algorithmicend\ \algorithmiccase}%
\algtext*{EndSwitch}%
\algtext*{EndCase}%
\algtext*{EndDefault}%
\def\Break{\algorithmicbreak}

% For theorem
\usepackage{amsthm}
\usepackage{amssymb}
%\usepackage{chngcntr}
%\usepackage{./template/libs/apptools}

% ------------------------------------------------

% 有關學校對論文要求的設定

% --------------------------

% 一些用來設定function和variable的command
%
% This file is part of the project of
% National Cheng Kung University (NCKU) Thesis/Dissertation Template in LaTex.
% This project is hold at
%     <https://github.com/wengan-li/ncku-thesis-template-latex>
% by Wen-Gan Li.
%
% This project is distributed in the hope of usefuling to someone,
% you can redistribute it and/or modify it under the terms of the
% Attribution-NonCommercial-ShareAlike 4.0 International.
%
% You should have received a copy of the
% Attribution-NonCommercial-ShareAlike 4.0 International
% along with this project.
% If not, see <http://creativecommons.org/licenses/by-nc-sa/4.0/legalcode.txt>.
%
% Please feel free to fork it, modify it, and try it.
% Have fun !!!
%

% ----------------------------------------------------------------------------
% 一些用來設定function和variable的command
% Some function and variable that let user use and configure
%
% 此處只是一些預設值和function
% 修改內容是在'conf/conf'
% ----------------------------------------------------------------------------

% Static variable and some provided API
%
% This file is part of the project of
% National Cheng Kung University (NCKU) Thesis/Dissertation Template in LaTex.
% This project is hold at
%     <https://github.com/wengan-li/ncku-thesis-template-latex>
% by Wen-Gan Li.
%
% This project is distributed in the hope of usefuling to someone,
% you can redistribute it and/or modify it under the terms of the
% Attribution-NonCommercial-ShareAlike 4.0 International.
%
% You should have received a copy of the
% Attribution-NonCommercial-ShareAlike 4.0 International
% along with this project.
% If not, see <http://creativecommons.org/licenses/by-nc-sa/4.0/legalcode.txt>.
%
% Please feel free to fork it, modify it, and try it.
% Have fun !!!
%

% Some common helper function

% ----------------------------------------------------------------------------

% Some helper functions

\newcommand{\GetMonthInEng}[1]
{%
  \ifthenelse{\equal{#1}{1}}{January}{}%
  \ifthenelse{\equal{#1}{2}}{February}{}%
  \ifthenelse{\equal{#1}{3}}{March}{}%
  \ifthenelse{\equal{#1}{4}}{April}{}%
  \ifthenelse{\equal{#1}{5}}{May}{}%
  \ifthenelse{\equal{#1}{6}}{June}{}%
  \ifthenelse{\equal{#1}{7}}{July}{}%
  \ifthenelse{\equal{#1}{8}}{August}{}%
  \ifthenelse{\equal{#1}{9}}{September}{}%
  \ifthenelse{\equal{#1}{10}}{October}{}%
  \ifthenelse{\equal{#1}{11}}{November}{}%
  \ifthenelse{\equal{#1}{12}}{December}{}%
} % End of \newcommand{}

% 計算出台灣民國幾年
% Get the year using Taiwans' year
\newcommand{\SetOralTaiwanYear}[1]%
{%
  \FPeval{\OralTaiwanYearResult}{clip(#1 - 1911)}%
} % End of \newcommand{}

% 計算出台灣民國幾年
% Get the year using Taiwans' year
\newcommand{\SetThesisTaiwanYear}[1]%
{%
  \FPeval{\ThesisTaiwanYearResult}{clip(#1 - 1911)}%
} % End of \newcommand{}

% ----------------------------------------------------------------------------

% In the minimal example below the macro \modulo{<a>}{<b>} stores the result of <a> mod <b> in the macro \result
\newcommand{\modulo}[2]{%
  \FPeval{\result}{trunc(#1-(#2*trunc(#1/#2,0)),0)}%
}

% ----------------------------------------------------------------------------

% 定義了 fmpage: 一個加框的展示區 framed minipage
% http://brunoj.wordpress.com/2009/10/08/latex-the-framed-minipage/
\newsavebox{\fmbox}
\newenvironment{fmpage}[1]
{\begin{lrbox}{\fmbox}\begin{minipage}{#1}}
{\end{minipage}\end{lrbox}\fbox{\usebox{\fmbox}}}

% ----------------------------------------------------------------------------

\newcommand{\EmptyLine}{\ \\ \par}

% ----------------------------------------------------------------------------

\global\mdfdefinestyle{DescriptionFrameStyle}{%
  linewidth=1pt, apptotikzsetting={%
    \tikzset{mdfbackground/.append style={opacity=0.75}}}%
} % End of \mdfdefinestyle{}
\newenvironment{DescriptionFrame}%
{\begin{mdframed}[style=DescriptionFrameStyle]}%
{\end{mdframed}}

% ----------------------------------------------------------------------------

%
% This file is part of the project of
% National Cheng Kung University (NCKU) Thesis/Dissertation Template in LaTex.
% This project is hold at
%     <https://github.com/wengan-li/ncku-thesis-template-latex>
% by Wen-Gan Li.
%
% This project is distributed in the hope of usefuling to someone,
% you can redistribute it and/or modify it under the terms of the
% Attribution-NonCommercial-ShareAlike 4.0 International.
%
% You should have received a copy of the
% Attribution-NonCommercial-ShareAlike 4.0 International
% along with this project.
% If not, see <http://creativecommons.org/licenses/by-nc-sa/4.0/legalcode.txt>.
%
% Please feel free to fork it, modify it, and try it.
% Have fun !!!
%

% ----------------------------------------------------------------------------
%
% http://tex.stackexchange.com/questions/34312/how-to-create-a-command-with-key-values
%
% 用\begin{figure} .. \end{figure}
% 可能會出現問題
% http://www.tex.ac.uk/cgi-bin/texfaq2html?label=ouparmd
%
% ----------------------------------------------------------------------------

\DeclareDocumentCommand{\SetFigureCaptionAndLabel}{+m +m}
{
  \ifthenelse{\equal{#1}{\empty}}{}%
  {%
    \ifthenelse{\equal{%
      \GetStartExtendedAbstractFigureTableControl}{%
      \ValueDisableExtendedAbstractFigureTableControl}}%
      {\caption{#1}}{\caption[]{#1}}
    \ifthenelse{\equal{#2}{\empty}}{}{\label{#2}}%
  }%
} % End of \DeclareDocumentCommand{}

\DeclareDocumentCommand{\SetFigureCaption}{+m}
{
  \ifthenelse{\equal{#1}{\empty}}{}{\IfNoValueF{#1}{\caption{#1}}}
} % End of \DeclareDocumentCommand{}

\DeclareDocumentCommand{\SetImageLabel}{+m}
{
  \ifthenelse{\equal{#1}{\empty}}{}{\IfNoValueF{#1}{\label{#1}}}
} % End of \DeclareDocumentCommand{}

% -----------------------------------------------------------------

\pgfkeys
{
  /InsertFigure/.is family, /InsertFigure,
  default/.style =
  {
    scale = 1.0,
    angle = 0,
    caption = \empty,
    label = \empty,
    pos = {H},      % Useless, for backporting
    align = \empty, % Useless, for backporting
    opacity = 0.4,
  },
  scale/.estore in = \TmpValueScale,
  angle/.estore in = \TmpValueAngle,
  caption/.estore in = \TmpValueCaption,
  label/.estore in = \TmpValueLabel,
  pos/.estore in = \TmpValuePosition,   % Useless, for backporting
  align/.estore in = \TmpValueAlign,    % Useless, for backporting
  opacity/.estore in = \TmpValueOpacity,
} % End of \pgfkeys{}

% Insert a single column image
\newcommand{\InsertFigure}[2][\empty]
{%
  % Parse the input
  \pgfkeys{/InsertFigure, default, #1}%
  %
  \begin{figure}[H]%
  \begin{minipage}[c]{\textwidth}%
  \begin{mdframed}[skipabove=0pt, skipbelow=0pt, leftmargin=0pt, rightmargin=0pt,
    innerleftmargin=0pt, innerrightmargin=0pt, innertopmargin=0pt,
    innerbottommargin=0pt, linewidth=0pt, apptotikzsetting={%
    \tikzset{mdfbackground/.append style={opacity=\TmpValueOpacity}}}]%
    \makebox[\textwidth]{%
      \includegraphics[
        scale=\TmpValueScale,
        angle=\TmpValueAngle]{#2}%
    }%
  \end{mdframed}%
  \end{minipage}%
  % Set Caption and Label
  \SetFigureCaptionAndLabel{\TmpValueCaption}{\TmpValueLabel}
  \end{figure}%
} % End of \newcommand{}

% -----------------------------------------------------------------

\def\ValueFigureNameDefault{Figure}
\def\ValueFigureNameCustom{Figure} % Default
\def\UseFigureNameDefault{%
  \renewcommand{\figurename}{\ValueFigureNameDefault}}
\def\UseFigureNameCustom{%
  \renewcommand{\figurename}{\ValueFigureNameCustom}}
\newcommand{\SetCustomFigureName}[1]{%
  \renewcommand{\ValueFigureNameCustom}{#1}}

\UseFigureNameCustom % Default

% -----------------------------------------------------------------

% 過去的API, 以 Error提醒不能再使用
\newcommand{\InsertCenterImage}{\errmessage{模版: 由v1.4.1開始, InsertCenterImage已不能再使用, 請改使用InsertFigure.}\stop}
\newcommand{\InsertImage}{\errmessage{模版: 由v1.4.1開始, InsertCenterImage已不能再使用, 請改使用InsertFigure.}\stop}

% -----------------------------------------------------------------
\begin{comment}
\def\ValueFigureNameBoldOn{1}
\def\ValueFigureNameBoldOff{0}
\def\VarFigureNameBoldOption{\ValueFigureNameBoldOn} %Default
\def\GetFigureNameBoldOption{\VarFigureNameBoldOption}
\newcommand{\EnableFigureNameBold}{%
  \renewcommand{\VarFigureNameBoldOption}{\ValueFigureNameBoldOn}}
\newcommand{\DisableFigureNameBold}{%
  \renewcommand{\VarFigureNameBoldOption}{\ValueFigureNameBoldOff}}

% ------------------------------------------

\def\ValueFigureTextBoldOn{3}
\def\ValueFigureTextBoldOff{2}
\def\VarFigureTextBoldOption{\ValueFigureTextBoldOff} %Default
\def\GetFigureTextBoldOption{\VarFigureTextBoldOption}
\newcommand{\EnableFigureTextBold}{%
  \renewcommand{\VarFigureTextBoldOption}{\ValueFigureTextBoldOn}}
\newcommand{\DisableFigureTextBold}{%
  \renewcommand{\VarFigureTextBoldOption}{\ValueFigureTextBoldOff}}
\end{comment}
% ------------------------------------------

% Default style
\newcommand{\UseFigureCaptionDefaultStyle}
{%
%  \ifthenelse{\equal{\GetFigureNameBoldOption}{\ValueFigureNameBoldOn}}
%  {%
%    \captionsetup[figure]{labelfont=bf}
%  }%
%  {%
%    \captionsetup[figure]{labelfont=normalfont}
%  }%
  %
%  \ifthenelse{\equal{\GetFigureTextBoldOption}{\ValueFigureTextBoldOn}}
%  {%
%    \captionsetup[figure]{textfont=bf}%
%  }%
%  {%
%    \captionsetup[figure]{textfont=normalfont}%
%  }%
  \captionsetup[figure]{labelfont=bf, textfont=normalfont}%
} % End of \newcommand{}

% Style for Extended Abstract
\newcommand{\UseFigureCaptionExtendedAbstractStyle}
{%
  \captionsetup[figure]{font=bf}%
  \renewcommand{\thefigure}{\arabic{figure}}%
} % End of \newcommand{}

\UseFigureCaptionDefaultStyle % Default

% -----------------------------------------------------------------

%
% This file is part of the project of
% National Cheng Kung University (NCKU) Thesis/Dissertation Template in LaTex.
% This project is hold at
%     <https://github.com/wengan-li/ncku-thesis-template-latex>
% by Wen-Gan Li.
%
% This project is distributed in the hope of usefuling to someone,
% you can redistribute it and/or modify it under the terms of the
% Attribution-NonCommercial-ShareAlike 4.0 International.
%
% You should have received a copy of the
% Attribution-NonCommercial-ShareAlike 4.0 International
% along with this project.
% If not, see <http://creativecommons.org/licenses/by-nc-sa/4.0/legalcode.txt>.
%
% Please feel free to fork it, modify it, and try it.
% Have fun !!!
%

% ----------------------------------------------------------------------------
%
% http://tex.stackexchange.com/questions/34312/how-to-create-a-command-with-key-values
%
% 用\begin{figure} .. \end{figure}
% 可能會出現問題
% http://www.tex.ac.uk/cgi-bin/texfaq2html?label=ouparmd
%
% ----------------------------------------------------------------------------

\pgfkeys
{
  /InsertFigures/.is family, /InsertFigures,
  default/.style =
  {
    perrow = 1,
    caption = \empty,
    label = \empty,
    align = \empty,      % Useless, for backporting
    opacity = 0.4,
  },
  perrow/.estore in = \TmpMIValueImagePerRow,
  caption/.estore in = \TmpMIValueCaption,
  label/.estore in = \TmpMIValueLabel,
  align/.estore in = \TmpMIValueAlign,      % Useless, for backporting
  opacity/.estore in = \TmpValueOpacity,
} % End of \pgfkeys{}

% Insert multi-figure
% Arg: 1st: Table configure
%      2~9th: Figure (Max 8 Figures)
\DeclareDocumentCommand{\InsertFigures}{
  +O{\empty} +m +G{\empty} +G{\empty} +G{\empty}
  +G{\empty} +G{\empty} +G{\empty} +G{\empty}}
{
  % Parse the input
  \pgfkeys{/InsertFigures, default, #1}%
  %
  \begin{figure}[H]%
  \begin{minipage}[c]{\textwidth}%
  \begin{mdframed}[skipabove=0pt, skipbelow=0pt, leftmargin=0pt, rightmargin=0pt,
    innerleftmargin=0pt, innerrightmargin=0pt, innertopmargin=0pt,
    innerbottommargin=0pt, linewidth=0pt, apptotikzsetting={%
    \tikzset{mdfbackground/.append style={opacity=\TmpValueOpacity}}}]%
      \if \TmpMIValueImagePerRow 1
        \InsertFiguresOnePerRow{#2}{#3}{#4}{#5}{#6}{#7}{#8}{#9}%
      \fi
      \if \TmpMIValueImagePerRow 2
        \InsertFiguresTwoPerRow{#2}{#3}{#4}{#5}{#6}{#7}{#8}{#9}%
      \fi
      \if \TmpMIValueImagePerRow 3
        \InsertFiguresThreePerRow{#2}{#3}{#4}{#5}{#6}{#7}{#8}{#9}%
      \fi
      \if \TmpMIValueImagePerRow 4
        \InsertFiguresFourPerRow{#2}{#3}{#4}{#5}{#6}{#7}{#8}{#9}%
      \fi
      %
  \end{mdframed}%
  \end{minipage}%
  % Set Caption and Label
  \SetFigureCaptionAndLabel{\TmpMIValueCaption}{\TmpMIValueLabel}
  \end{figure}%
} % End of \newcommand{}

% Low-level insert image
\newcommand{\InsertSubfigureBox}[2]
{
  \begin{subfigure}{#1\textwidth}%
  \centering
  %
  \InsertFiguresSubFigure#2
  % Set Caption and Label
  \SetFigureCaptionAndLabel{%
    \TmpMISubValueCaption}{\TmpMISubValueLabel}
  \end{subfigure}
} % End of \newcommand{}

%----------------------------------------------------------------

\newcommand{\InsertSubfigureOneFigure}[1]
{
  \InsertSubfigureBox{1.0}{#1}
} % End of \newcommand{}

\newcommand{\InsertSubfigureTwoFigure}[2]
{
  \InsertSubfigureBox{0.5}{#1}%
  ~
  \InsertSubfigureBox{0.5}{#2}%
} % End of \newcommand{}

\newcommand{\InsertSubfigureThreeFigure}[3]
{
  \InsertSubfigureBox{0.315}{#1}%
  ~
  \InsertSubfigureBox{0.315}{#2}%
  ~
  \InsertSubfigureBox{0.315}{#3}%
} % End of \newcommand{}

\newcommand{\InsertSubfigureFourFigure}[4]
{
  \InsertSubfigureBox{0.225}{#1}%
  ~
  \InsertSubfigureBox{0.225}{#2}%
  ~
  \InsertSubfigureBox{0.225}{#3}%
  ~
  \InsertSubfigureBox{0.225}{#4}%
} % End of \newcommand{}

%----------------------------------------------------------------

\DeclareDocumentCommand{\InsertFiguresOnePerRow}{
  +m                   +G{\empty} +G{\empty} +G{\empty}
  +G{\empty} +G{\empty} +G{\empty} +G{\empty}}
{
  \InsertSubfigureOneFigure{#1}
  %
  \ifthenelse{\equal{#2}{\empty}}{}%
  {

    \InsertSubfigureOneFigure{#2}
  }%
  %
  \ifthenelse{\equal{#3}{\empty}}{}%
  {

    \InsertSubfigureOneFigure{#3}
  }%  %
  \ifthenelse{\equal{#4}{\empty}}{}%
  {

    \InsertSubfigureOneFigure{#4}
  }%  %
  \ifthenelse{\equal{#5}{\empty}}{}%
  {

    \InsertSubfigureOneFigure{#5}
  }%  %
  \ifthenelse{\equal{#6}{\empty}}{}%
  {

    \InsertSubfigureOneFigure{#6}
  }%  %
  \ifthenelse{\equal{#7}{\empty}}{}%
  {

    \InsertSubfigureOneFigure{#7}
  }%  %
  \ifthenelse{\equal{#8}{\empty}}{}%
  {

    \InsertSubfigureOneFigure{#8}
  }%
} % End of \newcommand{}

\DeclareDocumentCommand{\InsertFiguresTwoPerRow}{
  +m                   +G{\empty} +G{\empty} +G{\empty}
  +G{\empty} +G{\empty} +G{\empty} +G{\empty}}
{
  %
  \ifthenelse{\equal{#2}{\empty}}%
  {
    \InsertSubfigureOneFigure{#1}%
  }%
  {
    \InsertSubfigureTwoFigure{#1}{#2}%
  }%
  %
  \ifthenelse{\equal{#4}{\empty}}%
  {
    \ifthenelse{\equal{#3}{\empty}}{}%
    {

      \InsertSubfigureOneFigure{#3}%
    }%
  }%
  {

    \InsertSubfigureTwoFigure{#3}{#4}%
  }%
  %
  \ifthenelse{\equal{#6}{\empty}}%
  {
    \ifthenelse{\equal{#5}{\empty}}{}%
    {

      \InsertSubfigureOneFigure{#5}%
    }%
  }%
  {

    \InsertSubfigureTwoFigure{#5}{#6}%
  }%
  %
  \ifthenelse{\equal{#8}{\empty}}%
  {
    \ifthenelse{\equal{#7}{\empty}}{}%
    {

      \InsertSubfigureOneFigure{#7}%
    }%
  }%
  {

    \InsertSubfigureTwoFigure{#7}{#8}%
  }%
} % End of \newcommand{}

\DeclareDocumentCommand{\InsertFiguresThreePerRow}{
  +m                   +G{\empty} +G{\empty} +G{\empty}
  +G{\empty} +G{\empty} +G{\empty} +G{\empty}}
{
  \ifthenelse{\equal{#3}{\empty}}%
  {
    \ifthenelse{\equal{#2}{\empty}}%
    {
      \InsertSubfigureOneFigure{#1}%
    }%
    {
      \InsertSubfigureTwoFigure{#1}{#2}%
    }%
  }%
  {
    \InsertSubfigureThreeFigure{#1}{#2}{#3}%
  }%
  %
  \ifthenelse{\equal{#6}{\empty}}%
  {
    \ifthenelse{\equal{#5}{\empty}}%
    {

      \InsertSubfigureOneFigure{#4}%
    }%
    {

      \InsertSubfigureTwoFigure{#4}{#5}%
    }%
  }%
  {

    \InsertSubfigureThreeFigure{#4}{#5}{#6}%
  }%
  %
  \ifthenelse{\equal{#8}{\empty}}%
  {
    \ifthenelse{\equal{#7}{\empty}}{}%
    {

      \InsertSubfigureOneFigure{#7}%
    }%
  }%
  {

    \InsertSubfigureTwoFigure{#7}{#8}%
  }%
} % End of \newcommand{}

\DeclareDocumentCommand{\InsertFiguresFourPerRow}{
  +m                   +G{\empty} +G{\empty} +G{\empty}
  +G{\empty} +G{\empty} +G{\empty} +G{\empty}}
{
  %
  \ifthenelse{\equal{#4}{\empty}}%
  {
    \ifthenelse{\equal{#3}{\empty}}%
    {
      \ifthenelse{\equal{#2}{\empty}}%
      {
        \InsertSubfigureOneFigure{#1}%
      }%
      {
        \InsertSubfigureTwoFigure{#1}{#2}%
      }%
    }%
    {
      \InsertSubfigureThreeFigure{#1}{#2}{#3}%
    }%
  }%
  {
    \InsertSubfigureFourFigure{#1}{#2}{#3}{#4}%
  }%
  %
  \ifthenelse{\equal{#8}{\empty}}%
  {
    \ifthenelse{\equal{#7}{\empty}}%
    {
      \ifthenelse{\equal{#6}{\empty}}%
      {

        \InsertSubfigureOneFigure{#5}%
      }%
      {

        \InsertSubfigureTwoFigure{#5}{#6}%
      }%
    }%
    {

      \InsertSubfigureThreeFigure{#5}{#6}{#7}%
    }%
  }%
  {

    \InsertSubfigureFourFigure{#5}{#6}{#7}{#8}%
  }%
} % End of \newcommand{}

% ----------------------------------------------------------------------------

\pgfkeys
{
  /InsertFiguresSubFigure/.is family, /InsertFiguresSubFigure,
  default/.style =
  {
    scale = 1.0,
    angle = 0,
    caption = \empty,
    label = \empty,
    align = \empty,      % Useless, for backporting
  },
  scale/.estore in = \TmpMISubValueScale,
  angle/.estore in = \TmpMISubValueAngle,
  caption/.estore in = \TmpMISubValueCaption,
  label/.estore in = \TmpMISubValueLabel,
  align/.estore in = \TmpMISubValueAlign,      % Useless, for backporting
} % End of \pgfkeys{}

% Low-level insert image
\newcommand{\InsertFiguresSubFigure}[2][\empty]
{
  % Parse the input
  \pgfkeys{/InsertFiguresSubFigure, default, #1}
  %
  \includegraphics[
    scale=\TmpMISubValueScale,
    angle=\TmpMISubValueAngle]{#2}
  %
} % End of \newcommand{}

% ----------------------------------------------------------------------------

% 過去的API, 以 Error提醒不能再使用
\newcommand{\InsertMultiImages}{\errmessage{模版: 由v1.4.1開始, InsertCenterImage已不能再使用, 請改使用InsertFigures.}\stop}

% -----------------------------------------------------------------

%
% This file is part of the project of
% National Cheng Kung University (NCKU) Thesis/Dissertation Template in LaTex.
% This project is hold at
%     <https://github.com/wengan-li/ncku-thesis-template-latex>
% by Wen-Gan Li.
%
% This project is distributed in the hope of usefuling to someone,
% you can redistribute it and/or modify it under the terms of the
% Attribution-NonCommercial-ShareAlike 4.0 International.
%
% You should have received a copy of the
% Attribution-NonCommercial-ShareAlike 4.0 International
% along with this project.
% If not, see <http://creativecommons.org/licenses/by-nc-sa/4.0/legalcode.txt>.
%
% Please feel free to fork it, modify it, and try it.
% Have fun !!!
%

% ----------------------------------------------------------------------------

\DeclareDocumentCommand{\SetTableCaptionAndLabel}{+m +m}
{%
  \ifthenelse{\equal{#1}{\empty}}{}%
  {%
    \ifthenelse{\equal{%
      \GetStartExtendedAbstractFigureTableControl}{%
      \ValueDisableExtendedAbstractFigureTableControl}}%
      {\caption{#1}}{\caption[]{#1}}
    \ifthenelse{\equal{#2}{\empty}}{}{\label{#2}}%
  }%
} % End of \DeclareDocumentCommand{}

\DeclareDocumentCommand{\SetTableCaptionStarAndLabel}{+m +m}
{%
  \ifthenelse{\equal{#1}{\empty}}{}%
  {%
    \caption*{#1}%
    \ifthenelse{\equal{#2}{\empty}}{}{\label{#2}}%
  }%
} % End of \DeclareDocumentCommand{}

\newcommand{\DisplayTableContent}[5]
{%
  \begin{minipage}[c]{\textwidth}%
  \begin{mdframed}[skipabove=0pt, skipbelow=0pt, leftmargin=0pt, rightmargin=0pt,
    innerleftmargin=0pt, innerrightmargin=0pt, innertopmargin=0pt,
    innerbottommargin=0pt, linewidth=0pt, apptotikzsetting={%
    \tikzset{mdfbackground/.append style={opacity=#4}}}]%
  \ifthenelse{\equal{#1}{0.0}}%
  {%
    \makebox[\textwidth]%
    {%
      \setlength{\tabcolsep}{#2}%
      \renewcommand{\arraystretch}{#3}%
      #5%
    }%
  }{%
    \makebox[\textwidth]%
    {%
      \resizebox{#1\paperwidth}{!}%
      {%
        \setlength{\tabcolsep}{#2}%
        \renewcommand{\arraystretch}{#3}%
        #5%
      }%
    }%
  }%
  \end{mdframed}%
  \end{minipage}%
} % End of \newcommand{}

\pgfkeys
{
  /InsertTable/.is family, /InsertTable,
  default/.style =
  {
    scale = 0.0,
    nomtitle = \empty, % For nomenclature
    caption = \empty,
    label = \empty,
    pos = top, % Position of caption or nomtitle
    tabcolsep = 6pt,
    arraystretch = 1,
    opacity = 0.4,
  },
  scale/.estore in = \TmpValueScale,
  nomtitle/.estore in = \TmpValueNomTitle,
  caption/.estore in = \TmpValueCaption,
  label/.estore in = \TmpValueLabel,
  pos/.estore in = \TmpValuePosition,
  tabcolsep/.estore in = \TmpValueTabColSep,
  arraystretch/.estore in = \TmpValueArrayStretch,
  opacity/.estore in = \TmpValueOpacity,
} % End of \pgfkeys{}

\newcommand{\InsertTable}[2][\empty]
{%
  % Parse the input
  \pgfkeys{/InsertTable, default, #1}%
  %
  \begin{table}[H]%
  %
    \ifthenelse{\equal{\TmpValuePosition}{top}}%
    {%
      \ifthenelse{\equal{\TmpValueNomTitle}{\empty}}%
        {\SetTableCaptionAndLabel{\TmpValueCaption}{\TmpValueLabel}}%
        {\SetTableCaptionStarAndLabel{\TmpValueNomTitle}{\TmpValueLabel}}%
      \DisplayTableContent{%
        \TmpValueScale}{\TmpValueTabColSep}{%
        \TmpValueArrayStretch}{\TmpValueOpacity}{#2}%
    }%
    {%
      \DisplayTableContent{%
        \TmpValueScale}{\TmpValueTabColSep}{%
        \TmpValueArrayStretch}{\TmpValueOpacity}{#2}%
      \ifthenelse{\equal{\TmpValueNomTitle}{\empty}}%
        {\SetTableCaptionAndLabel{\TmpValueCaption}{\TmpValueLabel}}%
        {\SetTableCaptionStarAndLabel{\TmpValueNomTitle}{\TmpValueLabel}}%
    }%
  \end{table}%
} % End of \newcommand{}

% -----------------------------------------------------------------

\newcolumntype{L}[1]{>{\raggedright\let\newline\\\arraybackslash\hspace{0pt}}m{#1}}
\newcolumntype{C}[1]{>{\centering\let\newline\\\arraybackslash\hspace{0pt}}m{#1}}
\newcolumntype{R}[1]{>{\raggedleft\let\newline\\\arraybackslash\hspace{0pt}}m{#1}}

%\newcolumntype{LT}[1]{>{\raggedright\let\newline\\\arraybackslash\hspace{0pt}}p{#1}}
%\newcolumntype{LC}[1]{>{\raggedright\let\newline\\\arraybackslash\hspace{0pt}}m{#1}}
%\newcolumntype{LB}[1]{>{\raggedright\let\newline\\\arraybackslash\hspace{0pt}}b{#1}}

%\newcolumntype{CT}[1]{>{\centering\let\newline\\\arraybackslash\hspace{0pt}}p{#1}}
%\newcolumntype{CC}[1]{>{\centering\let\newline\\\arraybackslash\hspace{0pt}}m{#1}}
%\newcolumntype{CB}[1]{>{\centering\let\newline\\\arraybackslash\hspace{0pt}}b{#1}}

%\newcolumntype{RT}[1]{>{\raggedleft\let\newline\\\arraybackslash\hspace{0pt}}p{#1}}
%\newcolumntype{RC}[1]{>{\raggedleft\let\newline\\\arraybackslash\hspace{0pt}}m{#1}}
%\newcolumntype{RB}[1]{>{\raggedleft\let\newline\\\arraybackslash\hspace{0pt}}b{#1}}

% -----------------------------------------------------------------

\def\ValueTableNameDefault{Table}
\def\ValueTableNameCustom{Table} % Default
\def\UseTableNameDefault{%
  \renewcommand{\tablename}{\ValueTableNameDefault}}
\def\UseTableNameCustom{%
  \renewcommand{\tablename}{\ValueTableNameCustom}}
\newcommand{\SetCustomTableName}[1]{%
  \renewcommand{\ValueTableNameCustom}{#1}}

\UseTableNameCustom % Default

% -----------------------------------------------------------------
\begin{comment}
\def\ValueTableNameBoldOn{1}
\def\ValueTableNameBoldOff{0}
\def\VarTableNameBoldOption{\ValueTableNameBoldOn} %Default
\def\GetTableNameBoldOption{\VarTableNameBoldOption}
\newcommand{\EnableTableNameBold}{%
  \renewcommand{\VarTableNameBoldOption}{\ValueTableNameBoldOn}}
\newcommand{\DisableTableNameBold}{%
  \renewcommand{\VarTableNameBoldOption}{\ValueTableNameBoldOff}}

% ------------------------------------------

\def\ValueTableTextBoldOn{3}
\def\ValueTableTextBoldOff{2}
\def\VarTableTextBoldOption{\ValueTableTextBoldOff} %Default
\def\GetTableTextBoldOption{\VarTableTextBoldOption}
\newcommand{\EnableTableTextBold}{%
  \renewcommand{\VarTableTextBoldOption}{\ValueTableTextBoldOn}}
\newcommand{\DisableTableTextBold}{%
  \renewcommand{\VarTableTextBoldOption}{\ValueTableTextBoldOff}}
\end{comment}
% ------------------------------------------

%\newcommand\TableCaptionFormatStyleLabel[1]{#1}
%\newcommand\TableCaptionFormatStyleText[1]{\begin{bfseries}#1\end{bfseries}}
%\begin{bfseries} Text to bold \end{bfseries}
%\DeclareCaptionFormat{TableCaptionFormatStyle}{\TableCaptionFormatStyleLabel{#1#2}\TableCaptionFormatStyleText{#3}}

% Default style
\newcommand{\UseTableCaptionDefaultStyle}
{%
%  \ifthenelse{\equal{\GetTableNameBoldOption}{\ValueTableNameBoldOn}}
%  {%
%    \captionsetup[table]{labelfont=bf}
%  }%
%  {%
%    \captionsetup[table]{labelfont=normalfont}
%  }%
  %
%  \ifthenelse{\equal{\GetTableTextBoldOption}{\ValueTableTextBoldOn}}
%  {%
%    \captionsetup[table]{textfont=bf}%
%  }%
%  {%
%    \captionsetup[table]{textfont=normalfont}%
%  }%
  \captionsetup[table]{labelfont=bf, textfont=normalfont}%
} % End of \newcommand{}

% Style for Extended Abstract
\newcommand{\UseTableCaptionExtendedAbstractStyle}
{%
  \captionsetup[table]{font=bf}%
  \renewcommand{\thetable}{\arabic{table}}%
} % End of \newcommand{}

\UseTableCaptionDefaultStyle % Default

% -----------------------------------------------------------------

%
% This file is part of the project of
% National Cheng Kung University (NCKU) Thesis/Dissertation Template in LaTex.
% This project is hold at
%     <https://github.com/wengan-li/ncku-thesis-template-latex>
% by Wen-Gan Li.
%
% This project is distributed in the hope of usefuling to someone,
% you can redistribute it and/or modify it under the terms of the
% Attribution-NonCommercial-ShareAlike 4.0 International.
%
% You should have received a copy of the
% Attribution-NonCommercial-ShareAlike 4.0 International
% along with this project.
% If not, see <http://creativecommons.org/licenses/by-nc-sa/4.0/legalcode.txt>.
%
% Please feel free to fork it, modify it, and try it.
% Have fun !!!
%

% Some helper function use for oral document

% ----------------------------------------------------------------------------
\def \OralIndexHeader {Oral presentation document}
% ----------------------------------------------------------------------------

\newcommand{\StartOralTemplateDocChi}
{
  % 由於中文版的watermark是用文字, 所以先把Logo版關掉
  \ClearWatermarkStyle
  %
  % 使用文字版watermark
  \UseWatermarkTextStyle
  %
  \singlespacing%
  %
  \StartNewPage
  %
  % 設定使用 無頁碼
  \thispagestyle{empty}
  %
  % Aligned to the center of the page
  \begin{center}
} % End of \newcommand{}

\newcommand{\StartOralTemplateDocEng}
{
  %
  \singlespacing%
  %
  \StartNewPage
  %
  % 設定使用 無頁碼
  \thispagestyle{empty}
  %
  % Aligned to the center of the page
  \begin{center}
} % End of \newcommand{}

\newcommand{\EndOralTemplateDoc}
{
  % End of alignment
  \end{center}
  %
  % End of page
  \EndOfPage
  \UseDefaultLineStretch
  %
  % 重新使用學校浮水印 Watermark
  \ClearWatermarkStyle
  \UseWatermarkFigureStyle
} % End of \newcommand{}

% ----------------------------------------------------------------------------

% 口試委員 Committee member(s)
\newcommand{\CommitteeSize}{9} % Default
\newcommand{\GetCommitteeSize}{\CommitteeSize}
\newcommand{\SetCommitteeSize}[1]
{
  \ifthenelse{#1 < 2}
  {
    \renewcommand{\CommitteeSize}{2}
  } % End of if{}
  {
    \ifthenelse{#1 > 9}
    {\renewcommand{\CommitteeSize}{9}}
    {\renewcommand{\CommitteeSize}{#1}}
  } % End of else{}
} % End of \newcommand{}

% 口試委員簽名區
\newcommand{\DisplayCommitteeSignatureArea}
{
  \par
  % 口試委員 至少2位
  \ifthenelse{\CommitteeSize = 2}
  {
    \begin{minipage}[c][8.0cm][c]{\textwidth}
      \makebox[0.5\textwidth][c]{\namesigdate}
      \makebox[0.5\textwidth][c]{\namesigdate}
    \end{minipage}
  } % End of if{}
  {} % End of else{}
  %
  \ifthenelse{\CommitteeSize = 3}
  {
    \begin{minipage}[c][4.0cm][c]{\textwidth}
      \makebox[0.5\textwidth][c]{\namesigdate}
      \makebox[0.5\textwidth][c]{\namesigdate}
    \end{minipage}
    %
    \begin{minipage}[c][4.0cm][c]{\textwidth}
      \makebox[\textwidth][c]{\namesigdate}
    \end{minipage}
  } % End of if{}
  {} % End of else{}
  %
  \ifthenelse{\CommitteeSize = 4}
  {
    \begin{minipage}[c][4.0cm][c]{\textwidth}
      \makebox[0.5\textwidth][c]{\namesigdate}
      \makebox[0.5\textwidth][c]{\namesigdate}
    \end{minipage}
    %
    \begin{minipage}[c][4.0cm][c]{\textwidth}
      \makebox[0.5\textwidth][c]{\namesigdate}
      \makebox[0.5\textwidth][c]{\namesigdate}
    \end{minipage}
  } % End of if{}
  {} % End of else{}
  %
  \ifthenelse{\CommitteeSize = 5}
  {
    \begin{minipage}[c][2.65cm][c]{\textwidth}
      \makebox[0.5\textwidth][c]{\namesigdate}
      \makebox[0.5\textwidth][c]{\namesigdate}
    \end{minipage}
    %
    \begin{minipage}[c][2.65cm][c]{\textwidth}
      \makebox[0.5\textwidth][c]{\namesigdate}
      \makebox[0.5\textwidth][c]{\namesigdate}
    \end{minipage}
    %
    \begin{minipage}[c][2.65cm][c]{\textwidth}
      \makebox[\textwidth][c]{\namesigdate}
    \end{minipage}
  } % End of if{}
  {} % End of else{}
  %
  \ifthenelse{\CommitteeSize = 6}
  {
    \begin{minipage}[c][2.65cm][c]{\textwidth}
      \makebox[0.5\textwidth][c]{\namesigdate}
      \makebox[0.5\textwidth][c]{\namesigdate}
    \end{minipage}
    %
    \begin{minipage}[c][2.65cm][c]{\textwidth}
      \makebox[0.5\textwidth][c]{\namesigdate}
      \makebox[0.5\textwidth][c]{\namesigdate}
    \end{minipage}
    %
    \begin{minipage}[c][2.65cm][c]{\textwidth}
      \makebox[0.5\textwidth][c]{\namesigdate}
      \makebox[0.5\textwidth][c]{\namesigdate}
    \end{minipage}
  } % End of if{}
  {} % End of else{}
  %
  \ifthenelse{\CommitteeSize = 7}
  {
    \begin{minipage}[c][2.0cm][c]{\textwidth}
      \makebox[0.5\textwidth][c]{\namesigdate}
      \makebox[0.5\textwidth][c]{\namesigdate}
    \end{minipage}
    %
    \begin{minipage}[c][2.0cm][c]{\textwidth}
      \makebox[0.5\textwidth][c]{\namesigdate}
      \makebox[0.5\textwidth][c]{\namesigdate}
    \end{minipage}
    %
    \begin{minipage}[c][2.0cm][c]{\textwidth}
      \makebox[0.5\textwidth][c]{\namesigdate}
      \makebox[0.5\textwidth][c]{\namesigdate}
    \end{minipage}
    %
    \begin{minipage}[c][2.0cm][c]{\textwidth}
      \makebox[\textwidth][c]{\namesigdate}
    \end{minipage}
  } % End of if{}
  {} % End of else{}
  %
  \ifthenelse{\CommitteeSize = 8}
  {
    \begin{minipage}[c][2.0cm][c]{\textwidth}
      \makebox[0.5\textwidth][c]{\namesigdate}
      \makebox[0.5\textwidth][c]{\namesigdate}
    \end{minipage}
    %
    \begin{minipage}[c][2.0cm][c]{\textwidth}
      \makebox[0.5\textwidth][c]{\namesigdate}
      \makebox[0.5\textwidth][c]{\namesigdate}
    \end{minipage}
    %
    \begin{minipage}[c][2.0cm][c]{\textwidth}
      \makebox[0.5\textwidth][c]{\namesigdate}
      \makebox[0.5\textwidth][c]{\namesigdate}
    \end{minipage}
    %
    \begin{minipage}[c][2.0cm][c]{\textwidth}
      \makebox[0.5\textwidth][c]{\namesigdate}
      \makebox[0.5\textwidth][c]{\namesigdate}
    \end{minipage}
  } % End of if{}
  {} % End of else{}
  %
  \ifthenelse{\CommitteeSize = 9}
  {
    \begin{minipage}[c][1.6cm][c]{\textwidth}
      \makebox[0.5\textwidth][c]{\namesigdate}
      \makebox[0.5\textwidth][c]{\namesigdate}
    \end{minipage}
    %
    \begin{minipage}[c][1.6cm][c]{\textwidth}
      \makebox[0.5\textwidth][c]{\namesigdate}
      \makebox[0.5\textwidth][c]{\namesigdate}
    \end{minipage}
    %
    \begin{minipage}[c][1.6cm][c]{\textwidth}
      \makebox[0.5\textwidth][c]{\namesigdate}
      \makebox[0.5\textwidth][c]{\namesigdate}
    \end{minipage}
    %
    \begin{minipage}[c][1.6cm][c]{\textwidth}
      \makebox[0.5\textwidth][c]{\namesigdate}
      \makebox[0.5\textwidth][c]{\namesigdate}
    \end{minipage}
    %
    \begin{minipage}[c][1.6cm][c]{\textwidth}
      \makebox[\textwidth][c]{\namesigdate}
    \end{minipage}
  } % End of if{}
  {} % End of else{}
} % End of \newcommand{}

% ----------------------------------------------------------------------------

% Signature line
\newcommand{\namesigdate}{\rule{5.5cm}{1pt}}

% ----------------------------------------------------------------------------

% 學位考試論文證明書 Defense Certificate

% 要顯示圖片還是範例
\newcommand{\ValueOralDocumentTypeImage}{0}
\newcommand{\ValueOralDocumentTypeTemplate}{1}
\newcommand{\FlagOralDocumentType}{\OralDocumentTemplate} % Default
\newcommand{\GetOralDocumentType}{\FlagOralDocumentType} % Default

\newcommand{\DisplayOralTemplate}
  {\renewcommand{\FlagOralDocumentType}{\ValueOralDocumentTypeTemplate}}
\newcommand{\DisplayOralImage}
  {\renewcommand{\FlagOralDocumentType}{\ValueOralDocumentTypeImage}}

% The path of the image that the oral document
\newcommand\OralDocumentImageChiPath{\empty} % Default
\newcommand\OralDocumentImageEngPath{\empty} % Default
\newcommand{\SetOralImageChi}[1]
{\renewcommand{\OralDocumentImageChiPath}{./context/oral/#1}}
\newcommand{\SetOralImageEng}[1]
{\renewcommand{\OralDocumentImageEngPath}{./context/oral/#1}}

\newcommand{\GetOralImageChiPath}{\OralDocumentImageChiPath}
\newcommand{\GetOralImageEngPath}{\OralDocumentImageEngPath}

\newcommand{\ValueDisplayOralTemplateOn}{1}
\newcommand{\ValueDisplayOralTemplateOff}{0}
\newcommand{\VarDisplayOralTemplateChi}{\ValueDisplayOralTemplateOff}
\newcommand{\VarDisplayOralTemplateEng}{\ValueDisplayOralTemplateOff}

\newcommand{\DisplayOralChiTemplate}
  {\renewcommand{\VarDisplayOralTemplateChi}{\ValueDisplayOralTemplateOn}}
\newcommand{\GetDisplayOralChiTemplate}{\VarDisplayOralTemplateChi}

\newcommand{\DisplayOralEngTemplate}
  {\renewcommand{\VarDisplayOralTemplateEng}{\ValueDisplayOralTemplateOn}}
\newcommand{\GetDisplayOralTemplateEng}{\VarDisplayOralTemplateEng}

% ----------------------------------------------------------------------------

% Use to include the oral files
\newcommand{\DisplayOral}{% ------------------------------------------------
%                  產生口試同意書
% ------------------------------------------------
% 給學校的文件需要同時有中英兩版
%
% ------------------------------------------------
%          載入基本設定 Basic configuration
% ------------------------------------------------
\input{./configure/configure}
\begin{document}

% ------------------------------------------------
%                  中文口試同意書
% ------------------------------------------------
\ClearWatermarkStyle
\UseWatermarkTextStyle
\singlespacing
\newpage
\phantomsection
\thispagestyle{empty}
\begin{center}

% 顯示 校名, 論文種類
\begin{minipage}[c][1.5cm][t]{\textwidth}
  \begin{center}
    \makebox[\textwidth][c]{\Huge 國立成功大學}

    \vspace{0.5cm}

    \makebox[\textwidth][c]{\Huge 碩士論文}
    
  \end{center}
\end{minipage}

\vspace{1.6cm}

% Chinese and English title 中英文題目
\begin{minipage}[c][2.5cm][t]{\textwidth}
  \begin{center}
    \makebox[\textwidth][c]{\parbox{\paperwidth}{\center \Large 論文中文名稱}}

    \vspace{0.5cm}

    \makebox[\textwidth][c]{\parbox{\paperwidth}{\center \Large Thesis English Name}}\\
  \end{center}
\end{minipage}


\vspace{0.3cm}

% 顯示學生的名字
\begin{minipage}[c][1.5cm][t]{\textwidth}
  \begin{center}
    \hspace{2.4em}
    \makebox[4.8em][r]{\Large 研究生:}
    \makebox[7.2em][l]{\Large 作者名稱} \\

    \vspace{0.5cm}

    \makebox[\textwidth][c]{\Large 本論文業經審查及口試合格特此證明}\\
  \end{center}
\end{minipage}

% --------------------------

\vspace{0.7cm}

% --------------------------

% 博士學位考試委員會置委員五人至九人
% 碩士學位考試委員會置委員三人至五人
% 口試委員人數含指導教授
\begin{minipage}[c][9.0cm][t]{\textwidth}
  \begin{center}
    \makebox[\textwidth][l]{\Large 論文考試委員:}

    \vspace{1.07cm}

% 生成簽名欄位

    \begin{minipage}[c][4.0cm][c]{\textwidth}
      \makebox[0.5\textwidth][c]{\namesigdate}
      \makebox[0.5\textwidth][c]{\namesigdate}
    \end{minipage}
    
    \begin{minipage}[c][4.0cm][c]{\textwidth}
      \makebox[0.5\textwidth][c]{\namesigdate}
      \makebox[0.5\textwidth][c]{\namesigdate}
    \end{minipage}
    
  \end{center}
\end{minipage}

\vspace{1.5cm}

\begin{minipage}[c][3.0cm][t]{\textwidth}
  \begin{center}
    \makebox[4.8em][r]{\Large 指導教授}
    \makebox[1em][c]{\Large:}
    \makebox[7.2em][l]{\namesigdate}

    \vspace{1.0cm}

    \makebox[4.8em][r]{\Large 系(所)主管}
    \makebox[1em][c]{\Large:}
    \makebox[7.2em][l]{\namesigdate}

    \vspace{1.5cm}

    % Date 日期
    \makebox[\textwidth][s]{\Large 中華民國 108 年 6 月 26 日}
  \end{center}
\end{minipage}

\end{center}
\clearpage
\UseDefaultLineStretch
\ClearWatermarkStyle
\UseWatermarkFigureStyle

% ------------------------------------------------
%                  英文口試同意書
% ------------------------------------------------

\singlespacing
\newpage
\phantomsection
\thispagestyle{empty}
\begin{center}

\begin{minipage}[c][3.5cm][t]{\textwidth}
  \begin{center}
    \vspace{0.4cm}

    \makebox[\textwidth][c]{\parbox{\paperwidth}{\center \Large Thesis English Name}}

    \vspace{0.5cm}

    \makebox[\textwidth][c]{\Large by}

    \vspace{0.5cm}

    \makebox[\textwidth][c]{\Large Your Name}
  \end{center}
\end{minipage}

\vspace{1.2cm}

%---------------------------------------------
% 博士與碩士的口試同意書內容不同
%---------------------------------------------
% 博士 -
% Submitted in partial fulfillment of the requirements
% for the degree of Doctor of Philosophy in
% 碩士 -
% A thesis submitted to the graduate division in partial fulfillment
% of the requirements for the degree of Master of Science in

% 顯示 校名, 系所名, 論文種類
\begin{minipage}[c][5cm][t]{\textwidth}
  \begin{center}\Large A thesis submitted to the graduate division in partial fulfillment
  of the requirements for the degree of Master of Science in\\
  Institute of Computer Science and Information Engineering\\ %系所
  College of Electrical Engineering and Computer Science\\ %學院
  National Cheng Kung University\\ %學校
  Tainan, Taiwan, R.O.C.\\
  \vspace{0.1cm}
  26 \thinspace \thinspace Jun \thinspace \thinspace 2019
  \end{center}
\end{minipage}

% ------------------------------------------------

\vspace{0.8cm}

% ------------------------------------------------
\begin{minipage}[c][9.0cm][t]{\textwidth}
  \makebox[\textwidth][l]{\Large Approved by:} \\

  \vspace{0.48cm}

  % 生成簽名欄位

    \begin{minipage}[c][4.0cm][c]{\textwidth}
      \makebox[0.5\textwidth][c]{\namesigdate}
      \makebox[0.5\textwidth][c]{\namesigdate}
    \end{minipage}
    
    \begin{minipage}[c][4.0cm][c]{\textwidth}
      \makebox[0.5\textwidth][c]{\namesigdate}
      \makebox[0.5\textwidth][c]{\namesigdate}
    \end{minipage}
    
\end{minipage}

\vspace{1.5cm}

\begin{minipage}[c][2.0cm][t]{\textwidth}
  \begin{center}
    \makebox[4.8em][r]{\Large Advisor}
    \makebox[1em][c]{\Large:}
    \makebox[7.2em][l]{\namesigdate}\\

    \vspace{1.0cm}

    \makebox[4.8em][r]{\Large Chairman}
    \makebox[1em][c]{\Large:}
    \makebox[7.2em][l]{\namesigdate}\\
  \end{center}
\end{minipage}

\end{center}
\clearpage
\UseDefaultLineStretch
\ClearWatermarkStyle
\UseWatermarkFigureStyle

\end{document}
}

% ----------------------------------------------------------------------------

%
% This file is part of the project of
% National Cheng Kung University (NCKU) Thesis/Dissertation Template in LaTex.
% This project is hold at
%     <https://github.com/wengan-li/ncku-thesis-template-latex>
% by Wen-Gan Li.
%
% This project is distributed in the hope of usefuling to someone,
% you can redistribute it and/or modify it under the terms of the
% Attribution-NonCommercial-ShareAlike 4.0 International.
%
% You should have received a copy of the
% Attribution-NonCommercial-ShareAlike 4.0 International
% along with this project.
% If not, see <http://creativecommons.org/licenses/by-nc-sa/4.0/legalcode.txt>.
%
% Please feel free to fork it, modify it, and try it.
% Have fun !!!
%

% Some helper function about equation

% ----------------------------------------------------------------------------

\newcommand{\SetEquationLabel}[1]
{
  \SetImageLabel{#1}
} % End of \newcommand{}

\DeclareDocumentCommand{\EquationBegin}{G{\empty}}
{
  \begin{equation}
  \SetEquationLabel{#1}
  \begin{aligned}
} % End of \newcommand{}

\newcommand{\EquationEnd}
{
  \end{aligned}
  \end{equation}
} % End of \newcommand{}

% ----------------------------------------------------------------------------

%
% This file is part of the project of
% National Cheng Kung University (NCKU) Thesis/Dissertation Template in LaTex.
% This project is hold at
%     <https://github.com/wengan-li/ncku-thesis-template-latex>
% by Wen-Gan Li.
%
% This project is distributed in the hope of usefuling to someone,
% you can redistribute it and/or modify it under the terms of the
% Attribution-NonCommercial-ShareAlike 4.0 International.
%
% You should have received a copy of the
% Attribution-NonCommercial-ShareAlike 4.0 International
% along with this project.
% If not, see <http://creativecommons.org/licenses/by-nc-sa/4.0/legalcode.txt>.
%
% Please feel free to fork it, modify it, and try it.
% Have fun !!!
%

% Some helper function for reference

% ----------------------------------------------------------------------------

% 為了能連同顯示的內容都能控制, 故做多一層command來包
% Implatment a custom '\label' to have more control
% \LabelThisAs{ < label_name >}{ < format/display_value > }

\makeatletter
\newcommand{\LabelThisAs}[2]
{%
  \def\@currentlabel{#2}%
  \label{#1}%
} % End of \newcommand{}
\makeatother

% ----------------------------------------------------------------------------

% For equation
\newcommand{\RefEquation}[1]{\ref{#1}}

% For equation
\newcommand{\RefEquationB}[1]{\eqref{#1}}

% For bib
\newcommand{\RefBib}[1]{\cite{#1}}

% For figure, table, chapter, section, subsection, .etc
\newcommand{\RefTo}[1]
{%
  \ref{#1}%
} % End of \newcommand{}

\newcommand{\RefFigure}[1]{\RefTo{#1}}

\newcommand{\RefTable}[1]{\RefTo{#1}}

% For page
\newcommand{\RefPage}[1]{\pageref{#1}}
% ----------------------------------------------------------------------------

% 過去的API, 以 Error提醒不能再使用
%\newcommand{\RefTo}{\errmessage{模版: 由v1.4.5開始, RefTo已不再推薦使用. 請改用.}\stop}

%\makeatletter
%\newcommand{\todo}[1][]{\@latex@warning{TODO #1}\fbox{TODO\dots}}
%\makeatother

% ----------------------------------------------------------------------------

%
% This file is part of the project of
% National Cheng Kung University (NCKU) Thesis/Dissertation Template in LaTex.
% This project is hold at
%     <https://github.com/wengan-li/ncku-thesis-template-latex>
% by Wen-Gan Li.
%
% This project is distributed in the hope of usefuling to someone,
% you can redistribute it and/or modify it under the terms of the
% Attribution-NonCommercial-ShareAlike 4.0 International.
%
% You should have received a copy of the
% Attribution-NonCommercial-ShareAlike 4.0 International
% along with this project.
% If not, see <http://creativecommons.org/licenses/by-nc-sa/4.0/legalcode.txt>.
%
% Please feel free to fork it, modify it, and try it.
% Have fun !!!
%

% Some common helper function

% ----------------------------------------------------------------------------

% Common function
\def\AppendKeywordString#1#2{\edef#1{#1#2}}

% --- 關鍵字 Keyword ---
\def\VarPDFKeywords{\empty} % initialize
\def\GetPDFKeywords{\VarPDFKeywords} % initialize
\def\AppendPDFKeyword#1{\AppendKeywordString{\VarPDFKeywords}{#1}}

\DeclareDocumentCommand{\SetKeywords}{
  m G{\empty} G{\empty}
  G{\empty} G{\empty} G{\empty}
  G{\empty} G{\empty} G{\empty}}
{
  \AppendPDFKeyword{#1}
  \ifthenelse{\equal{#2}{\empty}}{}{\AppendPDFKeyword{, #2}}
  \ifthenelse{\equal{#3}{\empty}}{}{\AppendPDFKeyword{, #3}}
  \ifthenelse{\equal{#4}{\empty}}{}{\AppendPDFKeyword{, #4}}
  \ifthenelse{\equal{#5}{\empty}}{}{\AppendPDFKeyword{, #5}}
  \ifthenelse{\equal{#6}{\empty}}{}{\AppendPDFKeyword{, #6}}
  \ifthenelse{\equal{#7}{\empty}}{}{\AppendPDFKeyword{, #7}}
  \ifthenelse{\equal{#8}{\empty}}{}{\AppendPDFKeyword{, #8}}
  \ifthenelse{\equal{#9}{\empty}}{}{\AppendPDFKeyword{, #9}}
} % End of \newcommand{}

\def\VarAbstractChiKeywords{\empty} % initialize
\def\GetAbstractChiKeywords{\VarAbstractChiKeywords} % initialize
\DeclareDocumentCommand{\SetAbstractChiKeywords}{
  m G{\empty} G{\empty}
  G{\empty} G{\empty} G{\empty}
  G{\empty} G{\empty} G{\empty}}
{
  \AppendKeywordString{\VarAbstractChiKeywords}{#1}
  \ifthenelse{\equal{#2}{\empty}}{}{\AppendKeywordString{\VarAbstractChiKeywords}{, #2}}
  \ifthenelse{\equal{#3}{\empty}}{}{\AppendKeywordString{\VarAbstractChiKeywords}{, #3}}
  \ifthenelse{\equal{#4}{\empty}}{}{\AppendKeywordString{\VarAbstractChiKeywords}{, #4}}
  \ifthenelse{\equal{#5}{\empty}}{}{\AppendKeywordString{\VarAbstractChiKeywords}{, #5}}
  \ifthenelse{\equal{#6}{\empty}}{}{\AppendKeywordString{\VarAbstractChiKeywords}{, #6}}
  \ifthenelse{\equal{#7}{\empty}}{}{\AppendKeywordString{\VarAbstractChiKeywords}{, #7}}
  \ifthenelse{\equal{#8}{\empty}}{}{\AppendKeywordString{\VarAbstractChiKeywords}{, #8}}
  \ifthenelse{\equal{#9}{\empty}}{}{\AppendKeywordString{\VarAbstractChiKeywords}{, #9}}
} % End of \newcommand{}

\def\VarAbstractEngKeywords{\empty} % initialize
\def\GetAbstractEngKeywords{\VarAbstractEngKeywords} % initialize
\DeclareDocumentCommand{\SetAbstractEngKeywords}{
  m G{\empty} G{\empty}
  G{\empty} G{\empty} G{\empty}
  G{\empty} G{\empty} G{\empty}}
{
  \AppendKeywordString{\VarAbstractEngKeywords}{#1}
  \ifthenelse{\equal{#2}{\empty}}{}{\AppendKeywordString{\VarAbstractEngKeywords}{, #2}}
  \ifthenelse{\equal{#3}{\empty}}{}{\AppendKeywordString{\VarAbstractEngKeywords}{, #3}}
  \ifthenelse{\equal{#4}{\empty}}{}{\AppendKeywordString{\VarAbstractEngKeywords}{, #4}}
  \ifthenelse{\equal{#5}{\empty}}{}{\AppendKeywordString{\VarAbstractEngKeywords}{, #5}}
  \ifthenelse{\equal{#6}{\empty}}{}{\AppendKeywordString{\VarAbstractEngKeywords}{, #6}}
  \ifthenelse{\equal{#7}{\empty}}{}{\AppendKeywordString{\VarAbstractEngKeywords}{, #7}}
  \ifthenelse{\equal{#8}{\empty}}{}{\AppendKeywordString{\VarAbstractEngKeywords}{, #8}}
  \ifthenelse{\equal{#9}{\empty}}{}{\AppendKeywordString{\VarAbstractEngKeywords}{, #9}}
} % End of \newcommand{}

\def\VarAbstractExtKeywords{\empty} % initialize
\def\GetAbstractExtKeywords{\VarAbstractExtKeywords} % initialize
\DeclareDocumentCommand{\SetAbstractExtKeywords}{
  m G{\empty} G{\empty}
  G{\empty} G{\empty} G{\empty}
  G{\empty} G{\empty} G{\empty}}
{
  \AppendKeywordString{\VarAbstractExtKeywords}{#1}
  \ifthenelse{\equal{#2}{\empty}}{}{\AppendKeywordString{\VarAbstractExtKeywords}{, #2}}
  \ifthenelse{\equal{#3}{\empty}}{}{\AppendKeywordString{\VarAbstractExtKeywords}{, #3}}
  \ifthenelse{\equal{#4}{\empty}}{}{\AppendKeywordString{\VarAbstractExtKeywords}{, #4}}
  \ifthenelse{\equal{#5}{\empty}}{}{\AppendKeywordString{\VarAbstractExtKeywords}{, #5}}
  \ifthenelse{\equal{#6}{\empty}}{}{\AppendKeywordString{\VarAbstractExtKeywords}{, #6}}
  \ifthenelse{\equal{#7}{\empty}}{}{\AppendKeywordString{\VarAbstractExtKeywords}{, #7}}
  \ifthenelse{\equal{#8}{\empty}}{}{\AppendKeywordString{\VarAbstractExtKeywords}{, #8}}
  \ifthenelse{\equal{#9}{\empty}}{}{\AppendKeywordString{\VarAbstractExtKeywords}{, #9}}
} % End of \newcommand{}

% ----------------------------------------------------------------------------

%
% This file is part of the project of
% National Cheng Kung University (NCKU) Thesis/Dissertation Template in LaTex.
% This project is hold at
%     <https://github.com/wengan-li/ncku-thesis-template-latex>
% by Wen-Gan Li.
%
% This project is distributed in the hope of usefuling to someone,
% you can redistribute it and/or modify it under the terms of the
% Attribution-NonCommercial-ShareAlike 4.0 International.
%
% You should have received a copy of the
% Attribution-NonCommercial-ShareAlike 4.0 International
% along with this project.
% If not, see <http://creativecommons.org/licenses/by-nc-sa/4.0/legalcode.txt>.
%
% Please feel free to fork it, modify it, and try it.
% Have fun !!!
%

% Some common helper function

% ----------------------------------------------------------------------------

% 使用 hyperref 在 pdf 簡介欄裡填入相關資料
\newcommand{\FillInPDFData}
{
  \ifx \hypersetup \undefined
    % do nothing
    \relax
  \else
    \ifx \GetChiTitle \undefined
      \hypersetup
      {
        pdftitle  = {\GetEngTitle},
        pdfauthor = {\GetAuthorEngName},
      }
    \else
      \hypersetup
      {
        pdftitle  = {\GetEngTitle\ (\GetChiTitle)},
        pdfauthor = {\GetAuthorEngName\ (\GetAuthorChiName)},
      }
    \fi

    \hypersetup
    {
      unicode     = true,
      pdfcreator  = {\GetUniversityEngName},
%      pdfproducer = {\GetUniversityEngName},
      pdfsubject  = {},
    }

    \ifthenelse{\equal{\GetPDFKeywords}{\empty}}{}{%
      \hypersetup{pdfkeywords = {\GetPDFKeywords}}}
  \fi
} % End of \newcommand{}

% ----------------------------------------------------------------------------

%
% This file is part of the project of
% National Cheng Kung University (NCKU) Thesis/Dissertation Template in LaTex.
% This project is hold at
%     <https://github.com/wengan-li/ncku-thesis-template-latex>
% by Wen-Gan Li.
%
% This project is distributed in the hope of usefuling to someone,
% you can redistribute it and/or modify it under the terms of the
% Attribution-NonCommercial-ShareAlike 4.0 International.
%
% You should have received a copy of the
% Attribution-NonCommercial-ShareAlike 4.0 International
% along with this project.
% If not, see <http://creativecommons.org/licenses/by-nc-sa/4.0/legalcode.txt>.
%
% Please feel free to fork it, modify it, and try it.
% Have fun !!!
%

% ----------------------------------------------------------------------------

% Some helper function about font

% Reference from
% <http://texdoc.net/texmf-dist/doc/latex/fontspec/fontspec.pdf>

% ----------------------------------------------------------------------------

% Type TimesKaiu
% Eng: Times New Roman
% Chi: 標楷體
\def \VarFontTypeTimesKaiuEngFileNameNormal {times.ttf}
\def \VarFontTypeTimesKaiuEngFileNameItalic {timesi.ttf}
\def \VarFontTypeTimesKaiuEngFileNameBold {timesbd.ttf}
\def \VarFontTypeTimesKaiuEngFileNameBoldItalic {timesbi.ttf}
\def \VarFontTypeTimesKaiuChiFileNameNormal {kaiu.ttf}

\newcommand{\VarFontTypeTimesKaiu}{0}

% -------------------------------------------

% Type Noto Sans CJK (Eng & Chi)
\def \VarFontTypeNotoSansCJKEngFileNameNormal {NotoSansCJKtc-Medium.otf}
\def \VarFontTypeNotoSansCJKEngFileNameBold {NotoSansCJKtc-Bold.otf}
\def \VarFontTypeNotoSansCJKChiFileNameNormal {NotoSansCJKtc-Medium.otf}
\def \VarFontTypeNotoSansCJKChiFileNameBold {NotoSansCJKtc-Bold.otf}

\newcommand{\VarFontTypeNotoSansCJK}{1}

% -------------------------------------------

% Custom type
% Font files is needed to be provided
% Default it is the Type TimesKaiu
\def \VarFontTypeCustomEngFileNameNormal {times.ttf} % Default
\def \VarFontTypeCustomEngFileNameItalic {timesi.ttf} % Default
\def \VarFontTypeCustomEngFileNameBold {timesbd.ttf} % Default
\def \VarFontTypeCustomEngFileNameBoldItalic {timesbi.ttf} % Default
\def \VarFontTypeCustomChiFileNameNormal {kaiu.ttf} % Default
\def \VarFontTypeCustomChiFileNameItalic {} % Default
\def \VarFontTypeCustomChiFileNameBold {} % Default
\def \VarFontTypeCustomChiFileNameBoldItalic {} % Default

\pgfkeys
{
  /ParseCustomFontFiles/.is family, /ParseCustomFontFiles,
  default/.style =
  {
    NormalFont = \empty,
    ItalicFont = \empty,
    BoldFont = \empty,
    BoldItalicFont = \empty,
  },
  NormalFont/.estore in = \TmpValueNormalFont,
  ItalicFont/.estore in = \TmpValueItalicFont,
  BoldFont/.estore in = \TmpValueBoldFont,
  BoldItalicFont/.estore in = \TmpValueBoldItalicFont,
} % End of \pgfkeys{}

\newcommand{\SetCustomEngFontFiles}[1][\empty]
{%
  \SetFontUseType{\VarFontTypeCustom}
  %
  % Parse the input
  \pgfkeys{/ParseCustomFontFiles, default, #1}%
  %
  \ifthenelse{\equal{\TmpValueNormalFont}{\empty}}{}{%
    \renewcommand{\VarFontTypeCustomEngFileNameNormal}{%
        \TmpValueNormalFont}}%
  \ifthenelse{\equal{\TmpValueItalicFont}{\empty}}{}{%
    \renewcommand{\VarFontTypeCustomEngFileNameItalic}{%
        \TmpValueItalicFont}}%
  \ifthenelse{\equal{\TmpValueBoldFont}{\empty}}{}{%
    \renewcommand{\VarFontTypeCustomEngFileNameBold}{%
        \TmpValueBoldFont}}%
  \ifthenelse{\equal{\TmpValueBoldItalicFont}{\empty}}{}{%
    \renewcommand{\VarFontTypeCustomEngFileNameBoldItalic}{%
        \TmpValueBoldItalicFont}}%
} % End of \newcommand{}

\newcommand{\SetCustomChiFontFiles}[1][\empty]
{%
  \SetFontUseType{\VarFontTypeCustom}
  %
  % Parse the input
  \pgfkeys{/ParseCustomFontFiles, default, #1}%
  %
  \ifthenelse{\equal{\TmpValueNormalFont}{\empty}}{}{%
    \renewcommand{\VarFontTypeCustomChiFileNameNormal}{%
        \TmpValueNormalFont}}%
  \ifthenelse{\equal{\TmpValueItalicFont}{\empty}}{}{%
    \renewcommand{\VarFontTypeCustomChiFileNameItalic}{%
        \TmpValueItalicFont}}%
  \ifthenelse{\equal{\TmpValueBoldFont}{\empty}}{}{%
    \renewcommand{\VarFontTypeCustomChiFileNameBold}{%
        \TmpValueBoldFont}}%
  \ifthenelse{\equal{\TmpValueBoldItalicFont}{\empty}}{}{%
    \renewcommand{\VarFontTypeCustomChiFileNameBoldItalic}{%
        \TmpValueBoldItalicFont}}%
} % End of \newcommand{}

\newcommand{\VarFontTypeCustom}{10}

% -------------------------------------------

\def \VarFontDirPath {./configure/fonts/} % Fix font path
\def \GetFontDirPath {\VarFontDirPath}

\newcommand{\VarFontUseType}{\VarFontTypeTimesKaiu} % Default
\newcommand{\GetFontUseType}{\VarFontUseType}
\newcommand{\SetFontUseType}[1]
{%
  \renewcommand{\VarFontUseType}{#1}%
} % End of \newcommand{}

% -------------------------------------------

\pgfkeys
{
  /ParseFontOption/.is family, /ParseFontOption,
  default/.style =
  {
    NormalFont = \empty,
    ItalicFont = \empty,
    BoldFont = \empty,
    BoldItalicFont = \empty,
  },
  NormalFont/.estore in = \TmpValueNormalFont,
  ItalicFont/.estore in = \TmpValueItalicFont,
  BoldFont/.estore in = \TmpValueBoldFont,
  BoldItalicFont/.estore in = \TmpValueBoldItalicFont,
} % End of \pgfkeys{}

\newcommand{\SetEngMainFont}[2][\empty]
{%
  % Parse the input
  \pgfkeys{/ParseFontOption, default, #1}%
  %
  \defaultfontfeatures[#2]{%
    Path = \GetFontDirPath,
    UprightFont = \TmpValueNormalFont
  }%
  %
  \ifthenelse{\equal{\TmpValueItalicFont}{\empty}}{}{%
    \defaultfontfeatures+[#2]{%
      ItalicFont = \TmpValueItalicFont}}%
  \ifthenelse{\equal{\TmpValueBoldFont}{\empty}}{}{%
    \defaultfontfeatures+[#2]{%
      BoldFont = \TmpValueBoldFont}}%
  \ifthenelse{\equal{\TmpValueBoldItalicFont}{\empty}}{}{%
    \defaultfontfeatures+[#2]{%
      BoldItalicFont = \TmpValueBoldItalicFont}}%
  %
  \setmainfont{#2}
} % End of \newcommand{}

\newcommand{\SetChiMainFont}[2][\empty]
{%
  % Parse the input
  \pgfkeys{/ParseFontOption, default, #1}%
  %
  \setCJKmainfont[%
    Path = \GetFontDirPath,
    UprightFont = \TmpValueNormalFont,
    AutoFakeBold = true,
    AutoFakeSlant = true,
  ]{#2}%
  %
  \ifthenelse{\equal{\TmpValueItalicFont}{\empty}}{}{%
    \addCJKfontfeatures*{ItalicFont = \TmpValueItalicFont}}%
  \ifthenelse{\equal{\TmpValueBoldFont}{\empty}}{}{%
    \addCJKfontfeatures*{BoldFont = \TmpValueBoldFont}}%
  \ifthenelse{\equal{\TmpValueBoldItalicFont}{\empty}}{}{%
    \addCJKfontfeatures*{BoldItalicFont = \TmpValueBoldItalicFont}}%
  %
  %\setCJKmathfont{#2}%
} % End of \newcommand{}

\newcommand{\SetEngFontFamily}[1]
{%
  \fontspec{#1}
} % End of \newcommand{}

\newcommand{\SetChiFontFamily}[1]
{%
  \CJKfontspec{#1}%
} % End of \newcommand{}

% -------------------------------------------

\def \UseFontStyleTimesKaiu
{
  \SetEngFontFamily{TimesKaiuEngFont}%
  \SetChiFontFamily{TimesKaiuChiFont}%
} % End of \newcommand{}

\def \InitFontStyleTimesKaiu
{
  \SetEngMainFont[%
    NormalFont = \VarFontTypeTimesKaiuEngFileNameNormal,%
    ItalicFont = \VarFontTypeTimesKaiuEngFileNameItalic,%
    BoldFont = \VarFontTypeTimesKaiuEngFileNameBold,%
    BoldItalicFont = \VarFontTypeTimesKaiuEngFileNameBoldItalic,%
    ]{TimesKaiuEngFont}%
  \SetChiMainFont[%
    NormalFont = \VarFontTypeTimesKaiuChiFileNameNormal,%
    ]{TimesKaiuChiFont}%
} % End of \newcommand{}

% -------------------------------------------

\def \UseFontStyleNotoSansCJK
{
  \SetEngFontFamily{NotoSansCJKEngFont}%
  \SetChiFontFamily{NotoSansCJKChiFont}%
} % End of \newcommand{}

\def \InitFontStyleNotoSansCJK
{
  \SetEngMainFont[%
    NormalFont = \VarFontTypeNotoSansCJKEngFileNameNormal,%
    BoldFont = \VarFontTypeNotoSansCJKEngFileNameBold,%
    ]{NotoSansCJKEngFont}%
  \SetChiMainFont[%
    NormalFont = \VarFontTypeNotoSansCJKEngFileNameNormal,%
    BoldFont = \VarFontTypeNotoSansCJKEngFileNameBold,%
    ]{NotoSansCJKChiFont}%
} % End of \newcommand{}

% -------------------------------------------

\def \UseFontStyleCustom
{
  \SetEngFontFamily{CustomEngFont}%
  \SetChiFontFamily{CustomChiFont}%
} % End of \newcommand{}

\def \InitFontStyleCustom
{
  \SetEngMainFont[%
    NormalFont = \VarFontTypeCustomEngFileNameNormal,%
    ItalicFont = \VarFontTypeCustomEngFileNameItalic,%
    BoldFont = \VarFontTypeCustomEngFileNameBold,%
    BoldItalicFont = \VarFontTypeCustomEngFileNameBoldItalic,%
    ]{CustomEngFont}%
  \SetChiMainFont[%
    NormalFont = \VarFontTypeCustomChiFileNameNormal,%
    ItalicFont = \VarFontTypeCustomChiFileNameItalic,%
    BoldFont = \VarFontTypeCustomChiFileNameBold,%
    BoldItalicFont = \VarFontTypeCustomChiFileNameBoldItalic,%
    ]{CustomChiFont}%
} % End of \newcommand{}

% -------------------------------------------

\newcommand{\UseDefaultFontType}
{%
  \if \GetFontUseType \VarFontTypeTimesKaiu%
    \UseFontStyleTimesKaiu%
  \fi%
%
  \if \GetFontUseType \VarFontTypeNotoSansCJK%
    \UseFontStyleNotoSansCJK%
  \fi%

  \if \GetFontUseType \VarFontTypeCustom%
    \UseFontStyleCustom%
  \fi%
} % End of \newcommand{}

\newcommand{\InitDefaultFontType}
{%
  \if \GetFontUseType \VarFontTypeTimesKaiu%
    \InitFontStyleTimesKaiu%
  \fi%
%
  \if \GetFontUseType \VarFontTypeNotoSansCJK%
    \InitFontStyleNotoSansCJK%
  \fi%

  \if \GetFontUseType \VarFontTypeCustom%
    \InitFontStyleCustom%
  \fi%
} % End of \newcommand{}

% -------------------------------------------

%\setmathrm{hfont namei}[hfont featuresi]
%\setmathsf{hfont namei}[hfont featuresi]
%\setmathtt{hfont namei}[hfont featuresi]
%\setboldmathrm{hfont namei}[hfont featuresi]

% ----------------------------------------------------------------------------

% 原本提供用來給自行設定字型,
% 但發現轉換字型時好像有問題,
% 加入成TODO List

%\UseFontStyleTimesKaiu
%\UseFontStyleNotoSansCJK
%\UseFontStyleCustom

% .ttf/.otf
%\SetCustomEngFontFiles[%
%    NormalFont = custom_font.ttf,%
%    ItalicFont = custom_font.ttf,%
%    BoldFont = custom_font.ttf,%
%    BoldItalicFont = custom_font.ttf,%
%    ]%
%\SetCustomChiFontFiles[%
%    NormalFont = custom_font.ttf,%
%    ItalicFont = custom_font.ttf,%
%    BoldFont = custom_font.ttf,%
%    BoldItalicFont = custom_font.ttf,%
%    ]%

% ----------------------------------------------------------------------------

%
% This file is part of the project of
% National Cheng Kung University (NCKU) Thesis/Dissertation Template in LaTex.
% This project is hold at
%     <https://github.com/wengan-li/ncku-thesis-template-latex>
% by Wen-Gan Li.
%
% This project is distributed in the hope of usefuling to someone,
% you can redistribute it and/or modify it under the terms of the
% Attribution-NonCommercial-ShareAlike 4.0 International.
%
% You should have received a copy of the
% Attribution-NonCommercial-ShareAlike 4.0 International
% along with this project.
% If not, see <http://creativecommons.org/licenses/by-nc-sa/4.0/legalcode.txt>.
%
% Please feel free to fork it, modify it, and try it.
% Have fun !!!
%

% Some helper function use for counter

% ----------------------------------------------------------------------------

%    tiangan (天干: 甲乙丙丁戊癸, 正常範圍是 1–10, 超出範圍的數位將輸出空值)
%    arabic (阿拉伯數字)
%    roman (小寫的羅馬數字)
%    Roman (大寫的羅馬數字)
%    alph (小寫字母)
%    Alph (大寫字母)

%    計數器名  (用途)
%    part 	(部序號)
%    chapter 	(章序號)
%    section 	(節)
%    subsection 	(小節)
%    subsubsection 	(小小節)
%    paragraph 	(段)
%    subparagraph 	(小段)
%    figure 	(插圖序號)
%    table 	(表格序號)
%    equation 	(公式序號)
%    page 	(頁碼計數器)
%    footnote 	(註腳序號)
%    mpfootnote 	(小頁環境中註腳計數器)

% ----------------------------------------------------------------------------

\pgfkeys
{
  /SetupTitleNumberFormatString/.is family, /SetupTitleNumberFormatString,
  default/.style =
  {
    BeginText = \empty,
    EndText = \empty,
    CNumStyle = \empty,
    CCounterName = \empty,
    SNumStyle = \empty,
    SCounterName = \empty,
    SSNumStyle = \empty,
    SSCounterName = \empty,
    SSSNumStyle = \empty,
    SSSCounterName = \empty,
    SepAtIndex = \empty,
    SepBetweenCnS = \empty,
    SepBetweenSnSS = \empty,
    SepBetweenSSCnSSS = \empty,
  },
  BeginText/.estore in = \TmpValueBeginText,
  EndText/.estore in = \TmpValueEndText,
  CNumStyle/.estore in = \TmpValueCNumStyle,
  CCounterName/.estore in = \TmpValueCCounterName,
  SNumStyle/.estore in = \TmpValueSNumStyle,
  SCounterName/.estore in = \TmpValueSCounterName,
  SSNumStyle/.estore in = \TmpValueSSNumStyle,
  SSCounterName/.estore in = \TmpValueSSCounterName,
  SSSNumStyle/.estore in = \TmpValueSSSNumStyle,
  SSSCounterName/.estore in = \TmpValueSSSCounterName,
  SepAtIndex/.estore in = \TmpValueSepAtIndex,
  SepBetweenCnS/.estore in = \TmpValueSepBetweenCnS,
  SepBetweenSnSS/.estore in = \TmpValueSepBetweenSnSS,
  SepBetweenSSCnSSS/.estore in = \TmpValueSepBetweenSSCnSSS,
} % End of \pgfkeys{}

% Append counter value in style to string
\newcommand\AppendCounterStringToFormatString[3]
{%
  \ifthenelse{\equal{#2}{ChiNum}}{%
    \appto#1{\zhnum{#3}}}{}%
  \ifthenelse{\equal{#2}{Tiangan}}{%
    \appto#1{\zhtiangan{\value{#3}}}}{}%
  \ifthenelse{\equal{#2}{Arabic}}{%
    \appto#1{\arabic{#3}}}{}%
  \ifthenelse{\equal{#2}{LowerRoman}}{%
    \appto#1{\roman{#3}}}{}%
  \ifthenelse{\equal{#2}{UpperRoman}}{%
    \appto#1{\Roman{#3}}}{}%
  \ifthenelse{\equal{#2}{LowerAlph}}{%
    \appto#1{\alph{#3}}}{}%
  \ifthenelse{\equal{#2}{UpperAlph}}{%
    \appto#1{\Alph{#3}}}{}%
} % End of \newcommand{}

\newcommand\SetupTitleNumberFormatString[3]
{%
  \pgfkeys{/SetupTitleNumberFormatString, default, #2}%
  \renewcommand#3{}%
  \appto#3{\TmpValueBeginText}%
  \ifboolexpr{%
    test {\ifstrequal{#1}{Chapter}} or %
    test {\ifstrequal{#1}{AppendixChapter}} }
      {%
        \AppendCounterStringToFormatString{%
          #3}{\TmpValueCNumStyle}{\TmpValueCCounterName}%
      }{}%
  %
  \ifboolexpr{%
    test {\ifstrequal{#1}{Section}} or %
    test {\ifstrequal{#1}{AppendixSection}} }
      {%
        \AppendCounterStringToFormatString{%
          #3}{\TmpValueCNumStyle}{\TmpValueCCounterName}%
        \appto#3{\TmpValueSepBetweenCnS}%
        \AppendCounterStringToFormatString{%
          #3}{\TmpValueSNumStyle}{\TmpValueSCounterName}%
      }{}%
  %
  \ifboolexpr{%
    test {\ifstrequal{#1}{SubSection}} or %
    test {\ifstrequal{#1}{AppendixSubSection}} }
      {%
        \AppendCounterStringToFormatString{%
          #3}{\TmpValueCNumStyle}{\TmpValueCCounterName}%
        \appto#3{\TmpValueSepBetweenCnS}%
        \AppendCounterStringToFormatString{%
          #3}{\TmpValueSNumStyle}{\TmpValueSCounterName}%
        \appto#3{\TmpValueSepBetweenSnSS}%
        \AppendCounterStringToFormatString{%
          #3}{\TmpValueSSNumStyle}{\TmpValueSSCounterName}%
      }{}%
  %
  \ifboolexpr{%
    test {\ifstrequal{#1}{SubSubSection}} or %
    test {\ifstrequal{#1}{AppendixSubSubSection}} }
      {%
        \AppendCounterStringToFormatString{%
          #3}{\TmpValueCNumStyle}{\TmpValueCCounterName}%
        \appto#3{\TmpValueSepBetweenCnS}%
        \AppendCounterStringToFormatString{%
          #3}{\TmpValueSNumStyle}{\TmpValueSCounterName}%
        \appto#3{\TmpValueSepBetweenSnSS}%
        \AppendCounterStringToFormatString{%
          #3}{\TmpValueSSNumStyle}{\TmpValueSSCounterName}%
        \appto#3{\TmpValueSepBetweenSSCnSSS}%
        \AppendCounterStringToFormatString{%
          #3}{\TmpValueSSSNumStyle}{\TmpValueSSSCounterName}%
      }{}%
  \appto#3{\TmpValueEndText}%
} % End of \newcommand{}

% ---------------------------

\pgfkeys
{
  /CTitleNumberFormat/.is family, /CTitleNumberFormat,
  default/.style =
  {
    BeginText = {Chapter },
    EndText = \empty,
    TextAlign = {Left}, % Useless for Chapter
    CNumStyle = Arabic,
    SepAtIndex = {.},
  },
  BeginText/.estore in = \GetCTitleNumberFormatBeginText,
  EndText/.estore in = \GetCTitleNumberFormatEndText,
  TextAlign/.estore in = \GetCTitleNumberFormatTextAlign,
  CNumStyle/.estore in = \GetCTitleNumberFormatCNumStyle,
  SepAtIndex/.estore in = \GetCTitleNumberFormatSepAtIndex,
} % End of \pgfkeys{}

\newcommand\GetChapterTitleNumberFormatString{}
\newcommand\SetupChapterTitleNumberFormatString
{%
  \SetupTitleNumberFormatString{Chapter}%
  {%
    BeginText=\GetCTitleNumberFormatBeginText,%
    EndText=\GetCTitleNumberFormatEndText,%
    CNumStyle=\GetCTitleNumberFormatCNumStyle,%
    CCounterName=chapter,%
  }{\GetChapterTitleNumberFormatString}%
} % End of \newcommand{}

% ---------------------------

\pgfkeys
{
  /STitleNumberFormat/.is family, /STitleNumberFormat,
  default/.style =
  {
    BeginText = \empty,
    EndText = \empty,
    TextAlign = {Left},
    CNumStyle = Arabic,
    SNumStyle = Arabic,
    SepAtIndex = {.}, % 目錄中章節號碼跟章節題目中的分隔符號
    SepBetweenCnS = {.}, % 章號碼跟節號碼中的分隔符號
  },
  BeginText/.estore in = \GetSTitleNumberFormatBeginText,
  EndText/.estore in = \GetSTitleNumberFormatEndText,
  TextAlign/.estore in = \GetSTitleNumberFormatTextAlign,
  CNumStyle/.estore in = \GetSTitleNumberFormatCNumStyle,
  SNumStyle/.estore in = \GetSTitleNumberFormatSNumStyle,
  SepAtIndex/.estore in = \GetSTitleNumberFormatSepAtIndex,
  SepBetweenCnS/.estore in = \GetSTitleNumberFormatSepBetweenCnS,
} % End of \pgfkeys{}

\newcommand\GetSectionTitleNumberFormatString{}
\newcommand\SetupSectionTitleNumberFormatString
{%
  \SetupTitleNumberFormatString{Section}%
  {%
    BeginText=\GetSTitleNumberFormatBeginText,%
    EndText=\GetSTitleNumberFormatEndText,%
    CNumStyle=\GetSTitleNumberFormatCNumStyle,%
    SNumStyle=\GetSTitleNumberFormatSNumStyle,%
    SepAtIndex=\GetSTitleNumberFormatSepAtIndex,%
    SepBetweenCnS=\GetSTitleNumberFormatSepBetweenCnS,%
    CCounterName=chapter,%
    SCounterName=section,%
  }{\GetSectionTitleNumberFormatString}%
} % End of \newcommand{}

% ---------------------------

\pgfkeys
{
  /SSTitleNumberFormat/.is family, /SSTitleNumberFormat,
  default/.style =
  {
    BeginText = \empty,
    EndText = \empty,
    TextAlign = {Left},
    CNumStyle = Arabic,
    SNumStyle = Arabic,
    SSNumStyle = Arabic,
    SepAtIndex = {.}, % 目錄中章節號碼跟章節題目中的分隔符號
    SepBetweenCnS = {.}, % 章號碼跟節號碼中的分隔符號
    SepBetweenSnSS = {.}, % 節號碼跟小節號碼中的分隔符號
  },
  BeginText/.estore in = \GetSSTitleNumberFormatBeginText,
  EndText/.estore in = \GetSSTitleNumberFormatEndText,
  TextAlign/.estore in = \GetSSTitleNumberFormatTextAlign,
  CNumStyle/.estore in = \GetSSTitleNumberFormatCNumStyle,
  SNumStyle/.estore in = \GetSSTitleNumberFormatSNumStyle,
  SSNumStyle/.estore in = \GetSSTitleNumberFormatSSNumStyle,
  SepAtIndex/.estore in = \GetSSTitleNumberFormatSepAtIndex,
  SepBetweenCnS/.estore in = \GetSSTitleNumberFormatSepBetweenCnS,
  SepBetweenSnSS/.estore in = \GetSSTitleNumberFormatSepBetweenSnSS,
} % End of \pgfkeys{}

\newcommand\GetSubSectionTitleNumberFormatString{}
\newcommand\SetupSubSectionTitleNumberFormatString
{%
  \SetupTitleNumberFormatString{SubSection}%
  {%
    BeginText=\GetSSTitleNumberFormatBeginText,%
    EndText=\GetSSTitleNumberFormatEndText,%
    CNumStyle=\GetSSTitleNumberFormatCNumStyle,%
    SNumStyle=\GetSSTitleNumberFormatSNumStyle,%
    SSNumStyle=\GetSSTitleNumberFormatSSNumStyle,%
    SepAtIndex=\GetSSTitleNumberFormatSepAtIndex,%
    SepBetweenCnS=\GetSSTitleNumberFormatSepBetweenCnS,%
    SepBetweenSnSS=\GetSSTitleNumberFormatSepBetweenSnSS,%
    CCounterName=chapter,%
    SCounterName=section,%
    SSCounterName=subsection,%
  }{\GetSubSectionTitleNumberFormatString}%
} % End of \newcommand{}

% ---------------------------

\pgfkeys
{
  /SSSTitleNumberFormat/.is family, /SSSTitleNumberFormat,
  default/.style =
  {
    BeginText = \empty,
    EndText = \empty,
    TextAlign = {Left},
    CNumStyle = Arabic,
    SNumStyle = Arabic,
    SSNumStyle = Arabic,
    SSSNumStyle = Arabic,
    SepAtIndex = {.}, % 目錄中章節號碼跟章節題目中的分隔符號
    SepBetweenCnS = {.}, % 章號碼跟節號碼中的分隔符號
    SepBetweenSnSS = {.}, % 節號碼跟小節號碼中的分隔符號
    SepBetweenSSCnSSS = {.}, % 小節號碼跟小小節號碼中的分隔符號
  },
  BeginText/.estore in = \GetSSSTitleNumberFormatBeginText,
  EndText/.estore in = \GetSSSTitleNumberFormatEndText,
  TextAlign/.estore in = \GetSSSTitleNumberFormatTextAlign,
  CNumStyle/.estore in = \GetSSSTitleNumberFormatCNumStyle,
  SNumStyle/.estore in = \GetSSSTitleNumberFormatSNumStyle,
  SSNumStyle/.estore in = \GetSSSTitleNumberFormatSSNumStyle,
  SSSNumStyle/.estore in = \GetSSSTitleNumberFormatSSSNumStyle,
  SepAtIndex/.estore in = \GetSSSTitleNumberFormatSepAtIndex,
  SepBetweenCnS/.estore in = \GetSSSTitleNumberFormatSepBetweenCnS,
  SepBetweenSnSS/.estore in = \GetSSSTitleNumberFormatSepBetweenSnSS,
  SepBetweenSSCnSSS/.estore in = \GetSSSTitleNumberFormatSepBetweenSSCnSSS,
} % End of \pgfkeys{}

\newcommand\GetSubSubSectionTitleNumberFormatString{}
\newcommand\SetupSubSubSectionTitleNumberFormatString
{%
  \SetupTitleNumberFormatString{SubSubSection}%
  {%
    BeginText=\GetSSSTitleNumberFormatBeginText,%
    EndText=\GetSSSTitleNumberFormatEndText,%
    CNumStyle=\GetSSSTitleNumberFormatCNumStyle,%
    SNumStyle=\GetSSSTitleNumberFormatSNumStyle,%
    SSNumStyle=\GetSSSTitleNumberFormatSSNumStyle,%
    SSSNumStyle=\GetSSSTitleNumberFormatSSSNumStyle,%
    SepAtIndex=\GetSSSTitleNumberFormatSepAtIndex,%
    SepBetweenCnS=\GetSSSTitleNumberFormatSepBetweenCnS,%
    SepBetweenSnSS=\GetSSSTitleNumberFormatSepBetweenSnSS,%
    SepBetweenSSCnSSS=\GetSSSTitleNumberFormatSepBetweenSSCnSSS,%
    CCounterName=chapter,%
    SCounterName=section,%
    SSCounterName=subsection,%
    SSSCounterName=subsubsection,%
  }{\GetSubSubSectionTitleNumberFormatString}%
} % End of \newcommand{}

% ---------------------------

\pgfkeys
{
  /AppendixCTitleNumberFormat/.is family, /AppendixCTitleNumberFormat,
  default/.style =
  {
    BeginText = {Appendix },
    EndText = \empty,
    TextAlign = {Left}, % Useless for Chapter
    CNumStyle = UpperAlph,
    SepAtIndex = {.},
  },
  BeginText/.estore in = \GetAppendixCTitleNumberFormatBeginText,
  EndText/.estore in = \GetAppendixCTitleNumberFormatEndText,
  TextAlign/.estore in = \GetAppendixCTitleNumberFormatTextAlign,
  CNumStyle/.estore in = \GetAppendixCTitleNumberFormatCNumStyle,
  SepAtIndex/.estore in = \GetAppendixCTitleNumberFormatSepAtIndex,
} % End of \pgfkeys{}

\newcommand\GetAppendixChapterTitleNumberFormatString{}
\newcommand\SetupAppendixChapterTitleNumberFormatString
{%
  \SetupTitleNumberFormatString{AppendixChapter}%
  {%
    BeginText=\GetAppendixCTitleNumberFormatBeginText,%
    EndText=\GetAppendixCTitleNumberFormatEndText,%
    CNumStyle=\GetAppendixCTitleNumberFormatCNumStyle,%
    CCounterName=appendixchapter,%
  }{\GetAppendixChapterTitleNumberFormatString}%
} % End of \newcommand{}

% ---------------------------

\pgfkeys
{
  /AppendixSTitleNumberFormat/.is family, /AppendixSTitleNumberFormat,
  default/.style =
  {
    BeginText = \empty,
    EndText = \empty,
    TextAlign = {Left},
    CNumStyle = UpperAlph,
    SNumStyle = Arabic,
    SepAtIndex = {.}, % 目錄中章節號碼跟章節題目中的分隔符號
    SepBetweenCnS = {.}, % 章號碼跟節號碼中的分隔符號
  },
  BeginText/.estore in = \GetAppendixSTitleNumberFormatBeginText,
  EndText/.estore in = \GetAppendixSTitleNumberFormatEndText,
  TextAlign/.estore in = \GetAppendixSTitleNumberFormatTextAlign,
  CNumStyle/.estore in = \GetAppendixSTitleNumberFormatCNumStyle,
  SNumStyle/.estore in = \GetAppendixSTitleNumberFormatSNumStyle,
  SepAtIndex/.estore in = \GetAppendixSTitleNumberFormatSepAtIndex,
  SepBetweenCnS/.estore in = \GetAppendixSTitleNumberFormatSepBetweenCnS,
} % End of \pgfkeys{}

\newcommand\GetAppendixSectionTitleNumberFormatString{}
\newcommand\SetupAppendixSectionTitleNumberFormatString
{%
  \SetupTitleNumberFormatString{AppendixSection}%
  {%
    BeginText=\GetAppendixSTitleNumberFormatBeginText,%
    EndText=\GetAppendixSTitleNumberFormatEndText,%
    CNumStyle=\GetAppendixSTitleNumberFormatCNumStyle,%
    SNumStyle=\GetAppendixSTitleNumberFormatSNumStyle,%
    SepAtIndex=\GetAppendixSTitleNumberFormatSepAtIndex,%
    SepBetweenCnS=\GetAppendixSTitleNumberFormatSepBetweenCnS,%
    CCounterName=appendixchapter,%
    SCounterName=appendixsection,%
  }{\GetAppendixSectionTitleNumberFormatString}%
} % End of \newcommand{}

% ---------------------------

\pgfkeys
{
  /AppendixSSTitleNumberFormat/.is family, /AppendixSSTitleNumberFormat,
  default/.style =
  {
    BeginText = \empty,
    EndText = \empty,
    TextAlign = {Left},
    CNumStyle = UpperAlph,
    SNumStyle = Arabic,
    SSNumStyle = Arabic,
    SepAtIndex = {.}, % 目錄中章節號碼跟章節題目中的分隔符號
    SepBetweenCnS = {.}, % 章號碼跟節號碼中的分隔符號
    SepBetweenSnSS = {.}, % 節號碼跟小節號碼中的分隔符號
  },
  BeginText/.estore in = \GetAppendixSSTitleNumberFormatBeginText,
  EndText/.estore in = \GetAppendixSSTitleNumberFormatEndText,
  TextAlign/.estore in = \GetAppendixSSTitleNumberFormatTextAlign,
  CNumStyle/.estore in = \GetAppendixSSTitleNumberFormatCNumStyle,
  SNumStyle/.estore in = \GetAppendixSSTitleNumberFormatSNumStyle,
  SSNumStyle/.estore in = \GetAppendixSSTitleNumberFormatSSNumStyle,
  SepAtIndex/.estore in = \GetAppendixSSTitleNumberFormatSepAtIndex,
  SepBetweenCnS/.estore in = \GetAppendixSSTitleNumberFormatSepBetweenCnS,
  SepBetweenSnSS/.estore in = \GetAppendixSSTitleNumberFormatSepBetweenSnSS,
} % End of \pgfkeys{}

\newcommand\GetAppendixSubSectionTitleNumberFormatString{}
\newcommand\SetupAppendixSubSectionTitleNumberFormatString
{%
  \SetupTitleNumberFormatString{AppendixSubSection}%
  {%
    BeginText=\GetAppendixSSTitleNumberFormatBeginText,%
    EndText=\GetAppendixSSTitleNumberFormatEndText,%
    CNumStyle=\GetAppendixSSTitleNumberFormatCNumStyle,%
    SNumStyle=\GetAppendixSSTitleNumberFormatSNumStyle,%
    SSNumStyle=\GetAppendixSSTitleNumberFormatSSNumStyle,%
    SepAtIndex=\GetAppendixSSTitleNumberFormatSepAtIndex,%
    SepBetweenCnS=\GetAppendixSSTitleNumberFormatSepBetweenCnS,%
    SepBetweenSnSS=\GetAppendixSSTitleNumberFormatSepBetweenSnSS,%
    CCounterName=appendixchapter,%
    SCounterName=appendixsection,%
    SSCounterName=appendixsubsection,%
  }{\GetAppendixSubSectionTitleNumberFormatString}%
} % End of \newcommand{}

% ---------------------------

\pgfkeys
{
  /AppendixSSSTitleNumberFormat/.is family, /AppendixSSSTitleNumberFormat,
  default/.style =
  {
    BeginText = \empty,
    EndText = \empty,
    TextAlign = {Left},
    CNumStyle = UpperAlph,
    SNumStyle = Arabic,
    SSNumStyle = Arabic,
    SSSNumStyle = Arabic,
    SepAtIndex = {.}, % 目錄中章節號碼跟章節題目中的分隔符號
    SepBetweenCnS = {.}, % 章號碼跟節號碼中的分隔符號
    SepBetweenSnSS = {.}, % 節號碼跟小節號碼中的分隔符號
    SepBetweenSSCnSSS = {.}, % 小節號碼跟小小節號碼中的分隔符號
  },
  BeginText/.estore in = \GetAppendixSSSTitleNumberFormatBeginText,
  EndText/.estore in = \GetAppendixSSSTitleNumberFormatEndText,
  TextAlign/.estore in = \GetAppendixSSSTitleNumberFormatTextAlign,
  CNumStyle/.estore in = \GetAppendixSSSTitleNumberFormatCNumStyle,
  SNumStyle/.estore in = \GetAppendixSSSTitleNumberFormatSNumStyle,
  SSNumStyle/.estore in = \GetAppendixSSSTitleNumberFormatSSNumStyle,
  SSSNumStyle/.estore in = \GetAppendixSSSTitleNumberFormatSSSNumStyle,
  SepAtIndex/.estore in = \GetAppendixSSSTitleNumberFormatSepAtIndex,
  SepBetweenCnS/.estore in = \GetAppendixSSSTitleNumberFormatSepBetweenCnS,
  SepBetweenSnSS/.estore in = \GetAppendixSSSTitleNumberFormatSepBetweenSnSS,
  SepBetweenSSCnSSS/.estore in = \GetAppendixSSSTitleNumberFormatSepBetweenSSCnSSS,
} % End of \pgfkeys{}

\newcommand\GetAppendixSubSubSectionTitleNumberFormatString{}
\newcommand\SetupAppendixSubSubSectionTitleNumberFormatString
{%
  \SetupTitleNumberFormatString{AppendixSubSubSection}%
  {%
    BeginText=\GetAppendixSSSTitleNumberFormatBeginText,%
    EndText=\GetAppendixSSSTitleNumberFormatEndText,%
    CNumStyle=\GetAppendixSSSTitleNumberFormatCNumStyle,%
    SNumStyle=\GetAppendixSSSTitleNumberFormatSNumStyle,%
    SSNumStyle=\GetAppendixSSSTitleNumberFormatSSNumStyle,%
    SSSNumStyle=\GetAppendixSSSTitleNumberFormatSSSNumStyle,%
    SepAtIndex=\GetAppendixSSSTitleNumberFormatSepAtIndex,%
    SepBetweenCnS=\GetAppendixSSSTitleNumberFormatSepBetweenCnS,%
    SepBetweenSnSS=\GetAppendixSSSTitleNumberFormatSepBetweenSnSS,%
    SepBetweenSSCnSSS=\GetAppendixSSSTitleNumberFormatSepBetweenSSCnSSS,%
    CCounterName=appendixchapter,%
    SCounterName=appendixsection,%
    SSCounterName=appendixsubsection,%
    SSSCounterName=appendixsubsubsection,%
  }{\GetAppendixSubSubSectionTitleNumberFormatString}%
} % End of \newcommand{}

% ----------------------------------------------------------------------------

\newcommand\SetupGeneralFigureNumberFormatString
{%
  \renewcommand\GetGeneralFigureNumberFormatString{}%
  \AppendCounterStringToFormatString{%
    \GetGeneralFigureNumberFormatString}{%
    Arabic}{chapter}%
  \appto\GetGeneralFigureNumberFormatString{.}
  \AppendCounterStringToFormatString{%
    \GetGeneralFigureNumberFormatString}{Arabic}{figure}%
} % End of \newcommand{}

% ---------------------------

\newcommand\SetupGeneralTableNumberFormatString
{%
  \renewcommand\GetGeneralTableNumberFormatString{}%
  \AppendCounterStringToFormatString{%
    \GetGeneralTableNumberFormatString}{%
    Arabic}{chapter}%
  \appto\GetGeneralTableNumberFormatString{.}
  \AppendCounterStringToFormatString{%
    \GetGeneralTableNumberFormatString}{Arabic}{table}%
} % End of \newcommand{}

% ---------------------------

\newcommand\SetupGeneralEquationNumberFormatString
{%
  \renewcommand\GetGeneralEquationNumberFormatString{}%
  \AppendCounterStringToFormatString{%
    \GetGeneralEquationNumberFormatString}{%
    Arabic}{chapter}%
  \appto\GetGeneralEquationNumberFormatString{.}
  \AppendCounterStringToFormatString{%
    \GetGeneralEquationNumberFormatString}{Arabic}{equation}%
} % End of \newcommand{}

% ---------------------------

\newcommand\SetupAppendixFigureNumberFormatString
{%
  \renewcommand\GetAppendixFigureNumberFormatString{}%
  \AppendCounterStringToFormatString{%
    \GetAppendixFigureNumberFormatString}{%
    Arabic}{appendixchapter}%
  \appto\GetAppendixFigureNumberFormatString{.}
  \AppendCounterStringToFormatString{%
    \GetAppendixFigureNumberFormatString}{Arabic}{figure}%
} % End of \newcommand{}

% ---------------------------

\newcommand\SetupAppendixTableNumberFormatString
{%
  \renewcommand\GetAppendixTableNumberFormatString{}%
  \AppendCounterStringToFormatString{%
    \GetAppendixTableNumberFormatString}{%
    Arabic}{appendixchapter}%
  \appto\GetAppendixTableNumberFormatString{.}
  \AppendCounterStringToFormatString{%
    \GetAppendixTableNumberFormatString}{Arabic}{table}%
} % End of \newcommand{}

% ---------------------------

\newcommand\SetupAppendixEquationNumberFormatString
{%
  \appto\GetAppendixEquationNumberFormatString{}%
  \AppendCounterStringToFormatString{%
    \GetAppendixEquationNumberFormatString}{%
    Arabic}{appendixchapter}%
  \appto\GetAppendixEquationNumberFormatString{.}
  \AppendCounterStringToFormatString{%
    \GetAppendixEquationNumberFormatString}{Arabic}{equation}%
} % End of \newcommand{}

% ---------------------------

\pgfkeys
{
  /SetupGeneralAppendixNumberFormatString/.is family, /SetupGeneralAppendixNumberFormatString,
  default/.style =
  {
    CNumStyle = \empty,
    CCounterName = \empty,
    SNumStyle = \empty,
    SCounterName = \empty,
    SSNumStyle = \empty,
    SSCounterName = \empty,
    SSSNumStyle = \empty,
    SSSCounterName = \empty,
    SepBetweenCnS = \empty,
    SepBetweenSnSS = \empty,
    SepBetweenSSCnSSS = \empty,
    %
    FigureNumStyle = \empty,
    FigureCounterName = \empty,
    FigureSep = {.},
    %
    TableNumStyle = \empty,
    TableCounterName = \empty,
    TableSep = {.},
    %
    EquationNumStyle = \empty,
    EquationCounterName = \empty,
    EquationSep = {.},
  },
  CNumStyle/.estore in = \TmpValueCNumStyle,
  CCounterName/.estore in = \TmpValueCCounterName,
  SNumStyle/.estore in = \TmpValueSNumStyle,
  SCounterName/.estore in = \TmpValueSCounterName,
  SSNumStyle/.estore in = \TmpValueSSNumStyle,
  SSCounterName/.estore in = \TmpValueSSCounterName,
  SSSNumStyle/.estore in = \TmpValueSSSNumStyle,
  SSSCounterName/.estore in = \TmpValueSSSCounterName,
  SepBetweenCnS/.estore in = \TmpValueSepBetweenCnS,
  SepBetweenSnSS/.estore in = \TmpValueSepBetweenSnSS,
  SepBetweenSSCnSSS/.estore in = \TmpValueSepBetweenSSCnSSS,
  %
  FigureNumStyle/.estore in = \TmpValueFigureNumStyle,
  FigureCounterName/.estore in = \TmpValueFigureCounterName,
  FigureSep/.estore in = \TmpValueFigureSep,
  %
  TableNumStyle/.estore in = \TmpValueTableNumStyle,
  TableCounterName/.estore in = \TmpValueTableCounterName,
  TableSep/.estore in = \TmpValueTableSep,
  %
  EquationNumStyle/.estore in = \TmpValueEquationNumStyle,
  EquationCounterName/.estore in = \TmpValueEquationCounterName,
  EquationSep/.estore in = \TmpValueEquationSep,
} % End of \pgfkeys{}

\newcommand\SetupGeneralAppendixNumberFormatString[3]
{%
  \pgfkeys{/SetupGeneralAppendixNumberFormatString, default, #2}%
  \renewcommand#3{}%
  \ifboolexpr{%
    test {\ifstrequal{#1}{Chapter}} or %
    test {\ifstrequal{#1}{AppendixChapter}} }
      {%
        \AppendCounterStringToFormatString{%
          #3}{\TmpValueCNumStyle}{\TmpValueCCounterName}%
      }{}%
  %
  \ifboolexpr{%
    test {\ifstrequal{#1}{Section}} or %
    test {\ifstrequal{#1}{AppendixSection}} }
      {%
        \AppendCounterStringToFormatString{%
          #3}{\TmpValueCNumStyle}{\TmpValueCCounterName}%
        \appto#3{\TmpValueSepBetweenCnS}%
        \AppendCounterStringToFormatString{%
          #3}{\TmpValueSNumStyle}{\TmpValueSCounterName}%
      }{}%
  %
  \ifboolexpr{%
    test {\ifstrequal{#1}{SubSection}} or %
    test {\ifstrequal{#1}{AppendixSubSection}} }
      {%
        \AppendCounterStringToFormatString{%
          #3}{\TmpValueCNumStyle}{\TmpValueCCounterName}%
        \appto#3{\TmpValueSepBetweenCnS}%
        \AppendCounterStringToFormatString{%
          #3}{\TmpValueSNumStyle}{\TmpValueSCounterName}%
        \appto#3{\TmpValueSepBetweenSnSS}%
        \AppendCounterStringToFormatString{%
          #3}{\TmpValueSSNumStyle}{\TmpValueSSCounterName}%
      }{}%
  %
  \ifboolexpr{%
    test {\ifstrequal{#1}{SubSubSection}} or %
    test {\ifstrequal{#1}{AppendixSubSubSection}} }
      {%
        \AppendCounterStringToFormatString{%
          #3}{\TmpValueCNumStyle}{\TmpValueCCounterName}%
        \appto#3{\TmpValueSepBetweenCnS}%
        \AppendCounterStringToFormatString{%
          #3}{\TmpValueSNumStyle}{\TmpValueSCounterName}%
        \appto#3{\TmpValueSepBetweenSnSS}%
        \AppendCounterStringToFormatString{%
          #3}{\TmpValueSSNumStyle}{\TmpValueSSCounterName}%
        \appto#3{\TmpValueSepBetweenSSCnSSS}%
        \AppendCounterStringToFormatString{%
          #3}{\TmpValueSSSNumStyle}{\TmpValueSSSCounterName}%
      }{}%
  %
  \ifboolexpr{%
    test {\ifstrequal{#1}{Figure}} or %
    test {\ifstrequal{#1}{AppendixFigure}} }
      {%
        \AppendCounterStringToFormatString{%
          #3}{\TmpValueCNumStyle}{\TmpValueCCounterName}%
        \appto#3{\TmpValueFigureSep}%
        \AppendCounterStringToFormatString{%
          #3}{\TmpValueFigureNumStyle}{\TmpValueFigureCounterName}%
      }{}%
  %
  \ifboolexpr{%
    test {\ifstrequal{#1}{Table}} or %
    test {\ifstrequal{#1}{AppendixTable}} }
      {%
        \AppendCounterStringToFormatString{%
          #3}{\TmpValueCNumStyle}{\TmpValueCCounterName}%
        \appto#3{\TmpValueTableSep}%
        \AppendCounterStringToFormatString{%
          #3}{\TmpValueTableNumStyle}{\TmpValueTableCounterName}%
      }{}%
  %
  \ifboolexpr{%
    test {\ifstrequal{#1}{Equation}} or %
    test {\ifstrequal{#1}{AppendixEquation}} }
      {%
        \AppendCounterStringToFormatString{%
          #3}{\TmpValueCNumStyle}{\TmpValueCCounterName}%
        \appto#3{\TmpValueEquationSep}%
        \AppendCounterStringToFormatString{%
          #3}{\TmpValueEquationNumStyle}{\TmpValueEquationCounterName}%
      }{}%
  %
} % End of \newcommand{}

\newcommand\GetGeneralChapterNumberingFormatString{}
\newcommand\GetGeneralSectionNumberingFormatString{}
\newcommand\GetGeneralSubSectionNumberingFormatString{}
\newcommand\GetGeneralSubSubSectionNumberingFormatString{}
\newcommand\GetGeneralFigureNumberFormatString{}
\newcommand\GetGeneralTableNumberFormatString{}
\newcommand\GetGeneralEquationNumberFormatString{}

\newcommand\GetAppendixChapterNumberingFormatString{}
\newcommand\GetAppendixSectionNumberingFormatString{}
\newcommand\GetAppendixSubSectionNumberingFormatString{}
\newcommand\GetAppendixSubSubSectionNumberingFormatString{}
\newcommand\GetAppendixFigureNumberFormatString{}
\newcommand\GetAppendixTableNumberFormatString{}
\newcommand\GetAppendixEquationNumberFormatString{}

\newcommand\InitinalGeneralAppendixNumberingFormatString
{%
  \SetupGeneralAppendixNumberFormatString{Chapter}%
  {%
    CNumStyle=\GetCTitleNumberFormatCNumStyle,%
    CCounterName=chapter,%
  }{\GetGeneralChapterNumberingFormatString}%
  %
  \SetupGeneralAppendixNumberFormatString{Section}%
  {%
    CNumStyle=\GetSTitleNumberFormatCNumStyle,%
    SNumStyle=\GetSTitleNumberFormatSNumStyle,%
    SepBetweenCnS=\GetSTitleNumberFormatSepBetweenCnS,%
    CCounterName=chapter,%
    SCounterName=section,%
  }{\GetGeneralSectionNumberingFormatString}%
  %
  \SetupGeneralAppendixNumberFormatString{SubSection}%
  {%
    CNumStyle=\GetSSTitleNumberFormatCNumStyle,%
    SNumStyle=\GetSSTitleNumberFormatSNumStyle,%
    SSNumStyle=\GetSSTitleNumberFormatSSNumStyle,%
    SepBetweenCnS=\GetSSTitleNumberFormatSepBetweenCnS,%
    SepBetweenSnSS=\GetSSTitleNumberFormatSepBetweenSnSS,%
    CCounterName=chapter,%
    SCounterName=section,%
    SSCounterName=subsection,%
  }{\GetGeneralSubSectionNumberingFormatString}%
  %
  \SetupGeneralAppendixNumberFormatString{SubSubSection}%
  {%
    CNumStyle=\GetSSSTitleNumberFormatCNumStyle,%
    SNumStyle=\GetSSSTitleNumberFormatSNumStyle,%
    SSNumStyle=\GetSSSTitleNumberFormatSSNumStyle,%
    SSSNumStyle=\GetSSSTitleNumberFormatSSSNumStyle,%
    SepBetweenCnS=\GetSSSTitleNumberFormatSepBetweenCnS,%
    SepBetweenSnSS=\GetSSSTitleNumberFormatSepBetweenSnSS,%
    SepBetweenSSCnSSS=\GetSSSTitleNumberFormatSepBetweenSSCnSSS,%
    CCounterName=chapter,%
    SCounterName=section,%
    SSCounterName=subsection,%
    SSSCounterName=subsubsection,%
  }{\GetGeneralSubSubSectionNumberingFormatString}%
  %
  \SetupGeneralAppendixNumberFormatString{AppendixChapter}%
  {%
    CNumStyle=\GetAppendixCTitleNumberFormatCNumStyle,%
    CCounterName=appendixchapter,%
  }{\GetAppendixChapterNumberingFormatString}%
  %
  \SetupGeneralAppendixNumberFormatString{AppendixSection}%
  {%
    CNumStyle=\GetAppendixSTitleNumberFormatCNumStyle,%
    SNumStyle=\GetAppendixSTitleNumberFormatSNumStyle,%
    SepBetweenCnS=\GetAppendixSTitleNumberFormatSepBetweenCnS,%
    CCounterName=appendixchapter,%
    SCounterName=appendixsection,%
  }{\GetAppendixSectionNumberingFormatString}%
  %
  \SetupGeneralAppendixNumberFormatString{AppendixSubSection}%
  {%
    CNumStyle=\GetAppendixSSTitleNumberFormatCNumStyle,%
    SNumStyle=\GetAppendixSSTitleNumberFormatSNumStyle,%
    SSNumStyle=\GetAppendixSSTitleNumberFormatSSNumStyle,%
    SepBetweenCnS=\GetAppendixSSTitleNumberFormatSepBetweenCnS,%
    SepBetweenSnSS=\GetAppendixSSTitleNumberFormatSepBetweenSnSS,%
    CCounterName=appendixchapter,%
    SCounterName=appendixsection,%
    SSCounterName=appendixsubsection,%
  }{\GetAppendixSubSectionNumberingFormatString}%
  %
  \SetupGeneralAppendixNumberFormatString{AppendixSubSubSection}%
  {%
    CNumStyle=\GetAppendixSSSTitleNumberFormatCNumStyle,%
    SNumStyle=\GetAppendixSSSTitleNumberFormatSNumStyle,%
    SSNumStyle=\GetAppendixSSSTitleNumberFormatSSNumStyle,%
    SSSNumStyle=\GetAppendixSSSTitleNumberFormatSSSNumStyle,%
    SepBetweenCnS=\GetAppendixSSSTitleNumberFormatSepBetweenCnS,%
    SepBetweenSnSS=\GetAppendixSSSTitleNumberFormatSepBetweenSnSS,%
    SepBetweenSSCnSSS=\GetAppendixSSSTitleNumberFormatSepBetweenSSCnSSS,%
    CCounterName=appendixchapter,%
    SCounterName=appendixsection,%
    SSCounterName=appendixsubsection,%
    SSSCounterName=appendixsubsubsection,%
  }{\GetAppendixSubSubSectionNumberingFormatString}%
  %
} % End of \newcommand{}

  \begin{comment}
  \SetupGeneralAppendixNumberFormatString{Figure}%
  {%
    CNumStyle=\GetCTitleNumberFormatCNumStyle,%
    CCounterName=chapter,%
    FigureNumStyle=Arabic,%
    FigureCounterName=figure,%
    FigureSep=\GetSSTitleNumberFormatSepAtIndex,%
  }{\GetGeneralFigureNumberFormatString}%
  %
  \SetupGeneralAppendixNumberFormatString{AppendixFigure}%
  {%
    CNumStyle=\GetAppendixCTitleNumberFormatCNumStyle,%
    CCounterName=appendixchapter,%
    FigureNumStyle=Arabic,%
    FigureCounterName=figure,%
    FigureSep=\GetSSTitleNumberFormatSepAtIndex,%
  }{\GetAppendixFigureNumberFormatString}%
  %
  \SetupGeneralAppendixNumberFormatString{Table}%
  {%
    CNumStyle=\GetCTitleNumberFormatCNumStyle,%
    CCounterName=chapter,%
    TableNumStyle=Arabic,%
    TableCounterName=table,%
    TableSep=\GetSSTitleNumberFormatSepAtIndex,%
  }{\GetGeneralTableNumberFormatString}%
  %
  \SetupGeneralAppendixNumberFormatString{AppendixTable}%
  {%
    CNumStyle=\GetAppendixCTitleNumberFormatCNumStyle,%
    CCounterName=appendixchapter,%
    TableNumStyle=Arabic,%
    TableCounterName=table,%
    TableSep=\GetSSTitleNumberFormatSepAtIndex,%
  }{\GetAppendixTableNumberFormatString}%
  %
  \SetupGeneralAppendixNumberFormatString{Equation}%
  {%
    CNumStyle=\GetCTitleNumberFormatCNumStyle,%
    CCounterName=chapter,%
    EquationNumStyle=Arabic,%
    EquationCounterName=equation,%
    EquationSep=\GetSSTitleNumberFormatSepAtIndex,%
  }{\GetGeneralEquationNumberFormatString}%
  %
  \SetupGeneralAppendixNumberFormatString{AppendixEquation}%
  {%
    CNumStyle=\GetAppendixCTitleNumberFormatCNumStyle,%
    CCounterName=appendixchapter,%
    EquationNumStyle=Arabic,%
    EquationCounterName=equation,%
    EquationSep=\GetSSTitleNumberFormatSepAtIndex,%
  }{\GetAppendixEquationNumberFormatString}%
  %
  \end{comment}

% ---------------------------
\begin{comment}
% Set numbering format of general figure/table/equation
\newcommand\SetupFTENumberFormat
{%
  \SetupGeneralFigureNumberFormatString%
  \SetupGeneralTableNumberFormatString%
  \SetupGeneralEquationNumberFormatString%
  \renewcommand{\thefigure}{\GetGeneralFigureNumberFormatString}%
  \renewcommand{\thetable}{\GetGeneralTableNumberFormatString}%
  \renewcommand{\theequation}{\GetGeneralEquationNumberFormatString}%
} % End of \newcommand{}

\newcommand\SetupAppendixFTENumberFormat
{%
  \SetupAppendixFigureNumberFormatString%
  \SetupAppendixTableNumberFormatString%
  \SetupAppendixEquationNumberFormatString%
  \renewcommand{\thefigure}{\GetAppendixFigureNumberFormatString}
  \renewcommand{\thetable}{\GetAppendixTableNumberFormatString}
  \renewcommand{\theequation}{\GetAppendixEquationNumberFormatString}
} % End of \newcommand{}
\end{comment}
% ----------------------------------------------------------------------------

% Setup all custom numbering format
\newcommand\SetupNumberingFormat
{%
  \InitinalGeneralAppendixNumberingFormatString%
  %
%  \renewcommand{\thechapter}{\GetGeneralChapterNumberingFormatString}%
%  \renewcommand{\thesection}{\GetGeneralSectionNumberingFormatString}%
%  \renewcommand{\thesubsection}{\GetGeneralSubSectionNumberingFormatString}%
%  \renewcommand{\thesubsubsection}{\GetGeneralSubSubSectionNumberingFormatString}%
  %
  \SetupGeneralFigureNumberFormatString%
  \SetupGeneralTableNumberFormatString%
  \SetupGeneralEquationNumberFormatString%
  \renewcommand{\thefigure}{\GetGeneralFigureNumberFormatString}%
  \renewcommand{\thetable}{\GetGeneralTableNumberFormatString}%
  \renewcommand{\theequation}{\GetGeneralEquationNumberFormatString}%
} % End of \newcommand{}

% Setup all custom numbering format use in Appendix
\newcommand\SetupAppendixNumberingFormat
{%
  %
%  \renewcommand{\thechapter}{\GetAppendixChapterNumberingFormatString}%
%  \renewcommand{\thesection}{\GetAppendixSectionNumberingFormatString}%
%  \renewcommand{\thesubsection}{\GetAppendixSubSectionNumberingFormatString}%
%  \renewcommand{\thesubsubsection}{\GetAppendixSubSubSectionNumberingFormatString}%
  %
  \SetupAppendixFigureNumberFormatString%
  \SetupAppendixTableNumberFormatString%
  \SetupAppendixEquationNumberFormatString%
  \renewcommand{\thefigure}{\GetAppendixFigureNumberFormatString}
  \renewcommand{\thetable}{\GetAppendixTableNumberFormatString}
  \renewcommand{\theequation}{\GetAppendixEquationNumberFormatString}
} % End of \newcommand{}

% ----------------------------------------------------------------------------

\begin{comment}
\pgfkeys
{
  /SetNumberingFormat/.is family, /SetNumberingFormat,
  default/.style =
  {
    BeginText = \empty,
    EndText = \empty,
    CNumStyle = Arabic,
    SNumStyle = Arabic,
    SSNumStyle = Arabic,
    SSSNumStyle = Arabic,
    SepAtIndex = {.}, % 目錄中章節號碼跟章節題目中的分隔符號
    SepBetweenCnS = {.}, % 章號碼跟節號碼中的分隔符號
    SepBetweenSnSS = {.}, % 節號碼跟小節號碼中的分隔符號
    SepBetweenSSCnSSS = {.}, % 小節號碼跟小小節號碼中的分隔符號
  },
  BeginText/.estore in = \TmpValueBeginText,
  EndText/.estore in = \TmpValueEndText,
  CNumStyle/.estore in = \TmpValueCNumStyle,
  SNumStyle/.estore in = \TmpValueSNumStyle,
  SSNumStyle/.estore in = \TmpValueSSNumStyle,
  SSSNumStyle/.estore in = \TmpValueSSSNumStyle,
  SepAtIndex/.estore in = \TmpValueSepAtIndex,
  SepBetweenCnS/.estore in = \TmpValueSepBetweenCnS,
  SepBetweenSnSS/.estore in = \TmpValueSepBetweenSnSS,
  SepBetweenSSCnSSS/.estore in = \TmpValueSepBetweenSSCnSSS,
} % End of \pgfkeys{}
\end{comment}
% Parse the setting based on title type
% Chapter, Section, SubSection, SubSubSection
\newcommand{\SetNumberingFormat}[2][\empty]
{%
  \ifthenelse{\equal{#1}{Chapter}}
  {%
    \pgfkeys{/CTitleNumberFormat, default, #2}%
  }{}%
  %
  \ifthenelse{\equal{#1}{Section}}
  {%
    \pgfkeys{/STitleNumberFormat, default, #2}%
  }{}%
  %
  \ifthenelse{\equal{#1}{SubSection}}
  {%
    \pgfkeys{/SSTitleNumberFormat, default, #2}%
  }{}%
  %
  \ifthenelse{\equal{#1}{SubSubSection}}
  {%
    \pgfkeys{/SSSTitleNumberFormat, default, #2}%
  }{}%
  %
  \ifthenelse{\equal{#1}{AppendixChapter}}
  {%
    \pgfkeys{/AppendixCTitleNumberFormat, default, #2}%
  }{}%
  %
  \ifthenelse{\equal{#1}{AppendixSection}}
  {%
    \pgfkeys{/AppendixSTitleNumberFormat, default, #2}%
  }{}%
  %
  \ifthenelse{\equal{#1}{AppendixSubSection}}
  {%
    \pgfkeys{/AppendixSSTitleNumberFormat, default, #2}%
  }{}%
  %
  \ifthenelse{\equal{#1}{AppendixSubSubSection}}
  {%
    \pgfkeys{/AppendixSSSTitleNumberFormat, default, #2}%
  }{}%
} % End of \newcommand{}

% Default
% Chapter: Chapter 1
% Section: 1.1
% SubSection: 1.1.1
% SubSubSection: __EMPTY__
\SetNumberingFormat[Chapter]{%
  BeginText = {Chapter }, EndText = {},
  TextAlign = {Left},
  CNumStyle = {Arabic},
  SepAtIndex = {.},
} % End of \SetNumberingFormat{}

\SetNumberingFormat[Section]{%
  BeginText = {}, EndText = {},
  TextAlign = {Left},
  CNumStyle = {Arabic}, SNumStyle = {Arabic},
  SepAtIndex = {.}, SepBetweenCnS = {.},
} % End of \SetNumberingFormat{}

\SetNumberingFormat[SubSection]{%
  BeginText = {}, EndText = {},
  TextAlign = {Left},
  CNumStyle = {Arabic}, SNumStyle = {Arabic}, SSNumStyle = {Arabic},
  SepAtIndex = {.}, SepBetweenCnS = {.}, SepBetweenSnSS = {.},
} % End of \SetNumberingFormat{}

\SetNumberingFormat[SubSubSection]{%
  BeginText = {}, EndText = {},
  TextAlign = {Left},
  CNumStyle = {}, SNumStyle = {}, SSNumStyle = {}, SSSNumStyle = {},
  SepAtIndex = {}, SepBetweenCnS = {},
  SepBetweenSnSS = {}, SepBetweenSSCnSSS = {},
} % End of \SetNumberingFormat{}

% Default
% Chapter: Appendix A
% Section: A.1
% SubSection: A.1.1
% SubSubSection: __EMPTY__
\SetNumberingFormat[AppendixChapter]{%
  BeginText = {Appendix }, EndText = {},
  TextAlign = {Left},
  CNumStyle = {UpperAlph},
  SepAtIndex = {.},
} % End of \SetNumberingFormat{}

\SetNumberingFormat[AppendixSection]{%
  BeginText = {}, EndText = {},
  TextAlign = {Left},
  CNumStyle = {UpperAlph}, SNumStyle = {Arabic},
  SepAtIndex = {.}, SepBetweenCnS = {.},
} % End of \SetNumberingFormat{}

\SetNumberingFormat[AppendixSubSection]{%
  BeginText = {}, EndText = {},
  TextAlign = {Left},
  CNumStyle = {UpperAlph}, SNumStyle = {Arabic}, SSNumStyle = {Arabic},
  SepAtIndex = {.}, SepBetweenCnS = {.}, SepBetweenSnSS = {.},
} % End of \SetNumberingFormat{}

\SetNumberingFormat[AppendixSubSubSection]{%
  BeginText = {}, EndText = {},
  TextAlign = {Left},
  CNumStyle = {}, SNumStyle = {}, SSNumStyle = {}, SSSNumStyle = {},
  SepAtIndex = {}, SepBetweenCnS = {},
  SepBetweenSnSS = {}, SepBetweenSSCnSSS = {},
} % End of \SetNumberingFormat{}

% ----------------------------------------------------------------------------

% 過去的API, 以 Error提醒不能再使用

\newcommand{\ChapterTitleNumFormat}{\errmessage{模版: 由v1.4.5開始, ChapterTitleNumFormat已不能再使用. 請參考最新版的conf.tex使用方式.}\stop}

\newcommand{\SectionTitleNumFormat}{\errmessage{模版: 由v1.4.5開始, SectionTitleNumFormat已不能再使用. 請參考最新版的conf.tex使用方式.}\stop}

\newcommand{\SubSectionTitleNumFormat}{\errmessage{模版: 由v1.4.5開始, SubSectionTitleNumFormat已不能再使用. 請參考最新版的conf.tex使用方式.}\stop}

\newcommand{\SubSubSectionTitleNumFormat}{\errmessage{模版: 由v1.4.5開始, SubSubSectionTitleNumFormat已不能再使用. 請參考最新版的conf.tex使用方式.}\stop}

\newcommand{\AppendixChapterTitleNumFormat}{\errmessage{模版: 由v1.4.5開始, AppendixChapterTitleNumFormat已不能再使用. 請參考最新版的conf.tex使用方式.}\stop}

\newcommand{\AppendixSectionTitleNumFormat}{\errmessage{模版: 由v1.4.5開始, AppendixSectionTitleNumFormat已不能再使用. 請參考最新版的conf.tex使用方式.}\stop}

\newcommand{\AppendixSubSectionTitleNumFormat}{\errmessage{模版: 由v1.4.5開始, AppendixSubSectionTitleNumFormat已不能再使用. 請參考最新版的conf.tex使用方式.}\stop}

\newcommand{\AppendixSubSubSectionTitleNumFormat}{\errmessage{模版: 由v1.4.5開始, AppendixSubSubSectionTitleNumFormat已不能再使用. 請參考最新版的conf.tex使用方式.}\stop}

% ----------------------------------------------------------------------------

%
% This file is part of the project of
% National Cheng Kung University (NCKU) Thesis/Dissertation Template in LaTex.
% This project is hold at
%     <https://github.com/wengan-li/ncku-thesis-template-latex>
% by Wen-Gan Li.
%
% This project is distributed in the hope of usefuling to someone,
% you can redistribute it and/or modify it under the terms of the
% Attribution-NonCommercial-ShareAlike 4.0 International.
%
% You should have received a copy of the
% Attribution-NonCommercial-ShareAlike 4.0 International
% along with this project.
% If not, see <http://creativecommons.org/licenses/by-nc-sa/4.0/legalcode.txt>.
%
% Please feel free to fork it, modify it, and try it.
% Have fun !!!
%

% ----------------------------------------------------------------------------

\newcommand{\SetStyleItemizeItemLevelOne}[1]{%
  \renewcommand{\labelitemi}{$#1$}}
\newcommand{\SetStyleItemizeItemLevelTwo}[1]{%
  \renewcommand{\labelitemii}{$#1$}}
\newcommand{\SetStyleItemizeItemLevelThree}[1]{%
  \renewcommand{\labelitemiii}{$#1$}}
\newcommand{\SetStyleItemizeItemLevelFour}[1]{%
  \renewcommand{\labelitemiv}{$#1$}}

\newcommand{\SetStyleEnumItemLevelOne}[1]{%
  \renewcommand{\labelenumi}{$#1$}}
\newcommand{\SetStyleEnumItemLevelTwo}[1]{%
  \renewcommand{\labelenumii}{$#1$}}
\newcommand{\SetStyleEnumItemLevelThree}[1]{%
  \renewcommand{\labelenumiii}{$#1$}}
\newcommand{\SetStyleEnumItemLevelFour}[1]{%
  \renewcommand{\labelenumiv}{$#1$}}

% Default sytle of itemize items
\SetStyleItemizeItemLevelOne{\bullet}
\SetStyleItemizeItemLevelTwo{-}
\SetStyleItemizeItemLevelThree{\diamond}
\SetStyleItemizeItemLevelFour{\ast}

% ----------------------------------------------------------------------------

%
% This file is part of the project of
% National Cheng Kung University (NCKU) Thesis/Dissertation Template in LaTex.
% This project is hold at
%     <https://github.com/wengan-li/ncku-thesis-template-latex>
% by Wen-Gan Li.
%
% This project is distributed in the hope of usefuling to someone,
% you can redistribute it and/or modify it under the terms of the
% Attribution-NonCommercial-ShareAlike 4.0 International.
%
% You should have received a copy of the
% Attribution-NonCommercial-ShareAlike 4.0 International
% along with this project.
% If not, see <http://creativecommons.org/licenses/by-nc-sa/4.0/legalcode.txt>.
%
% Please feel free to fork it, modify it, and try it.
% Have fun !!!
%

% ----------------------------------------------------------------------------

%\def\singlespacing{%
%    \def\default@spacing{\baselineskip=15.5pt plus .5pt minus .2pt}}
\begin{comment}
\makeatletter

% Regular text spacing
\def\doublespacing{%
    \def\default@spacing{\baselineskip=20pt plus .5pt minus .2pt}}
\def\onehalfspacing{%
    \def\default@spacing{\baselineskip=20.5pt plus .5pt minus .2pt}}
\def\singlespacing{%
    \def\default@spacing{\baselineskip=15.5pt plus .5pt minus .2pt}}
\def\specialspacing{%
    \def\default@spacing{\baselineskip=21.5pt plus .5pt minus .2pt}}

\makeatother
\end{comment}
% ----------------------------------------------------------------------------

% 設定段落之間的距離
%\setlength{\parskip}{0.3cm}
  %
  % Reset page
%  \setlength{\parindent}{1.5em}
%  \baselineskip=26pt
  %
  % Set page
  %\baselineskip=15pt
%  \setlength{\parindent}{0.0pt}
  % 設定段落之間的距離
%  \setlength{\parskip}{0.5cm}
  %

% ----------------------------------------------------------------------------

% 	延伸baseline的高度
\newcommand{\ValueDefaultLineStretch}{1.2} % Default
\newcommand{\ValueCustomLineStretch}{1.2} % Default

\newcommand{\ValueDefaultLineStretchTypeDefault}{0}
\newcommand{\ValueDefaultLineStretchTypeCustom}{1}
\newcommand{\VarDefaultLineStretchType}{%
  \ValueDefaultLineStretchTypeDefault} % Default
\newcommand{\GetDefaultLineStretchType}{%
  \VarDefaultLineStretchType}
\newcommand{\SetDefaultLineStretchType}[1]{\renewcommand{\VarDefaultLineStretchType}{#1}}

% 公開的APIs
\newcommand{\SetLineStretch}[1]%
{%
  \renewcommand{\ValueCustomLineStretch}{#1}%
  \SetDefaultLineStretchType{\ValueDefaultLineStretchTypeCustom}%
} % End of \newcommand{}

\newcommand{\UseDefaultLineStretch}
{%
  \ifthenelse{\equal{\GetDefaultLineStretchType}{\ValueDefaultLineStretchTypeDefault}}%
  {\setstretch{\ValueDefaultLineStretch}}{}%
  %
  \ifthenelse{\equal{\GetDefaultLineStretchType}{\ValueDefaultLineStretchTypeCustom}}%
  {\setstretch{\ValueCustomLineStretch}}{}%
} % End of \newcommand{}

\UseDefaultLineStretch % Default

% ----------------------------------------------------------------------------

% 過去的API, 以 Error提醒不能再使用

\newcommand{\ThesisWroteInChi}{\errmessage{模版: 由v1.4.4開始, ThesisWroteInChi已不能再使用. 請參考最新版的conf.tex使用方式.}\stop}

% ----------------------------------------------------------------------------


% Helper function for different page or chapter
%
% This file is part of the project of
% National Cheng Kung University (NCKU) Thesis/Dissertation Template in LaTex.
% This project is hold at
%     <https://github.com/wengan-li/ncku-thesis-template-latex>
% by Wen-Gan Li.
%
% This project is distributed in the hope of usefuling to someone,
% you can redistribute it and/or modify it under the terms of the
% Attribution-NonCommercial-ShareAlike 4.0 International.
%
% You should have received a copy of the
% Attribution-NonCommercial-ShareAlike 4.0 International
% along with this project.
% If not, see <http://creativecommons.org/licenses/by-nc-sa/4.0/legalcode.txt>.
%
% Please feel free to fork it, modify it, and try it.
% Have fun !!!
%

% Some helper function about thesis

% ----------------------------------------------------------------------------

% Variable
\newcommand{\VarDemoModeOn}{1}
\newcommand{\VarDemoModeOff}{0}
\newcommand{\ValueDemoMode}{\VarDemoModeOff}
\newcommand{\GetDemoMode}{\ValueDemoMode}

% Mode
\newcommand{\DemoMode}{\renewcommand{\ValueDemoMode}{\VarDemoModeOn}}

% ----------------------------------------------------------------------------

\newcommand{\BeginThesis}
{
  % Start of paper
  \begin{document}
} % End of \newcommand{}

\newcommand{\EndThesis}
{
  % End of paper
  \end{document}
} % End of \newcommand{}

\newcommand{\CreateThesis}
{
  % Start of thesis
  \BeginThesis
  %
  % 內容 context
  \ifthenelse{\GetDemoMode = \VarDemoModeOn}%
  {% ------------------------------------------------
%
%論文內容次序:
% 1.考試合格證明
% 2.中英文摘要(論文以中文撰寫者須附英文延伸摘要)
% 3.誌謝
% 4.目錄
% 5.表目錄
% 6.圖目錄
% 7.符號
% 8.主文
% 9.參考文獻
% 10.附錄
%
% 註: 參考文獻書寫注意事項:
% (1).
%    文學院之中文文獻依分類及年代順序排列。
%    其他學院所之文獻依英文姓氏第一個字母
%    (或中文姓氏第一個字筆劃)及年代順序排列。
%
% (2).
%    期刊文獻之書寫依序為:
%        姓名、文章名稱、期刊名、卷別、期別、頁別、年代。
%
% (3).
%    書寫之文獻依序為:
%        姓名、書名、出版商名、出版地、頁別、年代。
%
% ------------------------------------------------

% 封面內頁 Inner Cover
%
% 封面: 顯示所有封面內容, 但沒有學校Logo)
%     主要用在印刷版, 如精裝版 或 平裝版
%     (使用cover.tex來產生)
%
% 內頁: 顯示所有封面內容, 但有學校Logo
%     主要用在電子版 + 印刷版
%
% 只要是印刷版, 不論是精裝版或平裝版, 都是 封面 (殼/皮) + 內頁.
% 只有在電子版時, 第一頁就是封面內頁.

\DisplayInnerCover

% ------------------------------------------------

% 學位考試論文證明書
\DisplayOral

% ------------------------------------------------

% 摘要 Abstract
% 除了外籍生, 本地生和僑生都是要編寫中文和英文摘要
% 論文以中文撰寫須以英文補寫 800 至 1200 字數的英文延伸摘要 (Extended Abstract)
% 詳細可看附件的學校要求或看example中的英文延伸摘要

%\input{./context/abstract/chi}             % 中文版
\input{./context/abstract/eng}             % 英文版
%\input{./context/abstract/extended}        % 英文延伸摘要

% ------------------------------------------------

% 誌謝 Acknowledgments
% 誌謝正常應該只要寫一種版本就可,
% 提供2種以自行選擇所顯示的語言.
% 2種同時編寫都是可以的.

%\input{./context/acknowledgments/chi}             % 中文版
\input{./context/acknowledgments/eng}             % 英文版

% ------------------------------------------------

% 目錄 (內容, 圖表和圖片) Index of contents, tables and figures.
% 內容會自動產生 The indices will generate in automate.
\DisplayIndex                 % 顯示索引
\DisplayTablesIndex   % 顯示表格索引
\DisplayFiguresIndex  % 顯示圖片索引

% ------------------------------------------------

% Nomenclature
\input{./example/nomenclature/nomenclature}

% ------------------------------------------------

% Introduction chapter
\input{./context/introduction/introduction}

% Objective chapter
%\input{./context/objective/objective}

% Related Work chapter
\input{./context/related-work/related-work}

% Algorithm chapter
%\input{./context/algorithm/algorithm}

% Performance chapter
%\input{./context/performance/performance}

% Conclusion chapter
\input{./context/conclusion/conclusion}

% Future work chapter
%\input{./context/future-work/future-work}

% ------------------------------------------------

% 參考文獻 References
\input{./context/references/references}

% ------------------------------------------------

% 附錄 Appendix
%\input{./context/appendix/appendix}

% ------------------------------------------------
}%
  {% ------------------------------------------------
%
%論文內容次序:
% 1.考試合格證明
% 2.中英文摘要(論文以中文撰寫者須附英文延伸摘要)
% 3.誌謝
% 4.目錄
% 5.表目錄
% 6.圖目錄
% 7.符號
% 8.主文
% 9.參考文獻
% 10.附錄
%
% 註: 參考文獻書寫注意事項:
% (1).
%    文學院之中文文獻依分類及年代順序排列。
%    其他學院所之文獻依英文姓氏第一個字母
%    (或中文姓氏第一個字筆劃)及年代順序排列。
%
% (2).
%    期刊文獻之書寫依序為:
%        姓名、文章名稱、期刊名、卷別、期別、頁別、年代。
%
% (3).
%    書寫之文獻依序為:
%        姓名、書名、出版商名、出版地、頁別、年代。
%
% ------------------------------------------------

% 封面內頁 Inner Cover
%
% 封面: 顯示所有封面內容, 但沒有學校Logo)
%     主要用在印刷版, 如精裝版 或 平裝版
%     (使用cover.tex來產生)
%
% 內頁: 顯示所有封面內容, 但有學校Logo
%     主要用在電子版 + 印刷版
%
% 只要是印刷版, 不論是精裝版或平裝版, 都是 封面 (殼/皮) + 內頁.
% 只有在電子版時, 第一頁就是封面內頁.

\DisplayInnerCover

% ------------------------------------------------

% 學位考試論文證明書
\DisplayOral

% ------------------------------------------------

% 摘要 Abstract
% 除了外籍生, 本地生和僑生都是要編寫中文和英文摘要
% 論文以中文撰寫須以英文補寫 800 至 1200 字數的英文延伸摘要 (Extended Abstract)
% 詳細可看附件的學校要求或看example中的英文延伸摘要

%\input{./context/abstract/chi}             % 中文版
\input{./context/abstract/eng}             % 英文版
%\input{./context/abstract/extended}        % 英文延伸摘要

% ------------------------------------------------

% 誌謝 Acknowledgments
% 誌謝正常應該只要寫一種版本就可,
% 提供2種以自行選擇所顯示的語言.
% 2種同時編寫都是可以的.

%\input{./context/acknowledgments/chi}             % 中文版
\input{./context/acknowledgments/eng}             % 英文版

% ------------------------------------------------

% 目錄 (內容, 圖表和圖片) Index of contents, tables and figures.
% 內容會自動產生 The indices will generate in automate.
\DisplayIndex                 % 顯示索引
\DisplayTablesIndex   % 顯示表格索引
\DisplayFiguresIndex  % 顯示圖片索引

% ------------------------------------------------

% Nomenclature
\input{./example/nomenclature/nomenclature}

% ------------------------------------------------

% Introduction chapter
\input{./context/introduction/introduction}

% Objective chapter
%\input{./context/objective/objective}

% Related Work chapter
\input{./context/related-work/related-work}

% Algorithm chapter
%\input{./context/algorithm/algorithm}

% Performance chapter
%\input{./context/performance/performance}

% Conclusion chapter
\input{./context/conclusion/conclusion}

% Future work chapter
%\input{./context/future-work/future-work}

% ------------------------------------------------

% 參考文獻 References
\input{./context/references/references}

% ------------------------------------------------

% 附錄 Appendix
%\input{./context/appendix/appendix}

% ------------------------------------------------
}
  %
  % End of thesis
  \EndThesis
} % End of \newcommand{}

% ----------------------------------------------------------------------------

%
% This file is part of the project of
% National Cheng Kung University (NCKU) Thesis/Dissertation Template in LaTex.
% This project is hold at
%     <https://github.com/wengan-li/ncku-thesis-template-latex>
% by Wen-Gan Li.
%
% This project is distributed in the hope of usefuling to someone,
% you can redistribute it and/or modify it under the terms of the
% Attribution-NonCommercial-ShareAlike 4.0 International.
%
% You should have received a copy of the
% Attribution-NonCommercial-ShareAlike 4.0 International
% along with this project.
% If not, see <http://creativecommons.org/licenses/by-nc-sa/4.0/legalcode.txt>.
%
% Please feel free to fork it, modify it, and try it.
% Have fun !!!
%

% Some helper function about page

% ----------------------------------------------------------------------------

\newcommand{\StartNewPage}
{
  % Page start
  \newpage
  \phantomsection
} % End of \newcommand{}

\newcommand{\EndOfPage}
{
  % End of page
  \clearpage
} % End of \newcommand{}

% ----------------------------------------------------------------------------

% 圖書館要求中/英文摘要是由羅馬數字頁碼'i'開始, 而非由封面算起.

\def\ValueEnablePageNumberCounting{1}
\def\ValueDisablePageNumberCounting{0}
\def\VarPageNumberCounting{\ValueEnablePageNumberCounting}
\def\GetPageNumberCounting{\VarPageNumberCounting}
\newcommand{\SetPageNumberCounting}[1]{\renewcommand{\VarPageNumberCounting}{#1}}

\def\ResetPageNumberCounting{%
  \ifthenelse{\equal{\GetPageNumberCounting}{\ValueEnablePageNumberCounting}}%
  {%
    \setcounter{page}{1}
    \SetPageNumberCounting{\ValueDisablePageNumberCounting}%
  }%
  {}%
} % End of \def{}

% ----------------------------------------------------------------------------

%
% This file is part of the project of
% National Cheng Kung University (NCKU) Thesis/Dissertation Template in LaTex.
% This project is hold at
%     <https://github.com/wengan-li/ncku-thesis-template-latex>
% by Wen-Gan Li.
%
% This project is distributed in the hope of usefuling to someone,
% you can redistribute it and/or modify it under the terms of the
% Attribution-NonCommercial-ShareAlike 4.0 International.
%
% You should have received a copy of the
% Attribution-NonCommercial-ShareAlike 4.0 International
% along with this project.
% If not, see <http://creativecommons.org/licenses/by-nc-sa/4.0/legalcode.txt>.
%
% Please feel free to fork it, modify it, and try it.
% Have fun !!!
%

% Some helper function use in cover

% ----------------------------------------------------------------------------
\newcommand{\StartCover}
{
  %
  \singlespacing%
  %
  \StartNewPage
  %
  % 設定使用 無頁碼
  \thispagestyle{empty}
  %
  \EnableCoverPageStyle
  %
  % Aligned to the center of the page
  \begin{center}
} % End of \newcommand{}

\newcommand{\EndCover}
{
  % End of alignment
  \end{center}
  \DisableCoverPageStyle
  \EndOfPage
  \UseDefaultLineStretch
} % End of \newcommand{}
% ----------------------------------------------------------------------------

% --- University name 學校名字 ---
\newcommand\VarUniversityChiName{國立成功大學}           % Default
\newcommand\VarUniversityEngName{National Cheng Kung University} % Default

\newcommand{\SetUniversityChiName}[1]{%
  \renewcommand{\VarUniversityChiName}{#1}%
} % End of \newcommand{}

\newcommand{\SetUniversityEngName}[1]{%
  \renewcommand{\VarUniversityEngName}{#1}%
} % End of \newcommand{}

\newcommand{\SetUniversityName}[2]
{%
  \SetUniversityChiName{#1}%
  \SetUniversityEngName{#2}%
} % End of \newcommand{}

\newcommand{\GetUniversityChiName}{\VarUniversityChiName}
\newcommand{\GetUniversityEngName}{\VarUniversityEngName}
% ----------------------------------------------------------------------------

% --- Chinese / English title 中英文論文題目 ---
\newcommand{\VarThesisChiName}{Chinese Title Here} % Default
\newcommand{\VarThesisEngName}{English Title Here} % Default
\newcommand{\SetChiTitle}[1]{\renewcommand{\VarThesisChiName}{#1}}
\newcommand{\SetEngTitle}[1]{\renewcommand{\VarThesisEngName}{#1}}
\newcommand{\SetTitle}[2]
{
  \SetChiTitle{#1}
  \SetEngTitle{#2}
} % End of \newcommand{}

\newcommand{\GetChiTitle}{\VarThesisChiName}
\newcommand{\GetEngTitle}{\VarThesisEngName}

% ----------------------------------------------------------------------------

% --- User's name 使用者名字 ---
\newcommand{\VarMyChiName}{你的名字}     % Default
\newcommand{\VarMyEngName}{Your name}   % Default
\newcommand{\SetMyChiName}[1]{\renewcommand{\VarMyChiName}{#1}}
\newcommand{\SetMyEngName}[1]{\renewcommand{\VarMyEngName}{#1}}
\newcommand{\SetMyName}[2]
{
  \SetMyChiName{#1}
  \SetMyEngName{#2}
} % End of \newcommand{}

\newcommand{\GetAuthorChiName}{\VarMyChiName}
\newcommand{\GetAuthorEngName}{\VarMyEngName}

% ----------------------------------------------------------------------------

% --- Degree name 學位 ---
% thesis 是指論文的通稱
% dissertation 指的是博士的論文

% 碩士論文  Master's thesis
% 博士論文  Doctoral dissertation

\newcommand{\ValueDegreeMaster}{0}
\newcommand{\ValueDegreePhd}{1}
\newcommand{\FlagDegreeType}{\ValueDegreePhd} % Default
\newcommand{\GetFlagDegreeType}{\FlagDegreeType}
\newcommand{\SetFlagDegreeType}[1]{\renewcommand{\FlagDegreeType}{#1}}

\newcommand{\VarDegreeChiName}{碩士/博士} % Default
\newcommand{\VarDegreeEngName}{Master / Doctor} % Default
\newcommand{\degreeThesisEname}{Master's Thesis / Doctoral Dissertation} % Default

\newcommand{\GetChiDegree}{\VarDegreeChiName}
\newcommand{\GetEngDegree}{\VarDegreeEngName}
\newcommand{\GetEngDegreeThesis}{\degreeThesisEname}
\newcommand{\SetChiDegree}[1]{\renewcommand{\VarDegreeChiName}{#1}}
\newcommand{\SetEngDegree}[1]{\renewcommand{\VarDegreeEngName}{#1}}
\newcommand{\SetEngDegreeThesis}[1]{\renewcommand{\degreeThesisEname}{#1}}

\newcommand{\PhdDegree}
{
  \SetFlagDegreeType{\ValueDegreePhd}
  \SetChiDegree{博士}
  \SetEngDegree{Doctor}
  \SetEngDegreeThesis{Doctoral Dissertation}
} % End of \newcommand{}

\newcommand{\MasterDegree}
{
  \SetFlagDegreeType{\ValueDegreeMaster}
  \SetChiDegree{碩士}
  \SetEngDegree{Master}
  \SetEngDegreeThesis{Master's Thesis}
} % End of \newcommand{}

% ----------------------------------------------------------------------------

% --- Date 日期 ---

% \CoverDateNumInChi: 日期使用中文數字,
% 而不是阿拉伯數字,
% 故使用'\CoverDateNumInChi'可以顯示
% '第一章' 而不是 '中華民國 103 年 12 月 31 日'.
% \CoverDateNumInChi必須配合\DisplayCoverInChi來使用, 否則會無效.
%\CoverDateNumInChi

% --- 論文的日期 ---
\newcommand{\ThesisYear}{2014}  % Default
\newcommand{\ThesisMonth}{1}    % Default

\newcommand{\SetThesisDate}[2]{\SetThesisDate{#1}{#2}} % For backporting
\newcommand{\SetCoverDate}[2]
{
  \SetThesisTaiwanYear{#1}
  \renewcommand{\ThesisYear}{#1}
  \renewcommand{\ThesisMonth}{#2}
} % End of \newcommand{}

\newcommand{\GetThesisYear}{\ThesisYear}
\newcommand{\GetThesisYearInTaiwanYear}{\ThesisTaiwanYearResult}
\newcommand{\GetThesisMonth}{\ThesisMonth}
\newcommand{\GetThesisMonthNumInChi}{\zhnumber{\ThesisMonth}}
\newcommand{\GetThesisMonthInEng}{\GetMonthInEng{\ThesisMonth}}

% ---  口試的日期 ---
\newcommand{\OralChiYear}{101}      % Default
\newcommand{\OralChiMonth}{1}       % Default
\newcommand{\OralChiDay}{1}         % Default
\newcommand{\OralEngYear}{2014}     % Default
\newcommand{\OralEngMonth}{January} % Default
\newcommand{\OralEngDay}{1}         % Default

\newcommand{\GetOralChiYear}{\OralChiYear}
\newcommand{\GetOralYearInTaiwanYear}
{\SetThesisTaiwanYear{\OralEngYear}\ThesisTaiwanYearResult}
\newcommand{\GetOralYearInTaiwanYearNumInChi}
{\SetThesisTaiwanYear{\OralEngYear}\zhdigits{\ThesisTaiwanYearResult}}
\newcommand{\GetOralChiMonth}{\OralChiMonth}
\newcommand{\GetOralChiDay}{\OralChiDay}
\newcommand{\GetOralEngYear}{\OralEngYear}
\newcommand{\GetOralEngMonth}{\OralEngMonth}
\newcommand{\GetOralEngDay}{\OralEngDay}
\newcommand{\GetOralEngDayNumInChi}{\zhnumber{\OralEngDay}}

\newcommand{\SetOralChiDate}[3]
{
  \SetOralTaiwanYear{#1}
  \renewcommand{\OralChiYear}{\OralTaiwanYearResult}
  \renewcommand{\OralChiMonth}{#2}
  \renewcommand{\OralChiDay}{#3}
} % End of \newcommand{}

\newcommand{\SetOralEngDate}[3]
{
  \renewcommand{\OralEngYear}{#1}
  \renewcommand{\OralEngMonth}{\GetMonthInEng{#2}}
  \renewcommand{\OralEngDay}{#3}
} % End of \newcommand{}

\newcommand{\SetOralDate}[3]
{
  \SetOralChiDate{#1}{#2}{#3}
  \SetOralEngDate{#1}{#2}{#3}
} % End of \newcommand{}

% ----------------------------------------------------------------------------

% --- 學院 College, 系所 Department and Institute ---

% --------------------------- College ---------------------------
\newcommand{\VarCollegeChiName}{學院 C}
\newcommand{\VarCollegeEngName}{College of C}
\newcommand{\SetCollChiName}[1]{\renewcommand{\VarCollegeChiName}{#1}}
\newcommand{\SetCollEngName}[1]{\renewcommand{\VarCollegeEngName}{#1}}
\newcommand{\SetCollName}[2]
{
  \SetCollChiName{#1}
  \SetCollEngName{#2}
} % End of \newcommand{}

\newcommand{\GetCollChiName}{\VarCollegeChiName}
\newcommand{\GetCollEngName}{\VarCollegeEngName}

% --------------------------- Department ---------------------------
\newcommand{\VarDepartmentChiName}{A 系 / 所}
%\newcommand{\VarDepartmentEngName}{DeptA} % Short form of department
\newcommand{\VarDepartmentEngFullName}{Department / Insitute A} % Full name of department
\newcommand{\SetDeptChiName}[1]{\renewcommand{\VarDepartmentChiName}{#1}}
%\newcommand{\SetDeptEngShortName}[1]{\renewcommand{\VarDepartmentEngName}{#1}}
\newcommand{\SetDeptEngFullName}[1]{\renewcommand{\VarDepartmentEngFullName}{#1}}
\newcommand{\SetDeptName}[3]
{
  \SetDeptChiName{#1}
%  \SetDeptEngShortName{#2}
  \SetDeptEngFullName{#3}
} % End of \newcommand{}

\newcommand{\GetDeptChiName}{\VarDepartmentChiName}
\newcommand{\GetDeptEngName}{\VarDepartmentEngFullName}

% ----------------------------------------------------------------------------

% --- 指導老師 Advisor(s) ---
% 在封面上預算了最多3位的空間
% 中文名字固定以 博士 結尾
% 英文名字固定以 Dr. 開頭

\newcommand{\VarAdvisorChiNameA}{X}
\newcommand{\VarAdvisorEngNameA}{X}
\newcommand{\VarAdvisorChiNameB}{}
\newcommand{\VarAdvisorEngNameB}{}
\newcommand{\VarAdvisorChiNameC}{}
\newcommand{\VarAdvisorEngNameC}{}

\newcommand{\GetAdvisorChiNameA}{\VarAdvisorChiNameA}
\newcommand{\GetAdvisorEngNameA}{\VarAdvisorEngNameA}
\newcommand{\GetAdvisorChiNameB}{\VarAdvisorChiNameB}
\newcommand{\GetAdvisorEngNameB}{\VarAdvisorEngNameB}
\newcommand{\GetAdvisorChiNameC}{\VarAdvisorChiNameC}
\newcommand{\GetAdvisorEngNameC}{\VarAdvisorEngNameC}

\newcommand{\SetAdvisorChiNameA}[1]{\renewcommand{\VarAdvisorChiNameA}{#1}}
\newcommand{\SetAdvisorEngNameA}[1]{\renewcommand{\VarAdvisorEngNameA}{#1}}
\newcommand{\SetAdvisorChiNameB}[1]{\renewcommand{\VarAdvisorChiNameB}{#1}}
\newcommand{\SetAdvisorEngNameB}[1]{\renewcommand{\VarAdvisorEngNameB}{#1}}
\newcommand{\SetAdvisorChiNameC}[1]{\renewcommand{\VarAdvisorChiNameC}{#1}}
\newcommand{\SetAdvisorEngNameC}[1]{\renewcommand{\VarAdvisorEngNameC}{#1}}

\newcommand{\SetAdvisorNameA}[2]
{
  \SetAdvisorChiNameA{#1}
  \SetAdvisorEngNameA{#2}
} % End of \newcommand{}

\newcommand{\SetAdvisorNameB}[2]
{
  \SetAdvisorChiNameB{#1}
  \SetAdvisorEngNameB{#2}
} % End of \newcommand{}

\newcommand{\SetAdvisorNameC}[2]
{
  \SetAdvisorChiNameC{#1}
  \SetAdvisorEngNameC{#2}
} % End of \newcommand{}

% ----------------------------------------------------------------------------

% Use to create cover
\newcommand{\CreateCover}%
{
  \begin{document}
  \input{./template/cover/cover}
  \end{document}
} % End of \newcommand{}

% Use to include and display inner cover
\newcommand{\DisplayInnerCover}{%
% This file is part of the project of
% National Cheng Kung University (NCKU) Thesis/Dissertation Template in LaTex.
% This project is hold at
%     <https://github.com/wengan-li/ncku-thesis-template-latex>
% by Wen-Gan Li.
%
% This project is distributed in the hope of usefuling to someone,
% you can redistribute it and/or modify it under the terms of the
% Attribution-NonCommercial-ShareAlike 4.0 International.
%
% You should have received a copy of the
% Attribution-NonCommercial-ShareAlike 4.0 International
% along with this project.
% If not, see <http://creativecommons.org/licenses/by-nc-sa/4.0/legalcode.txt>.
%
% Please feel free to fork it, modify it, and try it.
% Have fun !!!
% ------------------------------------------------
% 封面: 顯示所有封面內容, 但沒有學校Logo)
%     主要用在印刷版, 如精裝版 或 平裝版
%     (使用cover.tex來產生)
%
% 內頁: 顯示所有封面內容, 但有學校Logo
%     主要用在電子版 + 印刷版
%
% 只要是印刷版, 不論是精裝版或平裝版, 都是 封面 (殼/皮) + 內頁.
% 只有在電子版時, 第一頁就是封面內頁.

% ------------------------------------------------
%                Chinese cover
%                   中文封面
% ------------------------------------------------

\singlespacing 
\newpage
\phantomsection
\thispagestyle{empty}
\newgeometry{top=2.3cm,bottom=3cm,left=2cm,right=2cm,nohead,nofoot}
  
% Aligned to the center of the page
\begin{center}
% ------------------------------------------------

% 顯示 校名, 系所名, 論文種類
\begin{minipage}[c][5cm][t]{\textwidth}
  \begin{center}
    \makebox[10cm][s]{\Huge \GetUniversityChiName}

    \vspace{1cm}

    \makebox[8cm][s]{\Huge \GetDeptChiName} \\

    \vspace{1cm}

    \makebox[5cm][s]{\Huge \GetChiDegree 論文}%

    \ifthenelse{\equal{\GetFlagDisplayDraft}{1}}%
      {\vspace{1cm}\makebox[5cm][c]{\Huge \GetTextDraftChi}}{}
  \end{center}
\end{minipage}
% ------------------------------------------------

\vspace{5.5cm}

% ------------------------------------------------

% Chinese and English title 中英文題目
\begin{minipage}[c][5cm][t]{\textwidth}
  \begin{center}
    \makebox[\textwidth][c]{\parbox{\paperwidth}{\center \Large \GetChiTitle}}

    \vspace{0.5cm}
%    \parbox{\textwidth}{\center \Large \GetEngTitle}
    \makebox[\textwidth][c]{\parbox{\paperwidth}{\center \Large \GetEngTitle}}
  \end{center}
\end{minipage}

% ------------------------------------------------

\vspace{1.0cm}

% ------------------------------------------------

% 顯示學生和老師的名字

\begin{minipage}[c][4.5cm][t]{\textwidth}
  \begin{center}
    \ifthenelse{\equal{\GetCDBothName}{1}}%
      {%
        % ----- 中英文同時顯示 -----
        % --------------------------
        % 顯示 學生 的名字
        \hspace{2.0em}
        \makebox[4.0em][r]{\Large 學生:}
        \makebox[6.0em][l]{\Large \GetAuthorChiName}
        \makebox[8.0em][c]{}
        \makebox[4.0em][r]{\Large Student:}
        \makebox[8.0em][l]{\Large \GetAuthorEngName}\\
        % --------------------------
        \vspace{0.5cm}
        % --------------------------
        % 顯示 指導老師 A 的名字
        \hspace{2.0em}
        \makebox[4.0em][r]{\Large 指導老師:}
        \makebox[6.0em][l]{\Large \GetAdvisorChiNameA \thinspace 博士}
        \makebox[8.0em][c]{}
        \makebox[4.0em][r]{\Large Advisor:}
        \makebox[8.0em][l]{\Large Dr. \thinspace \GetAdvisorEngNameA}\\
        % --------------------------
        \vspace{0.1cm}
        % --------------------------
        % 顯示 指導老師 B 的名字
        \hspace{2.0em}
        \makebox[4.0em][r]
        {%
          \ifthenelse{\equal{\GetAdvisorChiNameB}{\empty}}%
            {}%
            {\Large 共同指導:}%
        }
        \makebox[6.0em][l]%
        {%
          \ifthenelse{\equal{\GetAdvisorChiNameB}{\empty}}%
            {}%
            {\Large \GetAdvisorChiNameB \thinspace 博士}%
        }
        \makebox[8.0em][c]{}
        \makebox[4.0em][r]
        {%
          \ifthenelse{\equal{\GetAdvisorEngNameB}{\empty}}%
            {}%
            {\Large Co-Advisor:}
        }
        \makebox[8.0em][l]%
        {%
          \ifthenelse{\equal{\GetAdvisorEngNameB}{\empty}}%
            {}%
            {\Large Dr. \thinspace \GetAdvisorEngNameB}%
        } \\
        % --------------------------
        \vspace{0.1cm}
        % --------------------------
        % 顯示 指導老師 C 的名字
        \hspace{2.0em}
        \makebox[4.0em][r]{}
        \makebox[6.0em][l]
        {%
          \ifthenelse{\equal{\GetAdvisorChiNameC}{\empty}}%
            {}%
            {\Large \GetAdvisorChiNameC \thinspace 博士}%
        }
        \makebox[8.0em][c]{}
        \makebox[4.0em][r]{}
        \makebox[8.0em][l]
        {%
          \ifthenelse{\equal{\GetAdvisorEngNameC}{\empty}}%
            {}%
            {\Large Dr. \thinspace \GetAdvisorEngNameC}%
        } \\
        % --------------------------
      }%
      {%
        % ----- 只顯示中文 -----
        % --------------------------
        % 顯示 學生 的名字
        \makebox[7em][r]{\Large 研究生:}
        \makebox[10.0em][l]{\Large \GetAuthorChiName}\\
        % --------------------------
        \vspace{0.5cm}
        % --------------------------
        % 顯示 指導老師 A 的名字
        \makebox[7em][r]{\Large 指導老師:}
        \makebox[10.0em][l]{\Large \GetAdvisorChiNameA \thinspace 博士} \\
        % --------------------------
        \vspace{0.1cm}
        % --------------------------
        % 顯示 指導老師 B 的名字
        \makebox[7em][r]
        {%
          \ifthenelse{\equal{\GetAdvisorChiNameB}{\empty}}%
            {}%
            {\Large 共同指導:}%
        }
        \makebox[10.0em][l]
        {%
          \ifthenelse{\equal{\GetAdvisorChiNameB}{\empty}}%
            {}%
            {\Large \GetAdvisorChiNameB \thinspace 博士}%
        } \\
        % --------------------------
        \vspace{0.1cm}
        % --------------------------
        % 顯示 指導老師 C 的名字
        \makebox[7em][r]{}
        \makebox[10.0em][l]
        {%
          \ifthenelse{\equal{\GetAdvisorChiNameC}{\empty}}%
            {}%
            {\Large \GetAdvisorChiNameC \thinspace 博士}%
        } \\
        % --------------------------
      }%
  \end{center}
\end{minipage}
% ------------------------------------------------

% Date 日期
\vspace{0.5cm}
\ifthenelse{\equal{\GetFlagDegreeType}{\ValueDegreeMaster}}%
{%
  \ifthenelse{\equal{\VarDisplayCoverDateNum}{\ValueDisplayCoverDateNumInChi}}%
  {\makebox[8cm][s]{\Large 中華民國\zhnumber{\GetThesisYearInTaiwanYear}年\zhnumber{\GetThesisMonth}月}}%
  {\makebox[8cm][s]{\Large 中華民國 \GetThesisYearInTaiwanYear 年 \GetThesisMonth 月}}%
}%
{
  \ifthenelse{\equal{\VarDisplayCoverDateNum}{\ValueDisplayCoverDateNumInChi}}%
  {%
    \makebox[8cm][s]{\Large 中華民國 \GetOralYearInTaiwanYearNumInChi 年 \GetThesisMonthNumInChi 月}%
    %\makebox[8cm][s]{\Large 中華民國 \GetOralYearInTaiwanYearNumInChi 年 \GetThesisMonthNumInChi 月 \GetOralEngDayNumInChi 日}%
  }%
  {%
    \makebox[8cm][s]{\Large 中華民國 \GetOralYearInTaiwanYear 年 \GetThesisMonth 月}%
    %\makebox[8cm][s]{\Large 中華民國 \GetOralYearInTaiwanYear 年 \GetThesisMonth 月 \GetOralEngDay 日}%
  }%
}

% ------------------------------------------------
\EndCover
% ------------------------------------------------

}

% ----------------------------------------------------------------------------

\newcommand{\ValueDisplayCoverLangEng}{0}
\newcommand{\ValueDisplayCoverLangChi}{1}
\newcommand{\VarDisplayCoverLang}{\ValueDisplayCoverLangEng}
\newcommand{\GetDisplayCoverLang}{\VarDisplayCoverLang}
\newcommand{\DisplayCoverInChi}{\renewcommand{\VarDisplayCoverLang}{\ValueDisplayCoverLangChi}}
\newcommand{\DisplayCoverInEng}{\renewcommand{\VarDisplayCoverLang}{\ValueDisplayCoverLangEng}}

% 日期顯示中文數字
\newcommand{\ValueDisplayCoverDateNumInNum}{0}
\newcommand{\ValueDisplayCoverDateNumInChi}{1}
\newcommand{\VarDisplayCoverDateNum}{\ValueDisplayCoverDateNumInNum}
\newcommand{\GetDisplayCoverDateNum}{\VarDisplayCoverDateNum}
\newcommand{\CoverDateNumInChi}{\renewcommand{\VarDisplayCoverDateNum}{\ValueDisplayCoverDateNumInChi}}

% ----------------------------------------------------------------------------

% Display Chinese and English name in english cover
\newcommand{\CoverDisplayNameChiEng}{0} % Default not display both
\newcommand{\SetCDBothName}{\renewcommand{\CoverDisplayNameChiEng}{1}}
\newcommand{\GetCDBothName}{\CoverDisplayNameChiEng}
\newcommand{\DisplayCoverPeoplesBothNames}{\SetCDBothName}

% A wrapper to handle \CDBothName{}
\newcommand{\CDBothName}{\DisplayCoverPeoplesBothNames}

% ----------------------------------------------------------------------------

% 顯示 '(初稿)' (中文版) 和 '(Draft)' (英文版) 在封面
\newcommand{\GetTextDraftChi}{(初稿)}
\newcommand{\GetTextDraftEng}{(Draft)}
\newcommand{\VarCoverDisplayDraft}{0} % Don't display in default
\newcommand{\EnableFlagDisplayDraft}{\renewcommand{\VarCoverDisplayDraft}{1}}
\newcommand{\DisplayDraft}{\EnableFlagDisplayDraft}
\newcommand{\GetFlagDisplayDraft}{\VarCoverDisplayDraft}

% ----------------------------------------------------------------------------

%
% This file is part of the project of
% National Cheng Kung University (NCKU) Thesis/Dissertation Template in LaTex.
% This project is hold at
%     <https://github.com/wengan-li/ncku-thesis-template-latex>
% by Wen-Gan Li.
%
% This project is distributed in the hope of usefuling to someone,
% you can redistribute it and/or modify it under the terms of the
% Attribution-NonCommercial-ShareAlike 4.0 International.
%
% You should have received a copy of the
% Attribution-NonCommercial-ShareAlike 4.0 International
% along with this project.
% If not, see <http://creativecommons.org/licenses/by-nc-sa/4.0/legalcode.txt>.
%
% Please feel free to fork it, modify it, and try it.
% Have fun !!!
%

% Some helper function about chapter and section

% ----------------------------------------------------------------------------

\newcommand{\DisplayChapterHeader}[2]%
{%
 \vspace*{0pt}%
  {%
    \parindent 0pt \centering%
    \Large\bf #1 \par%
    \vskip 15pt%
    \Large \bf #2 \par%
    \nobreak%
%    \vskip 35pt%
    \vskip 25pt%
  }%
  \addcontentsline{%
    toc}{chapter}{#1\GetCTitleNumberFormatSepAtIndex{\ \ \ }#2}%
  \addtocontents{lof}{\protect\addvspace{10 pt}}
  \addtocontents{lot}{\protect\addvspace{10 pt}}
} % End of \newcommand{}

\newcommand{\DisplayChapterHeaderStar}[1]%
{%
  \vspace*{0pt}%
  {%
    \parindent 0pt \centering \Large \bf #1\par%
    \nobreak%
%    \vskip 30pt%
    \vskip 20pt%
  }%
  \addcontentsline{toc}{chapter}{#1}%
} % End of \newcommand{}

% ----------------------------------------------------------------------------

% 使用 \frontmatter, \mainmatter 其實可簡化事情,
% 但這增加不需要的內容在context.tex, 使用bypass的方法已減少同學的煩惱.
\newcommand{\VarPageInMainmatter}{1}
\newcommand{\VarPageNotInMainmatter}{0}
\newcommand{\ValuePageInMainmatter}{\VarPageNotInMainmatter} % Default
\newcommand{\SetValuePageInMainmatter}{%
  \renewcommand{\ValuePageInMainmatter}{\VarPageInMainmatter}}
\newcommand{\GetValuePageInMainmatter}{\ValuePageInMainmatter}

% 這邊是overload latex原生的\chapter{}和\chapter*{}
\RenewDocumentCommand{\chapter}{s m}
{%
  %
  % 檢查是 \chapter*{} 或是 \chapter{}
  \IfBooleanTF{#1}%
  {%
    % \chapter*{}
    %
    % 用\chapter*{}都是前幾頁的內容, 用羅馬數字
    %
    %    Starred
    \DisplayChapterHeaderStar{#2}
  }%
  {%
    % \chapter{}
    %
    % 用\chapter{}都是主要內容, 故用數字
    % 如果第一次使用\chapter{}, 則把頁碼改使用成數字
    \ifthenelse{\equal{\GetValuePageInMainmatter}{\VarPageNotInMainmatter}}%
    {\pagenumbering{arabic}\SetValuePageInMainmatter}{}
    %
    %    Non-Starred
    %
    \ifthenelse{\equal{\GetStartAppendixChapter}{\ValueEnableAppendixChapter}}%
    {%
      % Appendix Chapter
      \refstepcounter{appendixchapter}
      %
      %
%      \SetAppendixTitleFormatFinalString
%      \DisplayChapterHeader{\GetAppendixTitleFormatFinalString}{#2}
      \SetupAppendixChapterTitleNumberFormatString%
      \DisplayChapterHeader{\GetAppendixChapterTitleNumberFormatString}{#2}
    }%
    {%
      % 一般Chapter
      \refstepcounter{chapter}
      %
      %
      \SetupChapterTitleNumberFormatString%
      \DisplayChapterHeader{\GetChapterTitleNumberFormatString}{#2}
%      \ifthenelse{\equal{\GetChapterTitleLang}{\VarChapterTitleLangEng}}%
%      {%
        % 題目為英文
%        \DisplayChapterHeader{第\thechapter 章}{#2}
%        \DisplayChapterHeader{\ValueChapterTitleNumberingFormat}{#2}
%      }%
%      {%
        % 題目為中文
%        \DisplayChapterHeader{第\thechapter 章}{#2}
%      }%
    }%
  }%
  %
  % Reset counters
  \setcounter{section}{0}%
  \setcounter{appendixsection}{0}%
  \setcounter{figure}{0}%
  \setcounter{table}{0}%
  \setcounter{equation}{0}%
} % End of \RenewDocumentCommand{}

% ----------------------------------------------------------------------------

\newcommand{\DisplaySectionHeader}[3]%
{%
  \par%
  \vspace{0.5cm}%
  %
  \ifthenelse{\equal{#3}{Left}}%
    {\begin{flushleft}\par}{}%
  \ifthenelse{\equal{#3}{Center}}%
    {\begin{center}\par}{}%
  \ifthenelse{\equal{#3}{Right}}%
    {\begin{flushright}\par}{}%
  %
  \ifthenelse{\equal{#1}{\empty}}%
  {%
    \parindent 0pt \textbf{#2} \par%
  }%
  {%
    \textbf{#1\hspace{0.5cm}#2} \par%
  }%
%  \vspace{0.1cm}%
  %
  \ifthenelse{\equal{#3}{Left}}%
    {\end{flushleft}\par}{}%
  \ifthenelse{\equal{#3}{Center}}%
    {\end{center}\par}{}%
  \ifthenelse{\equal{#3}{Right}}%
    {\end{flushright}\par}{}%
  %
  \addcontentsline{%
    toc}{section}{#1\GetSTitleNumberFormatSepAtIndex{\ \ \ }#2}%
} % End of \newcommand{}

\RenewDocumentCommand{\section}{s m}
{%
  %
  % 檢查是 \section*{} 或是 \section{}
  \IfBooleanTF{#1}%
  {%
    % \section*{}
%        Starred
  }%
  {%
    % \section{}
%        Non-Starred
    %
    \ifthenelse{\equal{\GetStartAppendixChapter}{\ValueEnableAppendixChapter}}%
    {%
      \refstepcounter{appendixsection}%
      \SetupAppendixSectionTitleNumberFormatString%
      \DisplaySectionHeader{\GetAppendixSectionTitleNumberFormatString}{#2}%
        {\GetAppendixSTitleNumberFormatTextAlign}%
    }%
    {%
      \refstepcounter{section}%
      \SetupSectionTitleNumberFormatString%
      \DisplaySectionHeader{\GetSectionTitleNumberFormatString}{#2}%
        {\GetSTitleNumberFormatTextAlign}%
    }%
  }%
  % Reset counters
  \setcounter{subsection}{0}%
  \setcounter{appendixsubsection}{0}%
} % End of \RenewDocumentCommand{}

% ----------------------------------------------------------------------------

\newcommand{\DisplaySubSectionHeader}[3]%
{%
  \par%
  \vspace{0.5cm}%
  %
  \ifthenelse{\equal{#3}{Left}}%
    {\begin{flushleft}\par}{}%
  \ifthenelse{\equal{#3}{Center}}%
    {\begin{center}\par}{}%
  \ifthenelse{\equal{#3}{Right}}%
    {\begin{flushright}\par}{}%
  %
  \ifthenelse{\equal{#1}{\empty}}%
  {%
    \parindent 0pt \textbf{#2} \par%
  }%
  {%
    \parindent 0pt \textbf{#1\hspace{0.5cm}#2} \par%
  }%
%  \vspace{0.1cm}%
  %
  \ifthenelse{\equal{#3}{Left}}%
    {\end{flushleft}\par}{}%
  \ifthenelse{\equal{#3}{Center}}%
    {\end{center}\par}{}%
  \ifthenelse{\equal{#3}{Right}}%
    {\end{flushright}\par}{}%
  %
  \addcontentsline{%
    toc}{subsection}{#1\GetSSTitleNumberFormatSepAtIndex{\ \ \ }#2}%
} % End of \newcommand{}

\RenewDocumentCommand{\subsection}{s m}
{%
  %
  % 檢查是 \subsection*{} 或是 \subsection{}
  \IfBooleanTF{#1}%
  {%
    % \subsection*{}
%        Starred
  }%
  {%
    % \subsection{}
%        Non-Starred
    %
    \ifthenelse{\equal{\GetStartAppendixChapter}{\ValueEnableAppendixChapter}}%
    {%
      \refstepcounter{appendixsubsection}%
      \SetupAppendixSubSectionTitleNumberFormatString%
      \DisplaySubSectionHeader{%
        \GetAppendixSubSectionTitleNumberFormatString}{#2}%
        {\GetAppendixSSTitleNumberFormatTextAlign}%
    }%
    {%
      \refstepcounter{subsection}%
      \SetupSubSectionTitleNumberFormatString%
      \DisplaySubSectionHeader{%
        \GetSubSectionTitleNumberFormatString}{#2}%
        {\GetSSTitleNumberFormatTextAlign}%
    }%
  }%
  % Reset counters
  \setcounter{subsubsection}{0}%
  \setcounter{appendixsubsubsection}{0}%
} % End of \RenewDocumentCommand{}

% ----------------------------------------------------------------------------

\newcommand{\DisplaySubSubSectionHeader}[3]%
{%
  \par%
  \vspace{0.5cm}%
  %
  \ifthenelse{\equal{#3}{Left}}%
    {\begin{flushleft}\par}{}%
  \ifthenelse{\equal{#3}{Center}}%
    {\begin{center}\par}{}%
  \ifthenelse{\equal{#3}{Right}}%
    {\begin{flushright}\par}{}%
  %
  \ifthenelse{\equal{#1}{\empty}}%
  {%
    \parindent 0pt \textbf{#2} \par%
  }%
  {%
    \parindent 0pt \textbf{#1\hspace{0.5cm}#2} \par%
  }%
%  \vspace{0.1cm}%
  %
  \ifthenelse{\equal{#3}{Left}}%
    {\end{flushleft}\par}{}%
  \ifthenelse{\equal{#3}{Center}}%
    {\end{center}\par}{}%
  \ifthenelse{\equal{#3}{Right}}%
    {\end{flushright}\par}{}%
  %
  \addcontentsline{%
    toc}{subsubsection}{#1\GetSSSTitleNumberFormatSepAtIndex{\ \ \ }#2}%
} % End of \newcommand{}

\RenewDocumentCommand{\subsubsection}{s m}
{%
  %
  % 檢查是 \subsubsection*{} 或是 \subsubsection{}
  \IfBooleanTF{#1}%
  {%
    % \subsubsection*{}
%        Starred
  }%
  {%
    % \subsubsection{}
%        Non-Starred
    %
    \ifthenelse{\equal{\GetStartAppendixChapter}{\ValueEnableAppendixChapter}}%
    {%
      \refstepcounter{appendixsubsubsection}%
      \SetupAppendixSubSubSectionTitleNumberFormatString%
      \DisplaySubSubSectionHeader{%
        \GetAppendixSubSubSectionTitleNumberFormatString}{#2}%
        {\GetAppendixSSSTitleNumberFormatTextAlign}%
    }%
    {%
      \refstepcounter{subsubsection}%
      \SetupSubSubSectionTitleNumberFormatString%
      \DisplaySubSubSectionHeader{%
        \GetSubSubSectionTitleNumberFormatString}{#2}%
        {\GetSSSTitleNumberFormatTextAlign}%
    }%
  }%
} % End of \RenewDocumentCommand{}

% ----------------------------------------------------------------------------
% Chapters (對應\chapter{})
\DeclareDocumentCommand{\StartChapter}{+m +g}
{%
  \StartNewPage%
  \chapter{#1}%
  %
  \ifthenelse{\equal{\GetStartAppendixChapter}{\ValueEnableAppendixChapter}}%
  {%
    \IfNoValueF{#2}{\LabelThisAs{#2}{\GetAppendixChapterNumberingFormatString}}%
  }%
  {%
    \IfNoValueF{#2}{\LabelThisAs{#2}{\GetGeneralChapterNumberingFormatString}}%
  }%
  %
  \pagestyle{plain}% Default page style
} % End of \DeclareDocumentCommand{}

% Chapters Star (對應\chapter*{})
\DeclareDocumentCommand{\StartChapterStar}{+m +g}
{%
  \StartNewPage%
  \chapter*{#1}%
  %
  \ifthenelse{\equal{\GetStartAppendixChapter}{\ValueEnableAppendixChapter}}%
  {%
    \IfNoValueF{#2}{\LabelThisAs{#2}{\GetAppendixChapterNumberingFormatString}}%
  }%
  {%
    \IfNoValueF{#2}{\LabelThisAs{#2}{\GetGeneralChapterNumberingFormatString}}%
  }%
  %
  \pagestyle{plain}% Default page style
} % End of \DeclareDocumentCommand{}

\newcommand{\EndChapter}
{%
  \EndOfPage%
  \UseDefaultLineStretch%
} % End of \newcommand{}

% Section
\DeclareDocumentCommand{\StartSection}{+m +g}
{%
  \section{#1}%
  %
  \ifthenelse{\equal{\GetStartAppendixChapter}{\ValueEnableAppendixChapter}}%
  {%
    \IfNoValueF{#2}{\LabelThisAs{#2}{\GetAppendixSectionNumberingFormatString}}%
  }%
  {%
    \IfNoValueF{#2}{\LabelThisAs{#2}{\GetGeneralSectionNumberingFormatString}}%
  }%
} % End of \DeclareDocumentCommand{}

% Sub-Section
\DeclareDocumentCommand{\StartSubSection}{+m +g}
{%
  \subsection{#1}%
  \ifthenelse{\equal{\GetStartAppendixChapter}{\ValueEnableAppendixChapter}}%
  {%
    \IfNoValueF{#2}{\LabelThisAs{#2}{%
      \GetAppendixSubSectionNumberingFormatString}}%
  }%
  {%
    \IfNoValueF{#2}{\LabelThisAs{#2}{%
      \GetGeneralSubSectionNumberingFormatString}}%
  }%
} % End of \DeclareDocumentCommand{}

% Sub-Sub-Section
\DeclareDocumentCommand{\StartSubSubSection}{+m +g}
{%
  \subsubsection{#1}%
  \ifthenelse{\equal{\GetStartAppendixChapter}{\ValueEnableAppendixChapter}}%
  {%
    \IfNoValueF{#2}{\LabelThisAs{#2}{%
      \GetAppendixSubSubSectionNumberingFormatString}}%
  }%
  {%
    \IfNoValueF{#2}{\LabelThisAs{#2}{%
      \GetGeneralSubSubSectionNumberingFormatString}}%
  }%
} % End of \DeclareDocumentCommand{}

% ----------------------------------------------------------------------------

% 過去的API, 以 Error提醒不能再使用
\newcommand{\ChapterTitleNumInChi}{\errmessage{模版: 由v1.4.1開始, ChapterTitleNumInChi已不能再使用. 請參考最新版的conf.tex使用方式.}\stop}

\newcommand{\ChapterTitleInChi}{\errmessage{模版: 由v1.4.4開始, ChapterTitleInChi已不能再使用. 請參考最新版的conf.tex使用方式.}\stop}

\newcommand{\ChapterSectionTitleInChi}{\errmessage{模版: 由v1.4.4開始, ChapterSectionTitleInChi已不能再使用. 請參考最新版的conf.tex使用方式.}\stop}

% ----------------------------------------------------------------------------

%
% This file is part of the project of
% National Cheng Kung University (NCKU) Thesis/Dissertation Template in LaTex.
% This project is hold at
%     <https://github.com/wengan-li/ncku-thesis-template-latex>
% by Wen-Gan Li.
%
% This project is distributed in the hope of usefuling to someone,
% you can redistribute it and/or modify it under the terms of the
% Attribution-NonCommercial-ShareAlike 4.0 International.
%
% You should have received a copy of the
% Attribution-NonCommercial-ShareAlike 4.0 International
% along with this project.
% If not, see <http://creativecommons.org/licenses/by-nc-sa/4.0/legalcode.txt>.
%
% Please feel free to fork it, modify it, and try it.
% Have fun !!!
%

% Some helper function use in abstract

% ----------------------------------------------------------------------------

% Abstract
\newcommand{\StartChiAbstract}{\StartAbstractChi}
\newcommand{\StartAbstractChi}
{%
  \ResetPageNumberCounting%
  \StartChapterStar{摘要}%
} % End of \newcommand{}

\newcommand{\StartAbstract}
{%
  \ResetPageNumberCounting%
  \StartChapterStar{Abstract}%
} % End of \newcommand{}

\newcommand{\EndAbstractChi}
{%
  % Keyword
  \ifthenelse{\equal{\GetAbstractChiKeywords}{\empty}}%
    {}{\par{\noindent \bf 關鍵字:} \GetAbstractChiKeywords}
  %
  \EndChapter%
} % End of \newcommand{}

\newcommand{\EndAbstract}
{%
  % Keyword
  \ifthenelse{\equal{\GetAbstractEngKeywords}{\empty}}%
    {}{\par{\noindent \bf Keyword:} \GetAbstractEngKeywords}
  %
  \EndChapter%
} % End of \newcommand{}

% ----------------------------------------------------------------------------

% 過去的API, 以 Error提醒不能再使用
\newcommand{\EndChiAbstract}{\errmessage{模版: 由v1.4.4開始, EndChiAbstract已不能再使用. 請改使用EndAbstractChi.}\stop}

% ----------------------------------------------------------------------------

%
% This file is part of the project of
% National Cheng Kung University (NCKU) Thesis/Dissertation Template in LaTex.
% This project is hold at
%     <https://github.com/wengan-li/ncku-thesis-template-latex>
% by Wen-Gan Li.
%
% This project is distributed in the hope of usefuling to someone,
% you can redistribute it and/or modify it under the terms of the
% Attribution-NonCommercial-ShareAlike 4.0 International.
%
% You should have received a copy of the
% Attribution-NonCommercial-ShareAlike 4.0 International
% along with this project.
% If not, see <http://creativecommons.org/licenses/by-nc-sa/4.0/legalcode.txt>.
%
% Please feel free to fork it, modify it, and try it.
% Have fun !!!
%

% Some helper function use in extended abstract

% ----------------------------------------------------------------------------

\def\ValueEnableExtendedAbstractFigureTableControl{1}
\def\ValueDisableExtendedAbstractFigureTableControl{0}
\def\VarStartExtendedAbstractFigureTableControl{%
  \ValueDisableExtendedAbstractFigureTableControl}
\def\BeginExtendedAbstractFigureTableControl{%
  \renewcommand{\VarStartExtendedAbstractFigureTableControl}{%
    \ValueEnableExtendedAbstractFigureTableControl}}
\def\EndExtendedAbstractFigureTableControl{%
  \renewcommand{\VarStartExtendedAbstractFigureTableControl}{%
    \ValueDisableExtendedAbstractFigureTableControl}}
\def\GetStartExtendedAbstractFigureTableControl{%
  \VarStartExtendedAbstractFigureTableControl}

% ----------------------------------------------------------------------------

% Extended Abstract
\newcommand{\StartExtendedAbstract}
{%
  \singlespacing%
  %
  \StartNewPage%
  %
  % Add to "Table of Contents"
%  \addcontentsline{toc}{chapter}{Extended Abstract}
  \addcontentsline{toc}{chapter}{英文延伸摘要}%
  %
  % Set style of caption for figure and table
  \clearcaptionsetup{table}
  \clearcaptionsetup{figure}
  \UseTableCaptionExtendedAbstractStyle%
  \UseFigureCaptionExtendedAbstractStyle%
  %
  \BeginExtendedAbstractFigureTableControl%
  %
  \UseTableNameDefault%
  \UseFigureNameDefault%
  %
  % -----------------------------------------------------------------
  %
  \begin{minipage}[c][5cm][c]{\textwidth}
  \parbox{\textwidth}{\center\large\textbf{\GetEngTitle}}
  \vspace{0.3cm}%
  \center\normalsize\GetAuthorEngName\par%
  \center\normalsize Dr. \GetAdvisorEngNameA\par%
  \ifthenelse{\equal{\GetAdvisorEngNameB}{\empty}}{}%
    {\center\normalsize Dr. \GetAdvisorEngNameB\par}
  \ifthenelse{\equal{\GetAdvisorEngNameC}{\empty}}{}%
    {\center\normalsize Dr. \GetAdvisorEngNameC\par}
  \center\normalsize\GetDeptEngName\par%
  \center\normalsize\GetCollEngName\par%
%  \center\normalsize\textit{\GetDeptEngName}\\%
%  \center\normalsize\textit{\GetCollEngName}\\%
  \end{minipage}%
  %
} % End of \newcommand{}

\newcommand{\EndExtendedAbstract}
{%
  \EndChapter%
  %
  \EndExtendedAbstractFigureTableControl%
  %
  \UseTableNameCustom%
  \UseFigureNameCustom%
  %
  % Reset figure and table counter to zero
  \setcounter{table}{0}%
  \setcounter{figure}{0}%
  %
  % Reset style of caption of figure and table
  \clearcaptionsetup{table}
  \clearcaptionsetup{figure}
  \UseTableCaptionDefaultStyle%
  \UseFigureCaptionDefaultStyle%
  \SetupNumberingFormat%
  %
  \UseDefaultLineStretch%
} % End of \newcommand{}

% Summary in Extended Abstract
\global\mdfdefinestyle{ExtAbstractSummaryStyle}{%
  linewidth=1pt, apptotikzsetting={%
    \tikzset{mdfbackground/.append style={opacity=0.4}}}%
} % End of \mdfdefinestyle{}
\newcommand{\ExtAbstractSummary}[1]
{%
  \par%
  \vspace{0.3cm}%
  \begin{mdframed}[style=ExtAbstractSummaryStyle]%
  \vspace{0.3cm}%
  \parbox{\textwidth}{\center\textbf{SUMMARY}}%
  \vspace{0.3cm}\par%
  #1%
  \EmptyLine%
%  \vspace{0.3cm}\par%
  \textbf{Keyword:} \GetAbstractExtKeywords%
  \end{mdframed}%
} % End of \newcommand{}

% Chapter in Extended Abstract
%\newcommand{\ExtAbstractChapter}[1]
\DeclareDocumentCommand{\ExtAbstractChapter}{+m +g} % Back-porting
{%
  \par%
  \vspace{0.5cm}%
  \centerline{\textbf{\MakeUppercase{#1}}}\par%
  \vspace{0.3cm}%
  \IfNoValueF{#2}{#2}%
} % End of \newcommand{}

% Section in Extended Abstract
%\newcommand{\ExtAbstractSection}[1]
\DeclareDocumentCommand{\ExtAbstractSection}{+m +g} % Back-porting
{%
  \par%
  \vspace{0.3cm}%
  \textbf{#1}\par%
  \vspace{0.1cm}%
  \IfNoValueF{#2}{#2}%
} % End of \newcommand{}

% ----------------------------------------------------------------------------

%
% This file is part of the project of
% National Cheng Kung University (NCKU) Thesis/Dissertation Template in LaTex.
% This project is hold at
%     <https://github.com/wengan-li/ncku-thesis-template-latex>
% by Wen-Gan Li.
%
% This project is distributed in the hope of usefuling to someone,
% you can redistribute it and/or modify it under the terms of the
% Attribution-NonCommercial-ShareAlike 4.0 International.
%
% You should have received a copy of the
% Attribution-NonCommercial-ShareAlike 4.0 International
% along with this project.
% If not, see <http://creativecommons.org/licenses/by-nc-sa/4.0/legalcode.txt>.
%
% Please feel free to fork it, modify it, and try it.
% Have fun !!!
%

% Some helper function about acknowledgments

% ----------------------------------------------------------------------------

\newcommand{\StartAcknowledgments}
{%
  \StartChapterStar{Acknowledgements}%
} % End of \newcommand{}

\newcommand{\StartAcknowledgmentsChi}
{%
  \StartChapterStar{誌謝}%
} % End of \newcommand{}

\newcommand{\EndAcknowledgments}{\EndChapter}

% ----------------------------------------------------------------------------

%
% This file is part of the project of
% National Cheng Kung University (NCKU) Thesis/Dissertation Template in LaTex.
% This project is hold at
%     <https://github.com/wengan-li/ncku-thesis-template-latex>
% by Wen-Gan Li.
%
% This project is distributed in the hope of usefuling to someone,
% you can redistribute it and/or modify it under the terms of the
% Attribution-NonCommercial-ShareAlike 4.0 International.
%
% You should have received a copy of the
% Attribution-NonCommercial-ShareAlike 4.0 International
% along with this project.
% If not, see <http://creativecommons.org/licenses/by-nc-sa/4.0/legalcode.txt>.
%
% Please feel free to fork it, modify it, and try it.
% Have fun !!!
%

% Some helper function about appendix

% ----------------------------------------------------------------------------

\newcounter{appendixchapter}
\newcounter{appendixsection}
\newcounter{appendixsubsection}
\newcounter{appendixsubsubsection}

% ----------------------------------------------------------------------------

%\def\VarAppendixTitleFormatPrefix{Appendix}
%\def\VarAppendixTitleFormatSuffix{}

%\newcommand{\VarAppendixTitleFormatFinalString}{}%
%\newcommand{\InitAppendixTitleFormatFinalString}{%
%  \renewcommand{\VarAppendixTitleFormatFinalString}{}}%

%\makeatletter
%\newcommand{\AppendAppendixTitleFormatFinalString}[1]%
%{%
%  \g@addto@macro\VarAppendixTitleFormatFinalString{#1}%
%} % End of \newcommand{}
%\makeatother

%\newcommand{\SetAppendixTitleFormatFinalString}%
%{%
%  \InitAppendixTitleFormatFinalString%
%  \AppendAppendixTitleFormatFinalString{\thechapter}%
%  \ifthenelse{\equal{\VarAppendixTitleFormatPrefix}{\empty}}%
%  {}{\AppendAppendixTitleFormatFinalString{\VarAppendixTitleFormatPrefix\space}}%
%  \AppendAppendixTitleFormatFinalString{\theappendixchapter}%
%  \ifthenelse{\equal{\VarAppendixTitleFormatSuffix}{\empty}}%
%  {}{\AppendAppendixTitleFormatFinalString{\space\VarAppendixTitleFormatSuffix}}%
%} % End of \newcommand{}

%\newcommand{\GetAppendixTitleFormatFinalString}{%
%  \VarAppendixTitleFormatFinalString}

% ----------------------------------------------------------------------------

%\def\VarAppendixTitleEng{Appendix}
%\def\VarAppendixTitleChi{附錄}
%\def\GetAppendixTitle{\VarAppendixTitleEng}

\def\ValueEnableAppendixChapter{1}
\def\ValueDisableAppendixChapter{0}
\def\VarStartAppendixChapter{\ValueDisableAppendixChapter}
\def\BeginAppendixChapter{%
  \renewcommand{\VarStartAppendixChapter}{\ValueEnableAppendixChapter}}
\def\GetStartAppendixChapter{\VarStartAppendixChapter}

% ----------------------------------------------------------------------------

\newcommand{\StartAppendix}
{%
  \BeginAppendixChapter%
  %
  % Set numbering format style
  \SetupAppendixNumberingFormat%
  %
  \StartNewPage%
} % End of \newcommand{}

\newcommand{\EndAppendix}{\EndChapter}

% ----------------------------------------------------------------------------

%
% This file is part of the project of
% National Cheng Kung University (NCKU) Thesis/Dissertation Template in LaTex.
% This project is hold at
%     <https://github.com/wengan-li/ncku-thesis-template-latex>
% by Wen-Gan Li.
%
% This project is distributed in the hope of usefuling to someone,
% you can redistribute it and/or modify it under the terms of the
% Attribution-NonCommercial-ShareAlike 4.0 International.
%
% You should have received a copy of the
% Attribution-NonCommercial-ShareAlike 4.0 International
% along with this project.
% If not, see <http://creativecommons.org/licenses/by-nc-sa/4.0/legalcode.txt>.
%
% Please feel free to fork it, modify it, and try it.
% Have fun !!!
%

% Some helper function about references and bibliography
% 參考文獻或資料

% Useful URL:
% https://www.sharelatex.com/learn/Bibtex_bibliography_styles
% http://ftp.isu.edu.tw/pub/Unix/CTAN/biblio/bibtex/contrib/apacite/apacite.pdf
% https://en.wikibooks.org/wiki/LaTeX/Bibliography_Management
% https://verbosus.com/bibtex-style-examples.html
% https://www.sharelatex.com/learn/Bibliography_management_with_bibtex
%     -- Style Name --    | 	 -- Author Name Format --  |  -- Reference Format --  |  -- Sorting --
%        plain                     |	       Homer Jay Simpson 	     |                  #ID# 	                  |   by author
%        unsrt                    |	        Homer Jay Simpson       |	                 #ID#                     |	   as referenced
%        abbrv                  |	             H. J. Simpson               |	                #ID#                     |	   by author
%       alpha                   |	       Homer Jay Simpson       |	   Sim95                              |	   by author
%      abstract              |	       Homer Jay Simpson      |	   Simpson-1995a
%         acm                  |	            Simpson, H. J.              |	   #ID#
%    authordate1      |	      Simpson, Homer Jay      |	   Simpson, 1995
%         apa                  |	       Simpson, H. J. (1995)      |	   Simpson1995
%        named            |	         Homer Jay Simpson     |	   Simpson 1995

% 使用package 'apacite' 的話
%         apacite                  |	       Ackerman, 1990

% ----------------------------------------------------------------------------

\newcommand{\TextDefaultTitleReferenceChi}{參考文獻}
\newcommand{\TextDefaultTitleReferenceEng}{References}
\newcommand{\TextDefaultTitleBibliographyEng}{Bibliography}

% ----------------------------------------------------------------------------

\begin{comment}
% Bibliography style 使用的變數
\newcommand{\BibStyleTypeAbbrv}{1}
\newcommand{\BibStyleTypePlain}{2}
\newcommand{\BibStyleTypeAlpha}{3}
\newcommand{\BibStyleTypeApacite}{4}
\newcommand{\BibStyleTypeVar}{\BibStyleTypePlain} % Default style: 'plain'
\newcommand{\SetBibStyleTypeVar}[1]{\renewcommand{\BibStyleTypeVar}{#1}}
\newcommand{\GetBibStyleTypeVar}{\BibStyleTypeVar}

% 公開使用的APIs
\newcommand{\BibStyleUseAbbrv}{\SetBibStyleTypeVar{\BibStyleTypeAbbrv}}
\newcommand{\BibStyleUsePlain}{\SetBibStyleTypeVar{\BibStyleTypePlain}}
\newcommand{\BibStyleUseAlpha}{\SetBibStyleTypeVar{\BibStyleTypeAlpha}}
\newcommand{\BibStyleUseApacite}{\SetBibStyleTypeVar{\BibStyleTypeApacite}}

% ----------------------------------------------------------------------------

\newcommand{\VarReferenceTitle}{\VarReferenceTitleCustom}
\newcommand{\SetReferenceTitle}[1]{\renewcommand{\VarReferenceTitle}{#1}}
\newcommand{\GetReferenceTitle}{\VarReferenceTitle}

\newcommand{\VarReferenceTitleChi}{參考文獻}
\newcommand{\VarReferenceTitleEng}{References}
\newcommand{\VarReferenceTitleCustom}{References / 參考文獻}

\newcommand{\ChapterReferenceTitleInChi}{%
  \SetReferenceTitle{\VarReferenceTitleChi}}
\newcommand{\ChapterReferenceTitleInEng}{%
  \SetReferenceTitle{\VarReferenceTitleEng}}
\newcommand{\SetChapterReferenceTitle}[1]{\SetReferenceTitle{#1}}

\ChapterReferenceTitleInEng % Default
\end{comment}
% ----------------------------------------------------------------------------

\DeclareDocumentCommand{\ReferencesFiles}{%
  G{\empty} G{\empty} G{\empty} %
  G{\empty} G{\empty} G{\empty} %
  G{\empty} G{\empty} G{\empty}}
{%
  \singlespacing%
  \StartNewPage%
  % ------------------------------------------------
  %
  \renewcommand\bibname{\GetReferenceTitle}
  %
  \bibliographystyle{\GetReferenceBibStyle}
  %
  % ------------------------------------------------
  % Bib files
  \ifthenelse{\equal{#9}{\empty}}
  {%
    \ifthenelse{\equal{#8}{\empty}}
    {%
      \ifthenelse{\equal{#7}{\empty}}
      {%
        \ifthenelse{\equal{#6}{\empty}}
        {%
          \ifthenelse{\equal{#5}{\empty}}
          {%
            \ifthenelse{\equal{#4}{\empty}}
            {%
              \ifthenelse{\equal{#3}{\empty}}
              {%
                \ifthenelse{\equal{#2}{\empty}}
                {%
                  \ifthenelse{\equal{#1}{\empty}}
                  {} % End of if{}
                  {\bibliography{#1}} % End of else{}
                } % End of if{}
                {%
                  \bibliography{#1,#2}%
                } % End of else{}
              } % End of if{}
              {%
                \bibliography{#1,#2,#3}%
              } % End of else{}
            } % End of if{}
            {%
              \bibliography{#1,#2,#3,#4}%
            } % End of else{}
          } % End of if{}
          {%
            \bibliography{#1,#2,#3,#4,#5}%
          } % End of else{}
        } % End of if{}
        {%
          \bibliography{#1,#2,#3,#4,#5,#6}%
        } % End of else{}
      } % End of if{}
      {%
        \bibliography{#1,#2,#3,#4,#5,#6,#7}%
      } % End of else{}
    } % End of if{}
    {%
      \bibliography{#1,#2,#3,#4,#5,#6,#7,#8}%
    } % End of else{}
  } % End of if{}
  {%
    \bibliography{#1,#2,#3,#4,#5,#6,#7,#8,#9}%
  } % End of else{}
  % ------------------------------------------------
  \EndChapter%
} % End of \DeclareDocumentCommand{}

% ----------------------------------------------------------------------------

\pgfkeys
{
  /SetupReference/.is family, /SetupReference,
  default/.style =
  {
    Title = {\TextDefaultTitleReferenceEng},
    BibStyle = {plain},
  },
  Title/.estore in = \GetReferenceTitle,
  BibStyle/.estore in = \GetReferenceBibStyle,
} % End of \pgfkeys{}

\newcommand\SetupReference[1]
{%
  \pgfkeys{/SetupReference, default, #1}%
  %
  % Include package<apacite> if needed
  \ifthenelse{\equal{\GetReferenceBibStyle}{apacite}}%
  {\usepackage[notocbib]{apacite}}{}
} % End of \newcommand{}

% -----------------------------------------------

% Default
% Title: References
% Style: plain
\SetupReference{%
  Title = {\TextDefaultTitleReferenceEng},
  BibStyle = {plain},
} % End of \SetupReference{}

% ----------------------------------------------------------------------------

\newcommand{\SetChapterReferenceTitle}{\errmessage{模版: 由v1.4.6開始, SetChapterReferenceTitle已不能再使用. 請參考最新版的conf.tex中SetupReference的使用方式.}\stop}

\newcommand{\ChapterReferenceTitleInChi}{\errmessage{模版: 由v1.4.6開始, ChapterReferenceTitleInChi已不能再使用. 請參考最新版的conf.tex中SetupReference的使用方式.}\stop}

\newcommand{\ChapterReferenceTitleInEng}{\errmessage{模版: 由v1.4.6開始, ChapterReferenceTitleInEng已不能再使用. 請參考最新版的conf.tex中SetupReference的使用方式.}\stop}

\newcommand{\BibStyleUseAbbrv}{\errmessage{模版: 由v1.4.6開始, BibStyleUseAbbrv已不能再使用. 請參考最新版的conf.tex中SetupReference的使用方式.}\stop}

\newcommand{\BibStyleUsePlain}{\errmessage{模版: 由v1.4.6開始, BibStyleUsePlain已不能再使用. 請參考最新版的conf.tex中SetupReference的使用方式.}\stop}

\newcommand{\BibStyleUseAlpha}{\errmessage{模版: 由v1.4.6開始, BibStyleUseAlpha已不能再使用. 請參考最新版的conf.tex中SetupReference的使用方式.}\stop}

\newcommand{\BibStyleUseApacite}{\errmessage{模版: 由v1.4.6開始, BibStyleUseApacite已不能再使用. 請參考最新版的conf.tex中SetupReference的使用方式.}\stop}

%
% This file is part of the project of
% National Cheng Kung University (NCKU) Thesis/Dissertation Template in LaTex.
% This project is hold at
%     <https://github.com/wengan-li/ncku-thesis-template-latex>
% by Wen-Gan Li.
%
% This project is distributed in the hope of usefuling to someone,
% you can redistribute it and/or modify it under the terms of the
% Attribution-NonCommercial-ShareAlike 4.0 International.
%
% You should have received a copy of the
% Attribution-NonCommercial-ShareAlike 4.0 International
% along with this project.
% If not, see <http://creativecommons.org/licenses/by-nc-sa/4.0/legalcode.txt>.
%
% Please feel free to fork it, modify it, and try it.
% Have fun !!!
%

% Some helper function use in index

% ----------------------------------------------------------------------------

\newcommand\VarIndexTitleChi{目錄}
\newcommand\VarIndexTitleEng{Table of Contents}
\newcommand\VarIndexTitleText{\VarIndexTitleEng}  % Default
\newcommand\GetIndexTitleText{\VarIndexTitleText}
\newcommand{\SetIndexTitleText}[1]
  {\renewcommand{\VarIndexTitleText}{#1}}

% ----------------------------------------------------------------------------

\newcommand\VarTablesIndexTitleChi{表格}
\newcommand\VarTablesIndexTitleEng{List of Tables}
\newcommand\VarTablesIndexTitleText{\VarTablesIndexTitleEng}  % Default
\newcommand\GetTablesIndexTitleText{\VarTablesIndexTitleText}
\newcommand{\SetTablesIndexTitleText}[1]
  {\renewcommand{\VarTablesIndexTitleText}{#1}}

% ----------------------------------------------------------------------------

\newcommand\VarFiguresIndexTitleChi{圖片}
\newcommand\VarFiguresIndexTitleEng{List of Figures}
\newcommand\VarFiguresIndexTitleText{\VarFiguresIndexTitleEng}  % Default
\newcommand\GetFiguresIndexTitleText{\VarFiguresIndexTitleText}
\newcommand{\SetFiguresIndexTitleText}[1]
  {\renewcommand{\VarFiguresIndexTitleText}{#1}}

% ----------------------------------------------------------------------------

\newcommand{\IndexChiMode}{%
  \SetIndexTitleText{\VarIndexTitleChi}%
  \SetTablesIndexTitleText{\VarTablesIndexTitleChi}%
  \SetFiguresIndexTitleText{\VarFiguresIndexTitleChi}%
} % End of \newcommand{}

\newcommand{\IndexEngMode}{%
  \SetIndexTitleText{\VarIndexTitleEng}%
  \SetTablesIndexTitleText{\VarTablesIndexTitleEng}%
  \SetFiguresIndexTitleText{\VarFiguresIndexTitleEng}%
} % End of \newcommand{}

\IndexEngMode % Default

% ----------------------------------------------------------------------------

\makeatletter

\renewcommand\listoftables
{%
  \singlespacing%
  \StartChapterStar{\VarTablesIndexTitleText}%
  \@starttoc{lot}%
  \EndChapter%
} % End of \renewcommand{}

\renewcommand\listoffigures
{%
  \singlespacing%
  \StartChapterStar{\VarFiguresIndexTitleText}%
  \@starttoc{lof}%
  \EndChapter%
} % End of \renewcommand{}

\makeatother

% ----------------------------------------------------------------------------

% 公開使用的APIs
\newcommand{\DisplayIndex}{\tableofcontents}
\newcommand{\DisplayTablesIndex}{\listoftables}
\newcommand{\DisplayFiguresIndex}{\listoffigures}

% ----------------------------------------------------------------------------

%
% This file is part of the project of
% National Cheng Kung University (NCKU) Thesis/Dissertation Template in LaTex.
% This project is hold at
%     <https://github.com/wengan-li/ncku-thesis-template-latex>
% by Wen-Gan Li.
%
% This project is distributed in the hope of usefuling to someone,
% you can redistribute it and/or modify it under the terms of the
% Attribution-NonCommercial-ShareAlike 4.0 International.
%
% You should have received a copy of the
% Attribution-NonCommercial-ShareAlike 4.0 International
% along with this project.
% If not, see <http://creativecommons.org/licenses/by-nc-sa/4.0/legalcode.txt>.
%
% Please feel free to fork it, modify it, and try it.
% Have fun !!!
%

% Some helper function about watermark

% ----------------------------------------------------------------------------

% Chapters
\DeclareDocumentCommand{\StartNomChapter}{+m +g}
{%
  \StartChapterStar{#1}{#2}%
} % End of \DeclareDocumentCommand{}

\newcommand{\EndNomChapter}{\EndChapter}

% ----------------------------------------------------------------------------

%
% This file is part of the project of
% National Cheng Kung University (NCKU) Thesis/Dissertation Template in LaTex.
% This project is hold at
%     <https://github.com/wengan-li/ncku-thesis-template-latex>
% by Wen-Gan Li.
%
% This project is distributed in the hope of usefuling to someone,
% you can redistribute it and/or modify it under the terms of the
% Attribution-NonCommercial-ShareAlike 4.0 International.
%
% You should have received a copy of the
% Attribution-NonCommercial-ShareAlike 4.0 International
% along with this project.
% If not, see <http://creativecommons.org/licenses/by-nc-sa/4.0/legalcode.txt>.
%
% Please feel free to fork it, modify it, and try it.
% Have fun !!!
%

% ----------------------------------------------------------------------------

% Reference from:
%   <https://www.sharelatex.com/learn/Theorems_and_proofs>

%    Definition       (定義)
%    Condition        (條件)
%    Theorem          (定理)
%    Lemma            (引理)
%    Example          (例子)
%    Corollary        (推論)
%    Proposition      (主張)
%    Remark           (備註)
%    Proof            (證明)
%    Conjecture       (猜想)
%    Note             (附註)
%    Annotation       (註解)
%    Claim            (主張)
%    Case             (情況)
%    Acknowledgment   (確認)
%    Conclusion       (結論)
%    Criterion        (標準)
%    Assertion        (斷言)
%    Problem          (問題)
%    Question         (問題)
%    Hypothesis       (假設)
%    Summary          (總結)

% ----------------------------------------------------------------------------

% Shared functions

\pgfkeys
{
  /InsertTheoremOptions/.is family, /InsertTheoremOptions,
  default/.style =
  {
    title = \empty,
    label = \empty,
  },
  title/.estore in = \TmpValueTitle,
  label/.estore in = \TmpValueLabel,
} % End of \pgfkeys{}

\newcommand{\SetTheoremContentLabel}[1][\empty]
{%
  \ifthenelse{\equal{#1}{\empty}}{}{\IfNoValueF{#1}{\label{#1}}}%
} % End of \DeclareDocumentCommand{}

\newcommand{\InsertTheoremContent}[3][\empty]
{%
  % Parse the input
  \pgfkeys{/InsertTheoremOptions, default, #1}%
  %
  \ifthenelse{\equal{\TmpValueTitle}{\empty}}%
  {%
    \begin{#2}%
  }%
  {%
    \begin{#2}[\TmpValueTitle]%
  }%
  %
  \SetTheoremContentLabel{\TmpValueLabel}%
  %
  #3%
  %
  \ifthenelse{\equal{#2}{\GetTheoremProofFormatEnvironmentName}}%
    {$\quad\blacksquare$}{}%
  %
  \end{#2}%
} % End of \newcommand{}

\newcommand{\MappingTheoremCounterResetCounter}[2][\empty]
{%
  \ifthenelse{\equal{#1}{Definition}}%
  {%
    \renewcommand{\GetTheoremDefinitionFormatFollowCounter}{#2}%
  }{}%
  %
  \ifthenelse{\equal{#1}{Condition}}%
  {%
    \renewcommand{\GetTheoremConditionFormatFollowCounter}{#2}%
  }{}%
  %
  \ifthenelse{\equal{#1}{Theorem}}%
  {%
    \renewcommand{\GetTheoremTheoremFormatFollowCounter}{#2}%
  }{}%
  %
  \ifthenelse{\equal{#1}{Lemma}}%
  {%
    \renewcommand{\GetTheoremLemmaFormatFollowCounter}{#2}%
  }{}%
  %
  \ifthenelse{\equal{#1}{Example}}%
  {%
    \renewcommand{\GetTheoremExampleFormatFollowCounter}{#2}%
  }{}%
  %
  \ifthenelse{\equal{#1}{Problem}}%
  {%
    \renewcommand{\GetTheoremProblemFormatFollowCounter}{#2}%
  }{}%
  %
  \ifthenelse{\equal{#1}{Corollary}}%
  {%
    \renewcommand{\GetTheoremCorollaryFormatFollowCounter}{#2}%
  }{}%
  %
  \ifthenelse{\equal{#1}{Proposition}}%
  {%
    \renewcommand{\GetTheoremPropositionFormatFollowCounter}{#2}%
  }{}%
  %
  \ifthenelse{\equal{#1}{Conjecture}}%
  {%
    \renewcommand{\GetTheoremConjectureFormatFollowCounter}{#2}%
  }{}%
  %
  \ifthenelse{\equal{#1}{Criterion}}%
  {%
    \renewcommand{\GetTheoremCriterionFormatFollowCounter}{#2}%
  }{}%
  %
  \ifthenelse{\equal{#1}{Assertion}}%
  {%
    \renewcommand{\GetTheoremAssertionFormatFollowCounter}{#2}%
  }{}%
  %
  \ifthenelse{\equal{#1}{Question}}%
  {%
    \renewcommand{\GetTheoremQuestionFormatFollowCounter}{#2}%
  }{}%
  %
  \ifthenelse{\equal{#1}{Hypothesis}}%
  {%
    \renewcommand{\GetTheoremHypothesisFormatFollowCounter}{#2}%
  }{}%
  %
} % End of \newcommand{}

\newcommand{\MappingTheoremCounter}[2][\empty]
{%
  \ifthenelse{\equal{#2}{Section}}%
  {%
    \MappingTheoremCounterResetCounter[#1]{section}%
  }{}%
  %
  \ifthenelse{\equal{#2}{Definition}}%
  {%
    \MappingTheoremCounterResetCounter[#1]{%
      \GetTheoremDefinitionFormatFollowCounter}%
  }{}%
  %
  \ifthenelse{\equal{#2}{Condition}}%
  {%
    \MappingTheoremCounterResetCounter[#1]{%
      \GetTheoremConditionFormatFollowCounter}%
  }{}%
  %
  \ifthenelse{\equal{#2}{Theorem}}%
  {%
    \MappingTheoremCounterResetCounter[#1]{%
      \GetTheoremTheoremFormatFollowCounter}%
  }{}%
  %
  \ifthenelse{\equal{#2}{Lemma}}%
  {%
    \MappingTheoremCounterResetCounter[#1]{%
      \GetTheoremLemmaFormatFollowCounter}%
  }{}%
  %
  \ifthenelse{\equal{#2}{Example}}%
  {%
    \MappingTheoremCounterResetCounter[#1]{%
      \GetTheoremExampleFormatFollowCounter}%
  }{}%
  %
  \ifthenelse{\equal{#2}{Proposition}}%
  {%
    \MappingTheoremCounterResetCounter[#1]{%
      \GetTheoremPropositionFormatFollowCounter}%
  }{}%
  %
  \ifthenelse{\equal{#2}{Conjecture}}%
  {%
    \MappingTheoremCounterResetCounter[#1]{%
      \GetTheoremConjectureFormatFollowCounter}%
  }{}%
  %
  \ifthenelse{\equal{#2}{Criterion}}%
  {%
    \MappingTheoremCounterResetCounter[#1]{%
      \GetTheoremCriterionFormatFollowCounter}%
  }{}%
  %
  \ifthenelse{\equal{#2}{Assertion}}%
  {%
    \MappingTheoremCounterResetCounter[#1]{%
      \GetTheoremAssertionFormatFollowCounter}%
  }{}%
  %
  \ifthenelse{\equal{#2}{Question}}%
  {%
    \MappingTheoremCounterResetCounter[#1]{%
      \GetTheoremQuestionFormatFollowCounter}%
  }{}%
  %
  \ifthenelse{\equal{#2}{Hypothesis}}%
  {%
    \MappingTheoremCounterResetCounter[#1]{%
      \GetTheoremHypothesisFormatFollowCounter}%
  }{}%
  %
  \ifthenelse{\equal{#2}{Problem}}%
  {%
    \MappingTheoremCounterResetCounter[#1]{%
      \GetTheoremProblemFormatFollowCounter}%
  }{}%
  %
  \ifthenelse{\equal{#2}{Corollary}}%
  {%
    \MappingTheoremCounterResetCounter[#1]{%
      \GetTheoremCorollaryFormatFollowCounter}%
  }{}%
  %
} % End of \newcommand{}

% ---------------------------------------------------------

\pgfkeys
{
  /TheoremTheoremFormat/.is family, /TheoremTheoremFormat,
  default/.style =
  {
    EnvironmentName = {EnvTheorem},
    ShowText = {Theorem},
    FollowCounter = Section,
  },
  EnvironmentName/.estore in = \GetTheoremTheoremFormatEnvironmentName,
  ShowText/.estore in = \GetTheoremTheoremFormatShowText,
  FollowCounter/.estore in = \GetTheoremTheoremFormatFollowCounter,
} % End of \pgfkeys{}

\newcommand{\InsertTheorem}[2][\empty]
{%
  \InsertTheoremContent[#1]{\GetTheoremTheoremFormatEnvironmentName}{#2}%
} % End of \newcommand{}

\newcommand{\InitTheoremTheoremFormat}
{%
  \theoremstyle{plain}%
  \ifthenelse{\equal{\GetTheoremTheoremFormatFollowCounter}{\empty}}%
  {%
    \newtheorem{%
      \GetTheoremTheoremFormatEnvironmentName}{%
      \GetTheoremTheoremFormatShowText}%
  }%
  {%
    %
    \MappingTheoremCounter[Theorem]{\GetTheoremTheoremFormatFollowCounter}%
    %
    \newtheorem{%
      \GetTheoremTheoremFormatEnvironmentName}{%
      \GetTheoremTheoremFormatShowText}[section]%
  }%
} % End of \newcommand{}

% ---------------------------------------------------------

\pgfkeys
{
  /TheoremDefinitionFormat/.is family, /TheoremDefinitionFormat,
  default/.style =
  {
    EnvironmentName = {EnvDefinition},
    ShowText = {Definition},
    FollowCounter = Section,
  },
  EnvironmentName/.estore in = \GetTheoremDefinitionFormatEnvironmentName,
  ShowText/.estore in = \GetTheoremDefinitionFormatShowText,
  FollowCounter/.estore in = \GetTheoremDefinitionFormatFollowCounter,
} % End of \pgfkeys{}

\newcommand{\InsertDefinition}[2][\empty]
{%
  \InsertTheoremContent[#1]{\GetTheoremDefinitionFormatEnvironmentName}{#2}%
} % End of \newcommand{}

\newcommand{\InitTheoremDefinitionFormat}
{%
  \theoremstyle{definition}%
  \ifthenelse{\equal{\GetTheoremDefinitionFormatFollowCounter}{\empty}}%
  {%
    \newtheorem{%
      \GetTheoremDefinitionFormatEnvironmentName}{%
      \GetTheoremDefinitionFormatShowText}%
  }%
  {%
    %
    \MappingTheoremCounter[Definition]{\GetTheoremDefinitionFormatFollowCounter}%
    %
    \newtheorem{%
      \GetTheoremDefinitionFormatEnvironmentName}{%
      \GetTheoremDefinitionFormatShowText}[%
      \GetTheoremDefinitionFormatFollowCounter]%
  }%
} % End of \newcommand{}

% ---------------------------------------------------------

\pgfkeys
{
  /TheoremConditionFormat/.is family, /TheoremConditionFormat,
  default/.style =
  {
    EnvironmentName = {EnvCondition},
    ShowText = {Condition},
    FollowCounter = Section,
  },
  EnvironmentName/.estore in = \GetTheoremConditionFormatEnvironmentName,
  ShowText/.estore in = \GetTheoremConditionFormatShowText,
  FollowCounter/.estore in = \GetTheoremConditionFormatFollowCounter,
} % End of \pgfkeys{}

\newcommand{\InsertCondition}[2][\empty]
{%
  \InsertTheoremContent[#1]{\GetTheoremConditionFormatEnvironmentName}{#2}%
} % End of \newcommand{}

\newcommand{\InitTheoremConditionFormat}
{%
  \theoremstyle{definition}%
  \ifthenelse{\equal{\GetTheoremConditionFormatFollowCounter}{\empty}}%
  {%
    \newtheorem{%
      \GetTheoremConditionFormatEnvironmentName}{%
      \GetTheoremConditionFormatShowText}%
  }%
  {%
    %
    \MappingTheoremCounter[Condition]{\GetTheoremConditionFormatFollowCounter}%
    %
    \newtheorem{%
      \GetTheoremConditionFormatEnvironmentName}{%
      \GetTheoremConditionFormatShowText}[%
      \GetTheoremConditionFormatFollowCounter]%
  }%
} % End of \newcommand{}

% ---------------------------------------------------------

\pgfkeys
{
  /TheoremProblemFormat/.is family, /TheoremProblemFormat,
  default/.style =
  {
    EnvironmentName = {EnvProblem},
    ShowText = {Problem},
    FollowCounter = Section,
  },
  EnvironmentName/.estore in = \GetTheoremProblemFormatEnvironmentName,
  ShowText/.estore in = \GetTheoremProblemFormatShowText,
  FollowCounter/.estore in = \GetTheoremProblemFormatFollowCounter,
} % End of \pgfkeys{}

\newcommand{\InsertProblem}[2][\empty]
{%
  \InsertTheoremContent[#1]{\GetTheoremProblemFormatEnvironmentName}{#2}%
} % End of \newcommand{}

\newcommand{\InitTheoremProblemFormat}
{%
  \theoremstyle{definition}%
  \ifthenelse{\equal{\GetTheoremProblemFormatFollowCounter}{\empty}}%
  {%
    \newtheorem{%
      \GetTheoremProblemFormatEnvironmentName}{%
      \GetTheoremProblemFormatShowText}%
  }%
  {%
    %
    \MappingTheoremCounter[Problem]{\GetTheoremProblemFormatFollowCounter}%
    %
    \newtheorem{%
      \GetTheoremProblemFormatEnvironmentName}{%
      \GetTheoremProblemFormatShowText}[%
      \GetTheoremProblemFormatFollowCounter]%
  }%
} % End of \newcommand{}

% ---------------------------------------------------------

\pgfkeys
{
  /TheoremExampleFormat/.is family, /TheoremExampleFormat,
  default/.style =
  {
    EnvironmentName = {EnvExample},
    ShowText = {Example},
    FollowCounter = Section,
  },
  EnvironmentName/.estore in = \GetTheoremExampleFormatEnvironmentName,
  ShowText/.estore in = \GetTheoremExampleFormatShowText,
  FollowCounter/.estore in = \GetTheoremExampleFormatFollowCounter,
} % End of \pgfkeys{}

\newcommand{\InsertExample}[2][\empty]
{%
  \InsertTheoremContent[#1]{\GetTheoremExampleFormatEnvironmentName}{#2}%
} % End of \newcommand{}

\newcommand{\InitTheoremExampleFormat}
{%
  \theoremstyle{definition}%
  \ifthenelse{\equal{\GetTheoremExampleFormatFollowCounter}{\empty}}%
  {%
    \newtheorem{%
      \GetTheoremExampleFormatEnvironmentName}{%
      \GetTheoremExampleFormatShowText}%
  }%
  {%
    %
    \MappingTheoremCounter[Example]{\GetTheoremExampleFormatFollowCounter}%
    %
    \newtheorem{%
      \GetTheoremExampleFormatEnvironmentName}{%
      \GetTheoremExampleFormatShowText}[%
      \GetTheoremExampleFormatFollowCounter]%
  }%
} % End of \newcommand{}

% ---------------------------------------------------------

\pgfkeys
{
  /TheoremLemmaFormat/.is family, /TheoremLemmaFormat,
  default/.style =
  {
    EnvironmentName = {EnvLemma},
    ShowText = {Lemma},
    FollowCounter = Section,
  },
  EnvironmentName/.estore in = \GetTheoremLemmaFormatEnvironmentName,
  ShowText/.estore in = \GetTheoremLemmaFormatShowText,
  FollowCounter/.estore in = \GetTheoremLemmaFormatFollowCounter,
} % End of \pgfkeys{}

\newcommand{\InsertLemma}[2][\empty]
{%
  \InsertTheoremContent[#1]{\GetTheoremLemmaFormatEnvironmentName}{#2}%
} % End of \newcommand{}

\newcommand{\InitTheoremLemmaFormat}
{%
  \theoremstyle{plain}%
  \ifthenelse{\equal{\GetTheoremLemmaFormatFollowCounter}{\empty}}%
  {%
    \newtheorem{%
      \GetTheoremLemmaFormatEnvironmentName}{%
      \GetTheoremLemmaFormatShowText}%
  }%
  {%
    %
    \MappingTheoremCounter[Lemma]{\GetTheoremLemmaFormatFollowCounter}%
    %
    \newtheorem{%
      \GetTheoremLemmaFormatEnvironmentName}{%
      \GetTheoremLemmaFormatShowText}[%
      \GetTheoremLemmaFormatFollowCounter]%
  }%
} % End of \newcommand{}

% ---------------------------------------------------------

\pgfkeys
{
  /TheoremCorollaryFormat/.is family, /TheoremCorollaryFormat,
  default/.style =
  {
    EnvironmentName = {EnvCorollary},
    ShowText = {Corollary},
    FollowCounter = Section,
  },
  EnvironmentName/.estore in = \GetTheoremCorollaryFormatEnvironmentName,
  ShowText/.estore in = \GetTheoremCorollaryFormatShowText,
  FollowCounter/.estore in = \GetTheoremCorollaryFormatFollowCounter,
} % End of \pgfkeys{}

\newcommand{\InsertCorollary}[2][\empty]
{%
  \InsertTheoremContent[#1]{\GetTheoremCorollaryFormatEnvironmentName}{#2}%
} % End of \newcommand{}

\newcommand{\InitTheoremCorollaryFormat}
{%
  \theoremstyle{plain}%
  \ifthenelse{\equal{\GetTheoremCorollaryFormatFollowCounter}{\empty}}%
  {%
    \newtheorem{%
      \GetTheoremCorollaryFormatEnvironmentName}{%
      \GetTheoremCorollaryFormatShowText}%
  }%
  {%
    %
    \MappingTheoremCounter[Corollary]{\GetTheoremCorollaryFormatFollowCounter}%
    %
    \newtheorem{%
      \GetTheoremCorollaryFormatEnvironmentName}{%
      \GetTheoremCorollaryFormatShowText}[%
      \GetTheoremCorollaryFormatFollowCounter]%
  }%
} % End of \newcommand{}

% ---------------------------------------------------------

\pgfkeys
{
  /TheoremPropositionFormat/.is family, /TheoremPropositionFormat,
  default/.style =
  {
    EnvironmentName = {EnvProposition},
    ShowText = {Proposition},
    FollowCounter = Section,
  },
  EnvironmentName/.estore in = \GetTheoremPropositionFormatEnvironmentName,
  ShowText/.estore in = \GetTheoremPropositionFormatShowText,
  FollowCounter/.estore in = \GetTheoremPropositionFormatFollowCounter,
} % End of \pgfkeys{}

\newcommand{\InsertProposition}[2][\empty]
{%
  \InsertTheoremContent[#1]{\GetTheoremPropositionFormatEnvironmentName}{#2}%
} % End of \newcommand{}

\newcommand{\InitTheoremPropositionFormat}
{%
  \theoremstyle{plain}%
  \ifthenelse{\equal{\GetTheoremPropositionFormatFollowCounter}{\empty}}%
  {%
    \newtheorem{%
      \GetTheoremPropositionFormatEnvironmentName}{%
      \GetTheoremPropositionFormatShowText}%
  }%
  {%
    %
    \MappingTheoremCounter[Proposition]{\GetTheoremPropositionFormatFollowCounter}%
    %
    \newtheorem{%
      \GetTheoremPropositionFormatEnvironmentName}{%
      \GetTheoremPropositionFormatShowText}[%
      \GetTheoremPropositionFormatFollowCounter]%
  }%
} % End of \newcommand{}

% ---------------------------------------------------------

\pgfkeys
{
  /TheoremConjectureFormat/.is family, /TheoremConjectureFormat,
  default/.style =
  {
    EnvironmentName = {EnvConjecture},
    ShowText = {Conjecture},
    FollowCounter = Section,
  },
  EnvironmentName/.estore in = \GetTheoremConjectureFormatEnvironmentName,
  ShowText/.estore in = \GetTheoremConjectureFormatShowText,
  FollowCounter/.estore in = \GetTheoremConjectureFormatFollowCounter,
} % End of \pgfkeys{}

\newcommand{\InsertConjecture}[2][\empty]
{%
  \InsertTheoremContent[#1]{\GetTheoremConjectureFormatEnvironmentName}{#2}%
} % End of \newcommand{}

\newcommand{\InitTheoremConjectureFormat}
{%
  \theoremstyle{plain}%
  \ifthenelse{\equal{\GetTheoremConjectureFormatFollowCounter}{\empty}}%
  {%
    \newtheorem{%
      \GetTheoremConjectureFormatEnvironmentName}{%
      \GetTheoremConjectureFormatShowText}%
  }%
  {%
    %
    \MappingTheoremCounter[Conjecture]{\GetTheoremConjectureFormatFollowCounter}%
    %
    \newtheorem{%
      \GetTheoremConjectureFormatEnvironmentName}{%
      \GetTheoremConjectureFormatShowText}[%
      \GetTheoremConjectureFormatFollowCounter]%
  }%
} % End of \newcommand{}

% ---------------------------------------------------------

\pgfkeys
{
  /TheoremHypothesisFormat/.is family, /TheoremHypothesisFormat,
  default/.style =
  {
    EnvironmentName = {EnvHypothesis},
    ShowText = {Hypothesis},
    FollowCounter = Section,
  },
  EnvironmentName/.estore in = \GetTheoremHypothesisFormatEnvironmentName,
  ShowText/.estore in = \GetTheoremHypothesisFormatShowText,
  FollowCounter/.estore in = \GetTheoremHypothesisFormatFollowCounter,
} % End of \pgfkeys{}

\newcommand{\InsertHypothesis}[2][\empty]
{%
  \InsertTheoremContent[#1]{\GetTheoremHypothesisFormatEnvironmentName}{#2}%
} % End of \newcommand{}

\newcommand{\InitTheoremHypothesisFormat}
{%
  \theoremstyle{definition}%
  \ifthenelse{\equal{\GetTheoremHypothesisFormatFollowCounter}{\empty}}%
  {%
    \newtheorem{%
      \GetTheoremHypothesisFormatEnvironmentName}{%
      \GetTheoremHypothesisFormatShowText}%
  }%
  {%
    %
    \MappingTheoremCounter[Hypothesis]{\GetTheoremHypothesisFormatFollowCounter}%
    %
    \newtheorem{%
      \GetTheoremHypothesisFormatEnvironmentName}{%
      \GetTheoremHypothesisFormatShowText}[%
      \GetTheoremHypothesisFormatFollowCounter]%
  }%
} % End of \newcommand{}

% ---------------------------------------------------------

\pgfkeys
{
  /TheoremQuestionFormat/.is family, /TheoremQuestionFormat,
  default/.style =
  {
    EnvironmentName = {EnvQuestion},
    ShowText = {Question},
    FollowCounter = Section,
  },
  EnvironmentName/.estore in = \GetTheoremQuestionFormatEnvironmentName,
  ShowText/.estore in = \GetTheoremQuestionFormatShowText,
  FollowCounter/.estore in = \GetTheoremQuestionFormatFollowCounter,
} % End of \pgfkeys{}

\newcommand{\InsertQuestion}[2][\empty]
{%
  \InsertTheoremContent[#1]{\GetTheoremQuestionFormatEnvironmentName}{#2}%
} % End of \newcommand{}

\newcommand{\InitTheoremQuestionFormat}
{%
  \theoremstyle{definition}%
  \ifthenelse{\equal{\GetTheoremQuestionFormatFollowCounter}{\empty}}%
  {%
    \newtheorem{%
      \GetTheoremQuestionFormatEnvironmentName}{%
      \GetTheoremQuestionFormatShowText}%
  }%
  {%
    %
    \MappingTheoremCounter[Question]{\GetTheoremQuestionFormatFollowCounter}%
    %
    \newtheorem{%
      \GetTheoremQuestionFormatEnvironmentName}{%
      \GetTheoremQuestionFormatShowText}[%
      \GetTheoremQuestionFormatFollowCounter]%
  }%
} % End of \newcommand{}

% ---------------------------------------------------------

\pgfkeys
{
  /TheoremAssertionFormat/.is family, /TheoremAssertionFormat,
  default/.style =
  {
    EnvironmentName = {EnvAssertion},
    ShowText = {Assertion},
    FollowCounter = Section,
  },
  EnvironmentName/.estore in = \GetTheoremAssertionFormatEnvironmentName,
  ShowText/.estore in = \GetTheoremAssertionFormatShowText,
  FollowCounter/.estore in = \GetTheoremAssertionFormatFollowCounter,
} % End of \pgfkeys{}

\newcommand{\InsertAssertion}[2][\empty]
{%
  \InsertTheoremContent[#1]{\GetTheoremAssertionFormatEnvironmentName}{#2}%
} % End of \newcommand{}

\newcommand{\InitTheoremAssertionFormat}
{%
  \theoremstyle{plain}%
  \ifthenelse{\equal{\GetTheoremAssertionFormatFollowCounter}{\empty}}%
  {%
    \newtheorem{%
      \GetTheoremAssertionFormatEnvironmentName}{%
      \GetTheoremAssertionFormatShowText}%
  }%
  {%
    %
    \MappingTheoremCounter[Assertion]{\GetTheoremAssertionFormatFollowCounter}%
    %
    \newtheorem{%
      \GetTheoremAssertionFormatEnvironmentName}{%
      \GetTheoremAssertionFormatShowText}[%
      \GetTheoremAssertionFormatFollowCounter]%
  }%
} % End of \newcommand{}

% ---------------------------------------------------------

\pgfkeys
{
  /TheoremCriterionFormat/.is family, /TheoremCriterionFormat,
  default/.style =
  {
    EnvironmentName = {EnvCriterion},
    ShowText = {Criterion},
    FollowCounter = Section,
  },
  EnvironmentName/.estore in = \GetTheoremCriterionFormatEnvironmentName,
  ShowText/.estore in = \GetTheoremCriterionFormatShowText,
  FollowCounter/.estore in = \GetTheoremCriterionFormatFollowCounter,
} % End of \pgfkeys{}

\newcommand{\InsertCriterion}[2][\empty]
{%
  \InsertTheoremContent[#1]{\GetTheoremCriterionFormatEnvironmentName}{#2}%
} % End of \newcommand{}

\newcommand{\InitTheoremCriterionFormat}
{%
  \theoremstyle{plain}%
  \ifthenelse{\equal{\GetTheoremCriterionFormatFollowCounter}{\empty}}%
  {%
    \newtheorem{%
      \GetTheoremCriterionFormatEnvironmentName}{%
      \GetTheoremCriterionFormatShowText}%
  }%
  {%
    %
    \MappingTheoremCounter[Criterion]{\GetTheoremCriterionFormatFollowCounter}%
    %
    \newtheorem{%
      \GetTheoremCriterionFormatEnvironmentName}{%
      \GetTheoremCriterionFormatShowText}[%
      \GetTheoremCriterionFormatFollowCounter]%
  }%
} % End of \newcommand{}

% ---------------------------------------------------------

\pgfkeys
{
  /TheoremProofFormat/.is family, /TheoremProofFormat,
  default/.style =
  {
    EnvironmentName = {EnvProof},
    ShowText = {Proof},
    FollowCounter = \empty,
  },
  EnvironmentName/.estore in = \GetTheoremProofFormatEnvironmentName,
  ShowText/.estore in = \GetTheoremProofFormatShowText,
  FollowCounter/.estore in = \GetTheoremProofFormatFollowCounter,
} % End of \pgfkeys{}

\newcommand{\InsertProof}[1]
{%
  \InsertTheoremContent[\empty]{%
    \GetTheoremProofFormatEnvironmentName}{#1}%
} % End of \newcommand{}

\newcommand{\InitTheoremProofFormat}
{%
  \theoremstyle{definition}%
  \ifthenelse{\equal{\GetTheoremProofFormatFollowCounter}{\empty}}%
  {%
    \newtheorem*{%
      \GetTheoremProofFormatEnvironmentName}{%
      \GetTheoremProofFormatShowText}
  }%
  {%
    %
    \MappingTheoremCounter[Proof]{\GetTheoremProofFormatFollowCounter}%
    %
    \newtheorem{%
      \GetTheoremProofFormatEnvironmentName}{%
      \GetTheoremProofFormatShowText}[%
      \GetTheoremProofFormatFollowCounter]%
  }%
} % End of \newcommand{}

% ---------------------------------------------------------

\pgfkeys
{
  /TheoremNoteFormat/.is family, /TheoremNoteFormat,
  default/.style =
  {
    EnvironmentName = {EnvNote},
    ShowText = {Note},
    FollowCounter = \empty,
  },
  EnvironmentName/.estore in = \GetTheoremNoteFormatEnvironmentName,
  ShowText/.estore in = \GetTheoremNoteFormatShowText,
  FollowCounter/.estore in = \GetTheoremNoteFormatFollowCounter,
} % End of \pgfkeys{}

\newcommand{\InsertNote}[1]
{%
  \InsertTheoremContent[\empty]{%
    \GetTheoremNoteFormatEnvironmentName}{#1}%
} % End of \newcommand{}

\newcommand{\InitTheoremNoteFormat}
{%
  \theoremstyle{definition}%
  \ifthenelse{\equal{\GetTheoremNoteFormatFollowCounter}{\empty}}%
  {%
    \newtheorem*{%
      \GetTheoremNoteFormatEnvironmentName}{%
      \GetTheoremNoteFormatShowText}
  }%
  {%
    %
    \MappingTheoremCounter[Note]{\GetTheoremNoteFormatFollowCounter}%
    %
    \newtheorem{%
      \GetTheoremNoteFormatEnvironmentName}{%
      \GetTheoremNoteFormatShowText}[%
      \GetTheoremNoteFormatFollowCounter]%
  }%
} % End of \newcommand{}

% ---------------------------------------------------------

\pgfkeys
{
  /TheoremAnnotationFormat/.is family, /TheoremAnnotationFormat,
  default/.style =
  {
    EnvironmentName = {EnvAnnotation},
    ShowText = {Annotation},
    FollowCounter = \empty,
  },
  EnvironmentName/.estore in = \GetTheoremAnnotationFormatEnvironmentName,
  ShowText/.estore in = \GetTheoremAnnotationFormatShowText,
  FollowCounter/.estore in = \GetTheoremAnnotationFormatFollowCounter,
} % End of \pgfkeys{}

\newcommand{\InsertAnnotation}[1]
{%
  \InsertTheoremContent[\empty]{%
    \GetTheoremAnnotationFormatEnvironmentName}{#1}%
} % End of \newcommand{}

\newcommand{\InitTheoremAnnotationFormat}
{%
  \theoremstyle{definition}%
  \ifthenelse{\equal{\GetTheoremAnnotationFormatFollowCounter}{\empty}}%
  {%
    \newtheorem*{%
      \GetTheoremAnnotationFormatEnvironmentName}{%
      \GetTheoremAnnotationFormatShowText}
  }%
  {%
    %
    \MappingTheoremCounter[Annotation]{\GetTheoremAnnotationFormatFollowCounter}%
    %
    \newtheorem{%
      \GetTheoremAnnotationFormatEnvironmentName}{%
      \GetTheoremAnnotationFormatShowText}[%
      \GetTheoremAnnotationFormatFollowCounter]%
  }%
} % End of \newcommand{}

% ---------------------------------------------------------

\pgfkeys
{
  /TheoremClaimFormat/.is family, /TheoremClaimFormat,
  default/.style =
  {
    EnvironmentName = {EnvClaim},
    ShowText = {Claim},
    FollowCounter = \empty,
  },
  EnvironmentName/.estore in = \GetTheoremClaimFormatEnvironmentName,
  ShowText/.estore in = \GetTheoremClaimFormatShowText,
  FollowCounter/.estore in = \GetTheoremClaimFormatFollowCounter,
} % End of \pgfkeys{}

\newcommand{\InsertClaim}[1]
{%
  \InsertTheoremContent[\empty]{%
    \GetTheoremClaimFormatEnvironmentName}{#1}%
} % End of \newcommand{}

\newcommand{\InitTheoremClaimFormat}
{%
  \theoremstyle{definition}%
  \ifthenelse{\equal{\GetTheoremClaimFormatFollowCounter}{\empty}}%
  {%
    \newtheorem*{%
      \GetTheoremClaimFormatEnvironmentName}{%
      \GetTheoremClaimFormatShowText}
  }%
  {%
    %
    \MappingTheoremCounter[Claim]{\GetTheoremClaimFormatFollowCounter}%
    %
    \newtheorem{%
      \GetTheoremClaimFormatEnvironmentName}{%
      \GetTheoremClaimFormatShowText}[%
      \GetTheoremClaimFormatFollowCounter]%
  }%
} % End of \newcommand{}

% ---------------------------------------------------------

\pgfkeys
{
  /TheoremCaseFormat/.is family, /TheoremCaseFormat,
  default/.style =
  {
    EnvironmentName = {EnvCase},
    ShowText = {Case},
    FollowCounter = \empty,
  },
  EnvironmentName/.estore in = \GetTheoremCaseFormatEnvironmentName,
  ShowText/.estore in = \GetTheoremCaseFormatShowText,
  FollowCounter/.estore in = \GetTheoremCaseFormatFollowCounter,
} % End of \pgfkeys{}

\newcommand{\InsertCase}[1]
{%
  \InsertTheoremContent[\empty]{%
    \GetTheoremCaseFormatEnvironmentName}{#1}%
} % End of \newcommand{}

\newcommand{\InitTheoremCaseFormat}
{%
  \theoremstyle{definition}%
  \ifthenelse{\equal{\GetTheoremCaseFormatFollowCounter}{\empty}}%
  {%
    \newtheorem*{%
      \GetTheoremCaseFormatEnvironmentName}{%
      \GetTheoremCaseFormatShowText}
  }%
  {%
    %
    \MappingTheoremCounter[Case]{\GetTheoremCaseFormatFollowCounter}%
    %
    \newtheorem{%
      \GetTheoremCaseFormatEnvironmentName}{%
      \GetTheoremCaseFormatShowText}[%
      \GetTheoremCaseFormatFollowCounter]%
  }%
} % End of \newcommand{}

% ---------------------------------------------------------

\pgfkeys
{
  /TheoremAcknowledgmentFormat/.is family, /TheoremAcknowledgmentFormat,
  default/.style =
  {
    EnvironmentName = {EnvAcknowledgment},
    ShowText = {Acknowledgment},
    FollowCounter = \empty,
  },
  EnvironmentName/.estore in = \GetTheoremAcknowledgmentFormatEnvironmentName,
  ShowText/.estore in = \GetTheoremAcknowledgmentFormatShowText,
  FollowCounter/.estore in = \GetTheoremAcknowledgmentFormatFollowCounter,
} % End of \pgfkeys{}

\newcommand{\InsertAcknowledgment}[1]
{%
  \InsertTheoremContent[\empty]{%
    \GetTheoremAcknowledgmentFormatEnvironmentName}{#1}%
} % End of \newcommand{}

\newcommand{\InitTheoremAcknowledgmentFormat}
{%
  \theoremstyle{definition}%
  \ifthenelse{\equal{\GetTheoremAcknowledgmentFormatFollowCounter}{\empty}}%
  {%
    \newtheorem*{%
      \GetTheoremAcknowledgmentFormatEnvironmentName}{%
      \GetTheoremAcknowledgmentFormatShowText}
  }%
  {%
    %
    \MappingTheoremCounter[Acknowledgment]{\GetTheoremAcknowledgmentFormatFollowCounter}%
    %
    \newtheorem{%
      \GetTheoremAcknowledgmentFormatEnvironmentName}{%
      \GetTheoremAcknowledgmentFormatShowText}[%
      \GetTheoremAcknowledgmentFormatFollowCounter]%
  }%
} % End of \newcommand{}

% ---------------------------------------------------------

\pgfkeys
{
  /TheoremConclusionFormat/.is family, /TheoremConclusionFormat,
  default/.style =
  {
    EnvironmentName = {EnvConclusion},
    ShowText = {Conclusion},
    FollowCounter = \empty,
  },
  EnvironmentName/.estore in = \GetTheoremConclusionFormatEnvironmentName,
  ShowText/.estore in = \GetTheoremConclusionFormatShowText,
  FollowCounter/.estore in = \GetTheoremConclusionFormatFollowCounter,
} % End of \pgfkeys{}

\newcommand{\InsertConclusion}[1]
{%
  \InsertTheoremContent[\empty]{%
    \GetTheoremConclusionFormatEnvironmentName}{#1}%
} % End of \newcommand{}

\newcommand{\InitTheoremConclusionFormat}
{%
  \theoremstyle{definition}%
  \ifthenelse{\equal{\GetTheoremConclusionFormatFollowCounter}{\empty}}%
  {%
    \newtheorem*{%
      \GetTheoremConclusionFormatEnvironmentName}{%
      \GetTheoremConclusionFormatShowText}
  }%
  {%
    %
    \MappingTheoremCounter[Conclusion]{\GetTheoremConclusionFormatFollowCounter}%
    %
    \newtheorem{%
      \GetTheoremConclusionFormatEnvironmentName}{%
      \GetTheoremConclusionFormatShowText}[%
      \GetTheoremConclusionFormatFollowCounter]%
  }%
} % End of \newcommand{}

% ---------------------------------------------------------

\pgfkeys
{
  /TheoremSummaryFormat/.is family, /TheoremSummaryFormat,
  default/.style =
  {
    EnvironmentName = {EnvSummary},
    ShowText = {Summary},
    FollowCounter = \empty,
  },
  EnvironmentName/.estore in = \GetTheoremSummaryFormatEnvironmentName,
  ShowText/.estore in = \GetTheoremSummaryFormatShowText,
  FollowCounter/.estore in = \GetTheoremSummaryFormatFollowCounter,
} % End of \pgfkeys{}

\newcommand{\InsertSummary}[1]
{%
  \InsertTheoremContent[\empty]{%
    \GetTheoremSummaryFormatEnvironmentName}{#1}%
} % End of \newcommand{}

\newcommand{\InitTheoremSummaryFormat}
{%
  \theoremstyle{definition}%
  \ifthenelse{\equal{\GetTheoremSummaryFormatFollowCounter}{\empty}}%
  {%
    \newtheorem*{%
      \GetTheoremSummaryFormatEnvironmentName}{%
      \GetTheoremSummaryFormatShowText}
  }%
  {%
    %
    \MappingTheoremCounter[Summary]{\GetTheoremSummaryFormatFollowCounter}%
    %
    \newtheorem{%
      \GetTheoremSummaryFormatEnvironmentName}{%
      \GetTheoremSummaryFormatShowText}[%
      \GetTheoremSummaryFormatFollowCounter]%
  }%
} % End of \newcommand{}

% ---------------------------------------------------------

\newcommand{\InitTheoremFormats}
{%
  \InitTheoremDefinitionFormat%
  \InitTheoremConditionFormat%
  \InitTheoremProblemFormat%
  \InitTheoremExampleFormat%
  \InitTheoremTheoremFormat%
  \InitTheoremLemmaFormat%
  \InitTheoremCorollaryFormat%
  \InitTheoremPropositionFormat%
  \InitTheoremConjectureFormat%
  \InitTheoremProofFormat%
  \InitTheoremNoteFormat%
  \InitTheoremAnnotationFormat%
  \InitTheoremClaimFormat%
  \InitTheoremCaseFormat%
  \InitTheoremAcknowledgmentFormat%
  \InitTheoremConclusionFormat%
  \InitTheoremCriterionFormat%
  \InitTheoremAssertionFormat%
  \InitTheoremQuestionFormat%
  \InitTheoremHypothesisFormat%
  \InitTheoremSummaryFormat%
} % End of \newcommand{}

\newcommand{\SetTheoremFormat}[2][\empty]
{%
  \ifthenelse{\equal{#1}{Definition}}%
  {%
    \pgfkeys{/TheoremDefinitionFormat, default, #2}%
  }{}%
  %
  \ifthenelse{\equal{#1}{Condition}}%
  {%
    \pgfkeys{/TheoremConditionFormat, default, #2}%
  }{}%
  %
  \ifthenelse{\equal{#1}{Problem}}%
  {%
    \pgfkeys{/TheoremProblemFormat, default, #2}%
  }{}%
  %
  \ifthenelse{\equal{#1}{Example}}%
  {%
    \pgfkeys{/TheoremExampleFormat, default, #2}%
  }{}%
  %
  \ifthenelse{\equal{#1}{Theorem}}%
  {%
    \pgfkeys{/TheoremTheoremFormat, default, #2}%
  }{}%
  %
  \ifthenelse{\equal{#1}{Lemma}}%
  {%
    \pgfkeys{/TheoremLemmaFormat, default, #2}%
  }{}%
  %
  \ifthenelse{\equal{#1}{Corollary}}%
  {%
    \pgfkeys{/TheoremCorollaryFormat, default, #2}%
  }{}%
  %
  \ifthenelse{\equal{#1}{Proposition}}%
  {%
    \pgfkeys{/TheoremPropositionFormat, default, #2}%
  }{}%
  %
  \ifthenelse{\equal{#1}{Conjecture}}%
  {%
    \pgfkeys{/TheoremConjectureFormat, default, #2}%
  }{}%
  %
  \ifthenelse{\equal{#1}{Proof}}%
  {%
    \pgfkeys{/TheoremProofFormat, default, #2}%
  }{}%
  %
  \ifthenelse{\equal{#1}{Note}}%
  {%
    \pgfkeys{/TheoremNoteFormat, default, #2}%
  }{}%
  %
  \ifthenelse{\equal{#1}{Annotation}}%
  {%
    \pgfkeys{/TheoremAnnotationFormat, default, #2}%
  }{}%
  %
  \ifthenelse{\equal{#1}{Claim}}%
  {%
    \pgfkeys{/TheoremClaimFormat, default, #2}%
  }{}%
  %
  \ifthenelse{\equal{#1}{Case}}%
  {%
    \pgfkeys{/TheoremCaseFormat, default, #2}%
  }{}%
  %
  \ifthenelse{\equal{#1}{Acknowledgment}}%
  {%
    \pgfkeys{/TheoremAcknowledgmentFormat, default, #2}%
  }{}%
  %
  \ifthenelse{\equal{#1}{Conclusion}}%
  {%
    \pgfkeys{/TheoremConclusionFormat, default, #2}%
  }{}%
  %
  \ifthenelse{\equal{#1}{Criterion}}%
  {%
    \pgfkeys{/TheoremCriterionFormat, default, #2}%
  }{}%
  %
  \ifthenelse{\equal{#1}{Assertion}}%
  {%
    \pgfkeys{/TheoremAssertionFormat, default, #2}%
  }{}%
  %
  \ifthenelse{\equal{#1}{Question}}%
  {%
    \pgfkeys{/TheoremQuestionFormat, default, #2}%
  }{}%
  %
  \ifthenelse{\equal{#1}{Hypothesis}}%
  {%
    \pgfkeys{/TheoremHypothesisFormat, default, #2}%
  }{}%
  %
  \ifthenelse{\equal{#1}{Summary}}%
  {%
    \pgfkeys{/TheoremSummaryFormat, default, #2}%
  }{}%
  %
} % End of \newcommand{}

% ---------------------------------------------------------

\SetTheoremFormat[Definition]{ShowText = {Definition}}%
\SetTheoremFormat[Condition]{ShowText = {Condition}}%
\SetTheoremFormat[Problem]{ShowText = {Problem}}%
\SetTheoremFormat[Example]{ShowText = {Example}}%
\SetTheoremFormat[Theorem]{ShowText = {Theorem}}%
\SetTheoremFormat[Lemma]{ShowText = {Lemma}}%
\SetTheoremFormat[Corollary]{ShowText = {Corollary}}%
\SetTheoremFormat[Proposition]{ShowText = {Proposition}}%
\SetTheoremFormat[Conjecture]{ShowText = {Conjecture}}%
\SetTheoremFormat[Proof]{ShowText = {Proof}}%
\SetTheoremFormat[Note]{ShowText = {Note}}%
\SetTheoremFormat[Annotation]{ShowText = {Annotation}}%
\SetTheoremFormat[Claim]{ShowText = {Claim}}%
\SetTheoremFormat[Case]{ShowText = {Case}}%
\SetTheoremFormat[Acknowledgment]{ShowText = {Acknowledgment}}%
\SetTheoremFormat[Conclusion]{ShowText = {Conclusion}}%
\SetTheoremFormat[Criterion]{ShowText = {Criterion}}%
\SetTheoremFormat[Assertion]{ShowText = {Assertion}}%
\SetTheoremFormat[Question]{ShowText = {Question}}%
\SetTheoremFormat[Hypothesis]{ShowText = {Hypothesis}}%
\SetTheoremFormat[Summary]{ShowText = {Summary}}%

% ----------------------------------------------------------------------------


% Some function that use for school
%
% This file is part of the project of
% National Cheng Kung University (NCKU) Thesis/Dissertation Template in LaTex.
% This project is hold at
%     <https://github.com/wengan-li/ncku-thesis-template-latex>
% by Wen-Gan Li.
%
% This project is distributed in the hope of usefuling to someone,
% you can redistribute it and/or modify it under the terms of the
% Attribution-NonCommercial-ShareAlike 4.0 International.
%
% You should have received a copy of the
% Attribution-NonCommercial-ShareAlike 4.0 International
% along with this project.
% If not, see <http://creativecommons.org/licenses/by-nc-sa/4.0/legalcode.txt>.
%
% Please feel free to fork it, modify it, and try it.
% Have fun !!!
%

% ----------------------------------------------------------------------------
% The list of all college in NCKU

% Use the list from:
%   http://web.ncku.edu.tw/files/11-1000-182.php

% 所有學院跟系所的設定
% 縮寫是靠學校給的Domain name所得出的, 故可能會有錯誤的時候
% ----------------------------------------------------------------------------

% ----------------------------------------------------------------------------
% 學院 College
% ----------------------------------------------------------------------------

% 文學院 College of Liberal Arts
\newcommand{\SetCollegeLiberalArts}
{
  \SetCollName{文學院}{College of Liberal Arts}
} % End of \newcommand{}

% 理學院 College of Sciences
\newcommand{\SetCollegeSciences}
{
  \SetCollName{理學院}{College of Sciences}
} % End of \newcommand{}

% 工學院 College of Engineering
\newcommand{\SetCollegeEngineering}
{
  \SetCollName{工學院}{College of Engineering}
} % End of \newcommand{}

% 電機資訊學院 College of Electrical Engineering & Computer Science
\newcommand{\SetCollegeElectricalEngineeringAndComputerScience}
{
  \SetCollName{電機資訊學院}{College of Electrical Engineering and Computer Science}
} % End of \newcommand{}

% 管理學院 College of Management
\newcommand{\SetCollegeManagement}
{
  \SetCollName{管理學院}{College of Management}
} % End of \newcommand{}

% 社會科學院 College of Social Science
\newcommand{\SetCollegeSocialScience}
{
  \SetCollName{社會科學院}{College of Social Science}
} % End of \newcommand{}

% 規劃與設計學院 College of Planning & Design
\newcommand{\SetCollegePlanningAndDesign}
{
  \SetCollName{規劃與設計學院}{College of Planning and Design}
} % End of \newcommand{}

% 生物科學與科技學院 College of Bioscience & Biotechnology
\newcommand{\SetCollegeBioscienceAndBiotechnology}
{
  \SetCollName{生物科學與科技學院}{College of Bioscience and Biotechnology}
} % End of \newcommand{}

% 醫學院 College of Medicine
\newcommand{\SetCollegeMedicine}
{
  \SetCollName{醫學院}{College of Medicine}
} % End of \newcommand{}

% ----------------------------------------------------------------------------

%
% This file is part of the project of
% National Cheng Kung University (NCKU) Thesis/Dissertation Template in LaTex.
% This project is hold at
%     <https://github.com/wengan-li/ncku-thesis-template-latex>
% by Wen-Gan Li.
%
% This project is distributed in the hope of usefuling to someone,
% you can redistribute it and/or modify it under the terms of the
% Attribution-NonCommercial-ShareAlike 4.0 International.
%
% You should have received a copy of the
% Attribution-NonCommercial-ShareAlike 4.0 International
% along with this project.
% If not, see <http://creativecommons.org/licenses/by-nc-sa/4.0/legalcode.txt>.
%
% Please feel free to fork it, modify it, and try it.
% Have fun !!!
%

% ----------------------------------------------------------------------------
% The list of all department and institute in NCKU

% Use the list from:
%   http://web.ncku.edu.tw/files/11-1000-182.php

% 所有學院跟系所的設定.
%
% 縮寫是靠學校給的Domain name所得出的, 故可能會有錯誤的時候.
% 所以如果錯了的話, 就請告知真正的寫法(或縮寫)是什麼.
%
% ----------------------------------------------------------------------------
% ----------------------------------------------------------------------------
%                       系所 Department and Institute
% ----------------------------------------------------------------------------
% ----------------------------------------------------------------------------
% --------------------- 文學院 College of Liberal Arts ---------------------
% ----------------------------------------------------------------------------

% 中國文學系 Department of Chinese Literature
\newcommand{\SetDeptChinese}
{
  \SetDeptName{中國文學系}{Chinese}{Department of Chinese Literature}
  \SetCollegeLiberalArts
} % End of \newcommand{}

% 藝術研究所 Institute of Art
\newcommand{\SetDeptArt}
{
  \SetDeptName{藝術研究所}{Art}{Institute of Art}
  \SetCollegeLiberalArts
} % End of \newcommand{}

% 閩南文化研究中心 Min-Nan Culture Studies Center
\newcommand{\SetDeptMinNan}
{
  \SetDeptName{閩南文化研究中心}{MinNan}{Min-Nan Culture Studies Center}
  \SetCollegeLiberalArts
} % End of \newcommand{}

% 外國語文學系 Department of Foreign Languages and Literature
\newcommand{\SetDeptFLLD}
{
  \SetDeptName{外國語文學系}{FLLD}{Department of Foreign Languages and Literature}
  \SetCollegeLiberalArts
} % End of \newcommand{}

% 臺灣文學系 Department of Taiwanese Literature
\newcommand{\SetDeptTWL}
{
  \SetDeptName{臺灣文學系}{TWL}{Department of Taiwanese Literature}
  \SetCollegeLiberalArts
} % End of \newcommand{}

% 華語中心 Chinese Language Center
\newcommand{\SetDeptKCLC}
{
  \SetDeptName{華語中心}{KCLC}{Chinese Language Center}
  \SetCollegeLiberalArts
} % End of \newcommand{}

% 外語中心 Foreign Language Center
\newcommand{\SetDeptLang}
{
  \SetDeptName{外語中心}{Lang}{Foreign Language Center}
  \SetCollegeLiberalArts
} % End of \newcommand{}

% 歷史學系 Department of History
\newcommand{\SetDeptHis}
{
  \SetDeptName{歷史學系}{His}{Department of History}
  \SetCollegeLiberalArts
} % End of \newcommand{}

% ----------------------------------------------------------------------------
% --------------------- 理學院 College of Sciences ---------------------
% ----------------------------------------------------------------------------

% 數學系 Department of Mathematics
\newcommand{\SetDeptMath}
{
  \SetDeptName{數學系}{Math}{Department of Mathematics}
  \SetCollegeSciences
} % End of \newcommand{}

% 光電科學與工程學系 Departmment of Photonics
\newcommand{\SetDeptDPS}
{
  \SetDeptName{光電科學與工程學系}{DPS}{Departmment of Photonics}
  \SetCollegeSciences
} % End of \newcommand{}

% 物理學系 Department of Physics
\newcommand{\SetDeptPhys}
{
  \SetDeptName{物理學系}{Phys}{Department of Physics}
  \SetCollegeSciences
} % End of \newcommand{}

% 化學系 Department of Chemistry
\newcommand{\SetDeptCh}
{
  \SetDeptName{化學系}{Ch}{Department of Chemistry}
  \SetCollegeSciences
} % End of \newcommand{}

% 地球科學系 Department of Earth Sciences
\newcommand{\SetDeptEarth}
{
  \SetDeptName{地球科學系}{Earth}{Department of Earth Sciences}
  \SetCollegeSciences
} % End of \newcommand{}

% 太空與電漿科學研究所 Institute of Space and Plasma Sciences
\newcommand{\SetDeptPSSC}
{
  \SetDeptName{太空與電漿科學研究所}{PSSC}{Institute of Space and Plasma Sciences}
  \SetCollegeSciences
} % End of \newcommand{}

% 國家理論科學研究中心 National Center for Theoretical Sciences (South)
\newcommand{\SetDeptNCTS}
{
  \SetDeptName{國家理論科學研究中心}{NCTS}{National Center for Theoretical Sciences (South)}
  \SetCollegeSciences
} % End of \newcommand{}

% ----------------------------------------------------------------------------
% --------------------- 工學院 College of Engineering ---------------------
% ----------------------------------------------------------------------------

% 機械工程學系 Department of Mechanical Engineering
\newcommand{\SetDeptME}
{
  \SetDeptName{機械工程學系}{ME}{Department of Mechanical Engineering}
  \SetCollegeEngineering
} % End of \newcommand{}

% 化學工程學系 Department of Chemical Engineering
\newcommand{\SetDeptChe}
{
  \SetDeptName{化學工程學系}{Che}{Department of Chemical Engineering}
  \SetCollegeEngineering
} % End of \newcommand{}

% 土木工程學系 Department of Civil Engineering
\newcommand{\SetDeptCivil}
{
  \SetDeptName{土木工程學系}{Civil}{Department of Civil Engineering}
  \SetCollegeEngineering
} % End of \newcommand{}

% 材料科學及工程學系 Department of Materials Science and Engineering
\newcommand{\SetDeptMSE}
{
  \SetDeptName{材料科學及工程學系}{MSE}{Department of Materials Science and Engineering}
  \SetCollegeEngineering
} % End of \newcommand{}

% 水利及海洋工程學系 Department of Hydraulic and Ocean Engineering
\newcommand{\SetDeptHyd}
{
  \SetDeptName{水利及海洋工程學系}{Hyd}{Department of Hydraulic and Ocean Engineering}
  \SetCollegeEngineering
} % End of \newcommand{}

% 工程科學系 Department of Engineering Science
\newcommand{\SetDeptES}
{
  \SetDeptName{工程科學系}{ES}{Department of Engineering Science}
  \SetCollegeEngineering
} % End of \newcommand{}

% 系統及船舶機電工程學系 Department of System and Naval Mechatronic Engineering
\newcommand{\SetDeptSNAME}
{
  \SetDeptName{系統及船舶機電工程學系}{SNAME}{Department of System and Naval Mechatronic Engineering}
  \SetCollegeEngineering
} % End of \newcommand{}

% 航空太空工程學系 Department of Aeronautics and Astronautics
\newcommand{\SetDeptIAA}
{
  \SetDeptName{航空太空工程學系}{IAA}{Department of Aeronautics and Astronautics}
  \SetCollegeEngineering
} % End of \newcommand{}

% 資源工程學系 Department of Resources Engineering
\newcommand{\SetDeptMP}
{
  \SetDeptName{資源工程學系}{MP}{Department of Resources Engineering}
  \SetCollegeEngineering
} % End of \newcommand{}

% 環境工程學系 Department of Environmental Engineering
\newcommand{\SetDeptEV}
{
  \SetDeptName{環境工程學系}{EV}{Department of Environmental Engineering}
  \SetCollegeEngineering
} % End of \newcommand{}

% 生物醫學工程學系 Department of BioMedical Engineering
\newcommand{\SetDeptBME}
{
  \SetDeptName{生物醫學工程學系}{BME}{Department of BioMedical Engineering}
  \SetCollegeEngineering
} % End of \newcommand{}

% 測量及空間資訊學系 Department of Geomatics
\newcommand{\SetDeptGeomatics}
{
  \SetDeptName{測量及空間資訊學系}{Geomatics}{Department of Geomatics}
  \SetCollegeEngineering
} % End of \newcommand{}

% 海洋科技與事務研究所 Institute of Ocean Technology and Marine Affairs
\newcommand{\SetDeptIOTMA}
{
  \SetDeptName{海洋科技與事務研究所}{IOTMA}{Institute of Ocean Technology and Marine Affairs}
  \SetCollegeEngineering
} % End of \newcommand{}

% 民航研究所 Institute of Civil Aviation
\newcommand{\SetDeptICA}
{
  \SetDeptName{民航研究所}{ICA}{Institute of Civil Aviation}
  \SetCollegeEngineering
} % End of \newcommand{}

% 能源國際學士學位學程 International Bachelor Degree Program on Energy
\newcommand{\SetDeptIBDPE}
{
  \SetDeptName{能源國際學士學位學程}{IBDPE}{International Bachelor Degree Program on Energy}
  \SetCollegeEngineering
} % End of \newcommand{}

% 尖端材料國際碩士學位學程 International Curriculum for Advanced Materials Program
\newcommand{\SetDeptICAMP}
{
  \SetDeptName{尖端材料國際碩士學位學程}{ICAMP}{International Curriculum for Advanced Materials Program}
  \SetCollegeEngineering
} % End of \newcommand{}

% 自然災害減災及管理國際碩士學位學程 International Master Program on Natural Hazards Mitigation and Management
\newcommand{\SetDeptINHMM}
{
  \SetDeptName{自然災害減災及管理國際碩士學位學程}{INHMM}{International Master Program on \\ Natural Hazards Mitigation and Management}
  \SetCollegeEngineering
} % End of \newcommand{}

% 工程管理碩士在職專班 International Graduate Program of Civil Engineering and Management
\newcommand{\SetDeptICEM}
{
  \SetDeptName{工程管理碩士在職專班}{ICEM}{International Graduate Program of \\ Civil Engineering and Management}
  \SetCollegeEngineering
} % End of \newcommand{}

% ----------------------------------------------------------------------------
% ------ 電機資訊學院 College of Electrical Engineering & Computer Science ------
% ----------------------------------------------------------------------------

% 電機工程學系 Department of Electrical Engineering
\newcommand{\SetDeptEE}
{
  \SetDeptName{電機工程學系}{EE}{Department of Electrical Engineering}
  \SetCollegeElectricalEngineeringAndComputerScience
} % End of \newcommand{}

% 資訊工程研究所 Institute of Computer Science and Information Engineering
\newcommand{\SetDeptCSIE}
{
  \SetDeptName{資訊工程研究所}{CSIE}{Institute of Computer Science and Information Engineering}
  \SetCollegeElectricalEngineeringAndComputerScience
} % End of \newcommand{}

% 微電子工程研究所 Institute of Microelectronics
\newcommand{\SetDeptIME}
{
  \SetDeptName{微電子工程研究所}{IME}{Institute of Microelectronics}
  \SetCollegeElectricalEngineeringAndComputerScience
} % End of \newcommand{}

% 電腦與通信工程研究所 Institute of Computer & Communication Engineering
\newcommand{\SetDeptCCE}
{
  \SetDeptName{電腦與通信工程研究所}{CCE}{Institute of Computer & Communication Engineering}
  \SetCollegeElectricalEngineeringAndComputerScience
} % End of \newcommand{}

% 製造資訊與系統研究所 Institute of Manufacturing Information and Systems
\newcommand{\SetDeptIMIS}
{
  \SetDeptName{製造資訊與系統研究所}{IMIS}{Institute of Manufacturing Information and Systems}
  \SetCollegeElectricalEngineeringAndComputerScience
} % End of \newcommand{}

% 醫學資訊研究所 Institute of Medical Informatics
\newcommand{\SetDeptIMI}
{
  \SetDeptName{醫學資訊研究所}{IMI}{Institute of Medical Informatics}
  \SetCollegeElectricalEngineeringAndComputerScience
} % End of \newcommand{}

% ----------------------------------------------------------------------------
% --------------------- 管理學院 College of Management ---------------------
% ----------------------------------------------------------------------------

% 統計學系 Department of Statistics
\newcommand{\SetDeptSTAT}
{
  \SetDeptName{統計學系}{STAT}{Department of Statistics}
  \SetCollegeManagement
} % End of \newcommand{}

% 會計學系 Department of Accountancy
\newcommand{\SetDeptACC}
{
  \SetDeptName{會計學系}{ACC}{Department of Accountancy}
  \SetCollegeManagement
} % End of \newcommand{}

% 交通管理科學系 Department of Transportation and Communication Management Science
\newcommand{\SetDeptTCM}
{
  \SetDeptName{交通管理科學系}{TCM}{Department of Transportation and Communication Management Science}
  \SetCollegeManagement
} % End of \newcommand{}

% 企業管理學系暨國際企業研究所 Department of Business Administration and Graduate Institute of International Business
\newcommand{\SetDeptBA}
{
  \SetDeptName{企業管理學系暨國際企業研究所}{BA}{Department of Business Administration and \\ Graduate Institute of International Business}
  \SetCollegeManagement
} % End of \newcommand{}

% 電信管理研究所 Institute of Telecommunications Management
\newcommand{\SetDeptTM}
{
  \SetDeptName{電信管理研究所}{TM}{Institute of Telecommunications Management}
  \SetCollegeManagement
} % End of \newcommand{}

% 工業與資訊管理學系暨資訊管理研究所 Institute of Information Management
\newcommand{\SetDeptIIM}
{
  \SetDeptName{工業與資訊管理學系暨資訊管理研究所}{IIM}{Institute of Information Management}
  \SetCollegeManagement
} % End of \newcommand{}

% 財務金融研究所 Institute of Finance & Banking
\newcommand{\SetDeptFin}
{
  \SetDeptName{財務金融研究所}{Fin}{Institute of Finance & Banking}
  \SetCollegeManagement
} % End of \newcommand{}

% 體育健康與休閒研究所 Institute of Physical Education, Health & Leisure Studies
\newcommand{\SetDeptPHEI}
{
  \SetDeptName{體育健康與休閒研究所}{PHEI}{Institute of Physical Education, \\ Health & Leisure Studies}
  \SetCollegeManagement
} % End of \newcommand{}

% 高階管理碩士在職專班 (EMBA) Executive Master of Business Administration
\newcommand{\SetDeptEMBA}
{
  \SetDeptName{高階管理碩士在職專班}{EMBA}{Executive Master of Business Administration}
  \SetCollegeManagement
} % End of \newcommand{}

% 國際經營管理研究所 (IMBA) Institute of International Management (IMBA)
\newcommand{\SetDeptIMBA}
{
  \SetDeptName{國際經營管理研究所}{IMBA}{Institute of International Management}
  \SetCollegeManagement
} % End of \newcommand{}

% 經營管理碩士班 (AMBA) Advanced Master of Business Administration
\newcommand{\SetDeptAMBA}
{
  \SetDeptName{經營管理碩士班}{AMBA}{Advanced Master of Business Administration}
  \SetCollegeManagement
} % End of \newcommand{}

% ----------------------------------------------------------------------------
% ------------------- 社會科學院 College of Social Science -------------------
% ----------------------------------------------------------------------------

% 政治學系 Department of Political Science
\newcommand{\SetDeptPolSci}
{
  \SetDeptName{政治學系}{PolSci}{Department of Political Science}
  \SetCollegeSocialScience
} % End of \newcommand{}

% 經濟學系 Department of Economics
\newcommand{\SetDeptEconomic}
{
  \SetDeptName{經濟學系}{Economic}{Department of Economics}
  \SetCollegeSocialScience
} % End of \newcommand{}

% 心理學系 Department of Psychology
\newcommand{\SetDeptPsychology}
{
  \SetDeptName{心理學系}{Psychology}{Department of Psychology}
  \SetCollegeSocialScience
} % End of \newcommand{}

% 法律學系 Department of Law and Institute of Law in Science and Technology
\newcommand{\SetDeptLaw}
{
  \SetDeptName{法律學系}{Law}{Department of Law and \\Institute of Law in Science and Technology}
  \SetCollegeSocialScience
} % End of \newcommand{}

% 教育研究所 Institute of Education
\newcommand{\SetDeptED}
{
  \SetDeptName{教育研究所}{ED}{Institute of Education}
  \SetCollegeSocialScience
} % End of \newcommand{}

% 認知科學研究所 Institute of Cognitive Science
\newcommand{\SetDeptIOCS}
{
  \SetDeptName{認知科學研究所}{IOCS}{Institute of Cognitive Science}
  \SetCollegeSocialScience
} % End of \newcommand{}

% 政治經濟學研究所 Institute of Political Economy
\newcommand{\SetDeptGIPE}
{
  \SetDeptName{政治經濟學研究所}{GIPE}{Institute of Political Economy}
  \SetCollegeSocialScience
} % End of \newcommand{}

% 心智影像研究中心 Mind Research and Image Center
\newcommand{\SetDeptFMRI}
{
  \SetDeptName{心智影像研究中心}{FMRI}{Mind Research and Image Center}
  \SetCollegeSocialScience
} % End of \newcommand{}

% ----------------------------------------------------------------------------
% ---------------- 規劃與設計學院 College of Planning & Design ----------------
% ----------------------------------------------------------------------------

% 建築學系 Department of Architecture
\newcommand{\SetDeptArch}
{
  \SetDeptName{建築學系}{Arch}{Department of Architecture}
  \SetCollegePlanningAndDesign
} % End of \newcommand{}

% 都市計劃學系 Department of Urban Planning
\newcommand{\SetDeptUP}
{
  \SetDeptName{都市計劃學系}{UP}{Department of Urban Planning}
  \SetCollegePlanningAndDesign
} % End of \newcommand{}

% 工業設計學系 Department of Industrial Design
\newcommand{\SetDeptID}
{
  \SetDeptName{工業設計學系}{ID}{Department of Industrial Design}
  \SetCollegePlanningAndDesign
} % End of \newcommand{}

% 創意產業設計研究所 Institute of Creative Industry Design
\newcommand{\SetDeptICID}
{
  \SetDeptName{創意產業設計研究所}{ICID}{Institute of Creative Industry Design}
  \SetCollegePlanningAndDesign
} % End of \newcommand{}

% ----------------------------------------------------------------------------
% ---------- 生物科學與科技學院 College of Bioscience & Biotechnology ----------
% ----------------------------------------------------------------------------

% 生命科學系 Department of Life Sciences
\newcommand{\SetDeptBio}
{
  \SetDeptName{生命科學系}{Bio}{Department of Life Sciences}
  \SetCollegeBioscienceAndBiotechnology
} % End of \newcommand{}

% 生物科技研究所 Institute of Biotechnology
\newcommand{\SetDeptBioTech}
{
  \SetDeptName{生物科技研究所}{BioTech}{Institute of Biotechnology}
  \SetCollegeBioscienceAndBiotechnology
} % End of \newcommand{}

% 生物資訊與訊息傳遞研究所 Institute of Bioinformatics and Biosignal Transduction
\newcommand{\SetDeptIBBT}
{
  \SetDeptName{生物資訊與訊息傳遞研究所}{IBBT}{Institute of Bioinformatics and \\ Biosignal Transduction}
  \SetCollegeBioscienceAndBiotechnology
} % End of \newcommand{}

% 熱帶植物科學研究所 Institute of Tropical Plant Sciences
\newcommand{\SetDeptITPS}
{
  \SetDeptName{熱帶植物科學研究所}{ITPS}{Institute of Tropical Plant Sciences}
  \SetCollegeBioscienceAndBiotechnology
} % End of \newcommand{}

% ----------------------------------------------------------------------------
% --------------------- 醫學院 College of Medicine ---------------------
% ----------------------------------------------------------------------------

% 醫學系 School of Medicine
\newcommand{\SetDeptEDUC}
{
  \SetDeptName{醫學系}{EDUC}{School of Medicine}
  \SetCollegeMedicine
} % End of \newcommand{}

% 生物化學暨分子生物學研究所 Department of Biochemistry and Molecular Biology
\newcommand{\SetDeptBiohem}
{
  \SetDeptName{生物化學暨分子生物學研究所}{Biohem}{Department of Biochemistry and \\ Molecular Biology}
  \SetCollegeMedicine
} % End of \newcommand{}

% 病理學科 Department of Pathology
\newcommand{\SetDeptPath}
{
  \SetDeptName{病理學科}{Path}{Department of Pathology}
  \SetCollegeMedicine
} % End of \newcommand{}

% 內科學科 Department of Internal Medicine
\newcommand{\SetDeptIntMed}
{
  \SetDeptName{內科學科}{IntMed}{Department of Internal Medicine}
  \SetCollegeMedicine
} % End of \newcommand{}

% 生理學研究所 Department of Physiology
\newcommand{\SetDeptPhysMed}
{
  \SetDeptName{生理學研究所}{PhysMed}{Department of Physiology}
  \SetCollegeMedicine
} % End of \newcommand{}

% 外科學科 Department of Surgery
\newcommand{\SetDeptSurgery}
{
  \SetDeptName{外科學科}{Surgery}{Department of Surgery}
  \SetCollegeMedicine
} % End of \newcommand{}

% 小兒學科 Department of Pediatrics
\newcommand{\SetDeptPed}
{
  \SetDeptName{小兒學科}{Ped}{Department of Pediatrics}
  \SetCollegeMedicine
} % End of \newcommand{}

% 解剖學科暨細胞生物與解剖學研究所 Department of Cell Biology and Anatomy
\newcommand{\SetDeptAnatomy}
{
  \SetDeptName{解剖學科暨細胞生物與解剖學研究所}{Anatomy}{Department of Cell Biology and Anatomy}
  \SetCollegeMedicine
} % End of \newcommand{}

% 婦產學科 Department of Obstetrics and Gynecology
\newcommand{\SetDeptObsGyn}
{
  \SetDeptName{婦產學科}{ObsGyn}{Department of Obstetrics and Gynecology}
  \SetCollegeMedicine
} % End of \newcommand{}

% 骨科學科 Department of Orthopaedics
\newcommand{\SetDeptBone}
{
  \SetDeptName{骨科學科}{Bone}{Department of Orthopaedics}
  \SetCollegeMedicine
} % End of \newcommand{}

% 公共衛生學科暨公共衛生研究所 Department of Public Health
\newcommand{\SetDeptPhMed}
{
  \SetDeptName{公共衛生學科暨公共衛生研究所}{PhMed}{Department of Public Health}
  \SetCollegeMedicine
} % End of \newcommand{}

% 神經學科 Department of Neurology
\newcommand{\SetDeptNeuro}
{
  \SetDeptName{神經學科}{Neuro}{Department of Neurology}
  \SetCollegeMedicine
} % End of \newcommand{}

% 精神學科 Department of Psychiatry
\newcommand{\SetDeptPsy}
{
  \SetDeptName{精神學科}{Psy}{Department of Psychiatry}
  \SetCollegeMedicine
} % End of \newcommand{}

% 寄生蟲學科 Department of Parasitology
\newcommand{\SetDeptParasite}
{
  \SetDeptName{寄生蟲學科}{Parasite}{Department of Parasitology}
  \SetCollegeMedicine
} % End of \newcommand{}

% 眼科學科 Department of Ophthalmology
\newcommand{\SetDeptOphth}
{
  \SetDeptName{眼科學科}{Ophth}{Department of Ophthalmology}
  \SetCollegeMedicine
} % End of \newcommand{}

% 耳鼻喉學科 Department of Otolaryngology
\newcommand{\SetDeptOtolaryngo}
{
  \SetDeptName{耳鼻喉學科}{Otolaryngo}{Department of Otolaryngology}
  \SetCollegeMedicine
} % End of \newcommand{}

% 工業衛生學科暨環境醫學研究所 Department of Environmental and Occupational Health
\newcommand{\SetDeptDEOH}
{
  \SetDeptName{工業衛生學科暨環境醫學研究所}{DEOH}{Department of Environmental and Occupational Health}
  \SetCollegeMedicine
} % End of \newcommand{}

% 皮膚學科 Department of Dermatology
\newcommand{\SetDeptDerm}
{
  \SetDeptName{皮膚學科}{Derm}{Department of Dermatology}
  \SetCollegeMedicine
} % End of \newcommand{}

% 泌尿學科 Department of Urology
\newcommand{\SetDeptUro}
{
  \SetDeptName{泌尿學科}{Uro}{Department of Urology}
  \SetCollegeMedicine
} % End of \newcommand{}

% 藥理學科暨藥理學研究所 Department of Pharmacology
\newcommand{\SetDeptPharmaco}
{
  \SetDeptName{藥理學科暨藥理學研究所}{Pharmaco}{Department of Pharmacology}
  \SetCollegeMedicine
} % End of \newcommand{}

% 麻醉學科 Department of Anesthesiology
\newcommand{\SetDeptAnesth}
{
  \SetDeptName{麻醉學科}{Anesth}{Department of Anesthesiology}
  \SetCollegeMedicine
} % End of \newcommand{}

% 復健學科 Department of Physical Medicine and Rehabilitation
\newcommand{\SetDeptRehab}
{
  \SetDeptName{復健學科}{Rehab}{Department of Physical Medicine and Rehabilitation}
  \SetCollegeMedicine
} % End of \newcommand{}

% 微生物學及免疫研究所 Department of Microbiology and Immunology
\newcommand{\SetDeptMicrobio}
{
  \SetDeptName{微生物學及免疫研究所}{Microbio}{Department of Microbiology and Immunology}
  \SetCollegeMedicine
} % End of \newcommand{}

% 放射線學科 Department of Diagnostic Radiology
\newcommand{\SetDeptRad}
{
  \SetDeptName{放射線學科}{Rad}{Department of Diagnostic Radiology}
  \SetCollegeMedicine
} % End of \newcommand{}

% 核子醫學科 Department of Nuclear Medicine
\newcommand{\SetDeptNM}
{
  \SetDeptName{核子醫學科}{NM}{Department of Nuclear Medicine}
  \SetCollegeMedicine
} % End of \newcommand{}

% 家庭醫學科 Department of Family Medicine
\newcommand{\SetDeptFamily}
{
  \SetDeptName{家庭醫學科}{Family}{Department of Family Medicine}
  \SetCollegeMedicine
} % End of \newcommand{}

% 急診學科 Department of Emergency Medicine
\newcommand{\SetDeptEmergency}
{
  \SetDeptName{急診學科}{Emergency}{Department of Emergency Medicine}
  \SetCollegeMedicine
} % End of \newcommand{}

% 牙科學科 Department of Dentistry
\newcommand{\SetDeptDentistry}
{
  \SetDeptName{牙科學科}{Dentistry}{Department of Dentistry}
  \SetCollegeMedicine
} % End of \newcommand{}

% 職業及環境醫學科 Department of Occupational and Environmental Medicine
\newcommand{\SetDeptOEM}
{
  \SetDeptName{職業及環境醫學科}{OEM}{Department of Occupational and Environmental Medicine}
  \SetCollegeMedicine
} % End of \newcommand{}

% 法醫學科 Department of Forensic Medicine
\newcommand{\SetDeptForensic}
{
  \SetDeptName{法醫學科}{Forensic}{Department of Forensic Medicine}
  \SetCollegeMedicine
} % End of \newcommand{}

% ----------------------------------------------
%                    護理 Nursing
% ----------------------------------------------

% 護理學系 Department of Nursing
\newcommand{\SetDeptNursing}
{
  \SetDeptName{護理學系}{Nursing}{Department of Nursing}
  \SetCollegeMedicine
} % End of \newcommand{}

% 醫學檢驗生物技術學系 Department of Medical Laboratory Science and Biotechnology
\newcommand{\SetDeptMT}
{
  \SetDeptName{醫學檢驗生物技術學系}{MT}{Department of Medical Laboratory \\Science and Biotechnology}
  \SetCollegeMedicine
} % End of \newcommand{}

% 物理治療學系 Department of Physical Therapy
\newcommand{\SetDeptPT}
{
  \SetDeptName{物理治療學系}{PT}{Department of Physical Therapy}
  \SetCollegeMedicine
} % End of \newcommand{}

% 職能治療學系 Department of Occupational Therapy
\newcommand{\SetDeptOT}
{
  \SetDeptName{職能治療學系}{OT}{Department of Occupational Therapy}
  \SetCollegeMedicine
} % End of \newcommand{}

% 藥學系 School of Pharmacy
\newcommand{\SetDeptPharmacy}
{
  \SetDeptName{藥學系}{Pharmacy}{School of Pharmacy}
  \SetCollegeMedicine
} % End of \newcommand{}

% 基礎醫學研究所 Institute of Basic Medical Sciences
\newcommand{\SetDeptBasicMed}
{
  \SetDeptName{基礎醫學研究所}{BasicMed}{Institute of Basic Medical Sciences}
  \SetCollegeMedicine
} % End of \newcommand{}

% 行為醫學研究所 Institute of Behavioral Medicine
\newcommand{\SetDeptBehMed}
{
  \SetDeptName{行為醫學研究所}{BehMed}{Institute of Behavioral Medicine}
  \SetCollegeMedicine
} % End of \newcommand{}

% 臨床藥學與藥物科技研究所 Institute of Clinical Pharmacy and Pharmaceutical Sciences
\newcommand{\SetDeptCLPARM}
{
  \SetDeptName{臨床藥學與藥物科技研究所}{CLPARM}{Institute of Clinical Pharmacy \\ and Pharmaceutical Sciences}
  \SetCollegeMedicine
} % End of \newcommand{}

% 分子醫學研究所 Institute of Molecular Medicine
\newcommand{\SetDeptIMM}
{
  \SetDeptName{分子醫學研究所}{IMM}{Institute of Molecular Medicine}
  \SetCollegeMedicine
} % End of \newcommand{}

% 口腔醫學研究所 Institute of Oral Medicine
\newcommand{\SetDeptIOM}
{
  \SetDeptName{口腔醫學研究所}{IOM}{Institute of Oral Medicine}
  \SetCollegeMedicine
} % End of \newcommand{}

% 臨床醫學研究所 Institute of Clinical Medicine
\newcommand{\SetDeptICMMed}
{
  \SetDeptName{臨床醫學研究所}{ICMMed}{Institute of Clinical Medicine}
  \SetCollegeMedicine
} % End of \newcommand{}

% 健康照護科學研究所 Institute of Allied Health Sciences
\newcommand{\SetDeptAlliedHealth}
{
  \SetDeptName{健康照護科學研究所}{AlliedHealth}{Institute of Allied Health Sciences}
  \SetCollegeMedicine
} % End of \newcommand{}

% 老年學研究所 Institute of Gerontology
\newcommand{\SetDeptIOG}
{
  \SetDeptName{老年學研究所}{IOG}{Institute of Gerontology}
  \SetCollegeMedicine
} % End of \newcommand{}

% ----------------------------------------------------------------------------

%
% This file is part of the project of
% National Cheng Kung University (NCKU) Thesis/Dissertation Template in LaTex.
% This project is hold at
%     <https://github.com/wengan-li/ncku-thesis-template-latex>
% by Wen-Gan Li.
%
% This project is distributed in the hope of usefuling to someone,
% you can redistribute it and/or modify it under the terms of the
% Attribution-NonCommercial-ShareAlike 4.0 International.
%
% You should have received a copy of the
% Attribution-NonCommercial-ShareAlike 4.0 International
% along with this project.
% If not, see <http://creativecommons.org/licenses/by-nc-sa/4.0/legalcode.txt>.
%
% Please feel free to fork it, modify it, and try it.
% Have fun !!!
%

% Some helper function about watermark

% ----------------------------------------------------------------------------

% 學校圖案版的watermark
% 預設使用NCKU的浮水印
\newcommand{\VarWatermaskTextStyle}{%
  \vfill%
  \centering%
  \makebox(0,0){\rotatebox{45}{\textcolor[gray]{0.75}%
    {\fontsize{2.0cm}{2.0cm}\selectfont{\GetUniversityChiName}}}}%
  \vfill%
} % End of \newcommand{}

% 學校文字版的watermark
% 預設使用NCKU的浮水印
\newcommand{\VarWatermaskFigureStyle}{%
  \vfill%
  \centering%
  \includegraphics[]{./configure/style/ncku/watermark-20160509_v2-a4.pdf}%
  \vfill%
} % End of \newcommand{}

% Prodive re-define APIs
\newcommand{\SetWatermaskTextStyle}[1]{%
  \renewcommand{\VarWatermaskTextStyle}{#1}%
} % End of \newcommand{}

\newcommand{\SetWatermaskFigureStyle}[1]{%
  \renewcommand{\VarWatermaskFigureStyle}{#1}%
} % End of \newcommand{}

\newcommand{\UseWatermarkFigureStyle}{%
  \AddToShipoutPicture{%
    \put(0,0){%
    \parbox[b][\paperheight]{\paperwidth}{%
      \VarWatermaskFigureStyle%
  }}}%
} % End of \newcommand{}

\newcommand{\UseWatermarkTextStyle}{%
  \AddToShipoutPicture{%
    \put(0,0){%
    \parbox[b][\paperheight]{\paperwidth}{%
      \VarWatermaskTextStyle%
  }}}%
} % End of \newcommand{}

\newcommand{\ClearWatermarkStyle}{\ClearShipoutPicture}

% 設定預設使用學校浮水印 Watermark
\UseWatermarkFigureStyle

% ----------------------------------------------------------------------------


% ----------------------------------------------------------------------------


% --------------------------

% 學校排版 Arrangement style
%
% This file is part of the project of
% National Cheng Kung University (NCKU) Thesis/Dissertation Template in LaTex.
% This project is hold at
%     <https://github.com/wengan-li/ncku-thesis-template-latex>
% by Wen-Gan Li.
%
% This project is distributed in the hope of usefuling to someone,
% you can redistribute it and/or modify it under the terms of the
% Attribution-NonCommercial-ShareAlike 4.0 International.
%
% You should have received a copy of the
% Attribution-NonCommercial-ShareAlike 4.0 International
% along with this project.
% If not, see <http://creativecommons.org/licenses/by-nc-sa/4.0/legalcode.txt>.
%
% Please feel free to fork it, modify it, and try it.
% Have fun !!!
%

% ------------------------------------------------

% 學校排版 Arrangement style

% 國立成功大學排版設定
%
% This file is part of the project of
% National Cheng Kung University (NCKU) Thesis/Dissertation Template in LaTex.
% This project is hold at
%     <https://github.com/wengan-li/ncku-thesis-template-latex>
% by Wen-Gan Li.
%
% This project is distributed in the hope of usefuling to someone,
% you can redistribute it and/or modify it under the terms of the
% Attribution-NonCommercial-ShareAlike 4.0 International.
%
% You should have received a copy of the
% Attribution-NonCommercial-ShareAlike 4.0 International
% along with this project.
% If not, see <http://creativecommons.org/licenses/by-nc-sa/4.0/legalcode.txt>.
%
% Please feel free to fork it, modify it, and try it.
% Have fun !!!
%
% ------------------------------------------------
%
% 台灣國立成功大學(NCKU)碩博士論文排版設定
%
% 基礎於陳朝鈞老師所提供的模版中的'utdiss.sty'進行了重新編寫和修改,
% 以對應模版設計和新版的LaTex (如LaTex 3).
% 有關修改內容可參考 <https://github.com/wengan-li/ncku-thesis-template-latex> 中
% 的ChangeLog.md.
% 對這檔案修改的版本和大約時間為:
%         v1.0.0 [Oct 14, 2014 ] 和更早的時間
%         v1.3.0 [Oct 26, 2016]
%         v1.4.1 [May 19, 2016]
%         v1.4.4 [May 25, 2016]
%         v1.4.5 [June 2, 2016]
%         v1.5.0 [Sep 11, 2016]
%         v1.5.2 [Jan 14, 2017]
%         v1.5.3 [Jul 11, 2018]
%
% 原生自utdiss.sty
% utdiss.sty --- Version 2.1.1 (April 1992)
% Doctoral Dissertation Format Macros for The Univ. of Texas at Austin
%     By Young U. Ryu
%     Modified by Glenn G. Lai for LaTeX2e (May 1995)
%
% ------------------------------------------------

% 頁面邊界

% 國立成功大學各系(所)博碩士撰寫論文須知105.12.15 105學年度第2次教務會議修正
% 論文封面及內頁紙張規格:寬21 公分,長29.6 公分 (即A4尺寸) 80磅模造紙。
% 封面邊界:
%     直式:上23mm、下30mm、左20mm、右20mm
%     橫式:上37mm、下32mm、左28mm、右20mm
% 內頁邊界:
%     上23mm、下35mm(含頁碼)、左30mm、右25mm

% 預設邊界為: 內頁邊界
\geometry{a4paper,%
  paperwidth=21cm,paperheight=29.6cm,%
  top=2.3cm,bottom=3.5cm,left=3.0cm,right=2.5cm,%
  nohead%
} % End of \geometry{}

% API用來 '開始使用' 和 '停止使用' 封面邊界
\newcommand{\EnableCoverPageStyle}{%
  \newgeometry{%
    top=2.3cm,bottom=3cm,left=2cm,right=2cm,%
    nohead,nofoot%
  }%
} % End of \newcommand{}

\newcommand{\DisableCoverPageStyle}{\restoregeometry}

% ------------------------------------------------

% 設定學校名字
\SetUniversityName{國立成功大學}{National Cheng Kung University}

% ------------------------------------------------

% 預設前幾頁的頁碼是用羅馬數字
\pagenumbering{roman}

% ------------------------------------------------

% 設定學校圖案版的浮水印樣子
\SetWatermaskFigureStyle{%
  \vfill%
  \centering%
  \includegraphics[]{./configure/style/ncku/watermark-20160509_v2-a4.pdf}%
  \vfill%
} % End of \SetWatermaskFigureStyle{}

% 設定學校文字版的浮水印樣子
% \GetUniversityChiName: 拿學校中文名字
% \GetUniversityEngName: 拿學校英文名字
\SetWatermaskTextStyle{%
  \vfill%
  \centering%
  \makebox(0,0){\rotatebox{45}{\textcolor[gray]{0.75}%
    {\fontsize{2.0cm}{2.0cm}\selectfont{\GetUniversityChiName}}}}%
  \vfill%
} % End of \SetWatermaskTextStyle{}

% 設定預設使用學校浮水印的類型
\UseWatermarkFigureStyle
%\UseWatermarkTextStyle

% ------------------------------------------------

% 因每個學校所要求的內容不一樣, 故overwrite一些內容或APIs.

% 封面日期是統一使用學位考試合格(口試合格單)單為主要參考日期 (年、月(學位考試通過日期)).
\renewcommand{\SetOralDate}[3]
{%
  \SetOralChiDate{#1}{#2}{#3}%
  \SetOralEngDate{#1}{#2}{#3}%
  %
  \SetThesisTaiwanYear{#1}%
  \renewcommand{\ThesisYear}{#1}%
  \renewcommand{\ThesisMonth}{#2}%
} % End of \renewcommand{}

\renewcommand{\SetCoverDate}[3]
{%
%  \SetThesisTaiwanYear{\GetOralEngYear}
%  \renewcommand{\ThesisYear}{\GetOralEngYear}
%  \renewcommand{\ThesisMonth}{\GetOralEngMonth}
} % End of \renewcommand{}


% ----------------------

% 自定的排版設定
%   請參考'template/style/ncku'的檔案和當中的'ncku.tex'
%   或'template/style/Customization.md'.

%\input{./template/style/custom/custom}

% ------------------------------------------------


% --------------------------

% 論文有關資料
% This file is need to encoded in utf-8
%
% Choose or fill in some needed information for this thesis or dissertation
% 選擇或填入你的論文一些需要使用的資料

% ----------------------------------------------------------------------------

% --- 使用的論文內容 ---
% 如果沒有打開\DemoMode
% 就會使用'./context/context.tex'中你所編寫論文內容.
% 否則會使用'./example/context.tex'的模版說明文件內容.

\DemoMode

% ----------------------------------------------------------------------------

% --- 行距 ---
% 同學可自行設定每行的距離, 這邊是以放大縮小方式來使用.
% 所以是輸入 0.1, 0.5, 1, 1.0, 1.5, 2.0, 2 等數字.
% 預設的行距: 1.2

%\SetLineStretch{1.2}

% ----------------------------------------------------------------------------

% --- 封面上語言和名字顯示方式 ---
%
% \DisplayCoverInChi:  封面以全中文顯示
% \DisplayCoverInEng:  封面以全英文顯示
% 只能選擇其中一個, 但只有最後設定的一方有效
% 預設使用\DisplayCoverInEng
%
% 另外預設在封面上只會顯示中文或英文名字而已.
% 不論你是使用\DisplayCoverInChi或\DisplayCoverInEng,
% 使用\DisplayCoverPeoplesBothNames以設定同時顯示中英文名字.

%\DisplayCoverInChi
\DisplayCoverInEng
\DisplayCoverPeoplesBothNames

% ----------------------------------------------------------------------------

% --- Title 論文題目 ---
% 填寫中文和(或)英文
% 如果題目內有必須以數學模式表示的符號,請用 \mbox{} 包住數學模式
% 如果覺得自動產生出來的題目斷行位置不適合, 可以手動加'\\'來強制斷行
% (圖書館說不管你是編寫中英混合或全英文版, 都必須同時存在中英題目)
%
% 有3種可使用, 可獨立使用, 但只有最後設定的一方有效
% \SetTitle{你的題目}{Your Title}   % 同時設定中英文題目
% \SetChiTitle{你的題目}            % 只設定中文題目
% \SetEngTitle{Your Title}         % 只設定英文題目
%
% e.g:
%
% \SetTitle %
% {國立成功大學碩博士用畢業論文LaTex模版} %
% {National Cheng Kung University (NCKU) Thesis/Dissertation Template in LaTex}
%
% or
%
% \SetChiTitle{國立成功大學碩博士用畢業論文LaTex模版}
% \SetEngTitle{National Cheng Kung University (NCKU) \\Thesis/Dissertation Template in LaTex}

\SetTitle %
{國立成功大學碩博士用畢業論文LaTex模版} %
{National Cheng Kung University (NCKU) \\Thesis/Dissertation Template in LaTex}

% ----------------------------------------------------------------------------

% --- Draft 初稿 ---
% 顯示 '(初稿)' (中文版) 和 '(Draft)' (英文版) 在封面
\DisplayDraft

% ----------------------------------------------------------------------------

% --- Degree name 學位 ---
%
% 有2種可選擇, 但只有最後設定的一方有效
% \PhdDegree    % 博士學位
% \MasterDegree % 碩士學位

\PhdDegree

% ----------------------------------------------------------------------------

% --- Your name 你的名字 ---
% 填寫你的中文和(或)英文

% 有3種可使用, 可獨立使用, 但只有最後設定的一方有效
% \SetMyName{你的名字}{Your name}   % 同時設定你的中英文名字
% \SetMyChiName{你的名字}           % 只設定你的中文名字
% \SetMyEngName{Your name}         % 只設定你的英文名字

\SetMyName{你的名字}{Your name}

% ----------------------------------------------------------------------------

% --- Date 日期 ---

% 封面日期是統一使用學位考試合格(口試合格單)單為主要參考日期 (年、月(學位考試通過日期)).
% 例如105年7月口試,則封面日期為 中華民國105年7月 或 2016年7月.

% --- 口試的日期 ---
% 設定西元的年月日, 會自動計算出民國的年份, 和英文的月份轉換
% 次序: {年份}{月份}{日}
% \SetOralDate{2016}{12}{31}

\SetOralDate{2016}{12}{31}

%--------------------------------------------------

% --- 論文封面上的日期 ---

% 如是你是國立成功大學的學生, 則封面日期直接使用口試日期, 故不需再另設定.
% 但如果你不是國立成功大學的學生, 那本模版則不清楚 貴學校所定的規範是否要分開, 故先保留這功能.

% 設定西元的年月, 會自動計算出民國的年份, 和英文的月份轉換
% 次序: {年份}{月份}
% \SetCoverDate{2016}{12}

\SetCoverDate{2016}{12}

% ----------------------------------------------------------------------------

% --- 系所 Department or Institute ---
%
% 設定你的系所名字, e.g:
% \SetDeptMath 數學系
% \SetDeptCSIE 資訊工程學系

\SetDeptCSIE

% ----------------------------------------------------------------------------

% --- 指導老師 Advisor(s) ---
% 在封面上預算了最多3位的空間
% 中文名字固定以 博士  為結尾
% 英文名字固定以 Dr. 為開頭

% 有3種可使用, 用來設定3位老師的名字
% \SetAdvisorNameX{老師的名字}{Professor's name} % 同時設定中英文名字
% \SetAdvisorChiNameX{老師的名字}                % 只設定中文名字
% \SetAdvisorEngNameX{Professor's name}         % 只設定英文名字
% (NameX為NameA, NameB, NameC)

% 使用\SetAdvisorNameA是必須的, 而如果你的指導教授有2或3位,
% 那只要增加\SetAdvisorNameB和\SetAdvisorNameC則可

\SetAdvisorNameA{A}{A}
\SetAdvisorNameB{B}{B}
\SetAdvisorNameC{C}{C}

% ----------------------------------------------------------------------------

% --- 學位考試論文證明書 Defense Certificate ---
% 使用範例版本 或 使用檔案 只能選擇其中一方

% 使用範例版本
\DisplayOralTemplate

% --- 範例版本的語言 ---
% 選擇你需要的範例
% (Only work with \DisplayOralTemplate)
% \DisplayOralChiTemplate    % 顯示中文範例版本
% \DisplayOralEngTemplate    % 顯示英文範例版本

\DisplayOralChiTemplate    % 顯示中文範例版本
\DisplayOralEngTemplate    % 顯示英文範例版本

% --- 口試委員 Committee member(s) ---
% 口試委員數量 (至少2位, 最多9位, 預設為9位)
% (Only work with \DisplayOralTemplate)
% 博士學位考試委員會置委員五人至九人
% 碩士學位考試委員會置委員三人至五人
% 口試委員人數含指導教授
\SetCommitteeSize{9}

%--------------------------------------------------

% 使用學位考試論文證明書圖片檔案
% 把你的圖片放在'context/oral'下
% 之後設定中英文版所對應是哪一個檔案
% 就算已啟用\DisplayOralImage,
% 但沒有填寫圖檔檔名的話, 都不會顯示出來.
% (例子用的'example-oral-chi.pdf'和'example-oral-eng.pdf'已放在'context/oral'中)

%\DisplayOralImage                % 顯示圖檔
%\SetOralImageChi{example-oral-chi.pdf}   % 中文版檔案
%\SetOralImageEng{example-oral-eng.pdf}   % 英文版檔案

% ----------------------------------------------------------------------------

% --- 關鍵字 Keyword ---
% 最多9個關鍵字
% 為了方便同學自行設定
% 故所產出來的PDF檔案中的關鍵字和內文摘要的關鍵字
% 可獨立個別設定

% \SetKeywords是設定所產出來的PDF中的Keyword項目
% 可同時填寫中英文
% e.g
% \SetKeywords{Keyword A (關鍵字 A)}{Keyword B (關鍵字 B)}{Keyword C (關鍵字 C)}
% 或單純中文或英文
% \SetKeywords{Keyword A}{Keyword B}{Keyword C}
% \SetKeywords{關鍵字 A}{關鍵字 B}{關鍵字 C}

\SetKeywords{NCKU Thesis/Dissertation template}{Graduate}{LaTex/XeLaTex}

% 摘要中的關鍵字
% 為了方便同學們能達到以下情況:
% a. 只寫中文版摘要
% b. 只寫英文版摘要
% c. 同時寫中英文版摘要
% 故中英文版的關鍵字都是可個別設定
% \SetAbstractChiKeywords: 用來設定中文版摘要的關鍵字
% \SetAbstractEngKeywords: 用來設定英文版摘要的關鍵字
% \SetAbstractExtKeywords: 用來設定英文延伸摘要的關鍵字 (只有你要編寫英文延伸摘要才需要設定)
% 所以只要使用你需要寫的版本則可.
% 但如果2個版本都要寫, 則2個都同時使用則可.
% 沒有填寫的話, 則摘要中的關鍵字部份是不會顯示出來.
%
% e.g
% \SetAbstractChiKeywords{關鍵字 A}{關鍵字 B}{關鍵字 C}
% \SetAbstractEngKeywords{Keyword A}{Keyword B}{Keyword C}
% \SetAbstractExtKeywords{Keyword A}{Keyword B}{Keyword C}
% 英文延伸摘要的關鍵字理應會跟英文版摘要的關鍵字是一樣,
% 但為了同學能編寫不同內容和關鍵字, 故可獨立設定.

\SetAbstractChiKeywords{國立成功大學畢業論文模版}{碩博士}{LaTex/XeLaTex}
\SetAbstractEngKeywords{NCKU Thesis/Dissertation Template}{Graduate}{LaTex/XeLaTex}
\SetAbstractExtKeywords{NCKU Thesis/Dissertation Template}{Graduate}{LaTex/XeLaTex}

% ----------------------------------------------------------------------------

% --- 目錄 Index ---
% 設定可獨立使用, 但只有最後設定的一方有效

% 標題文字語言 Language
% 目錄的標題文字使用預設的中文或是英文
% \IndexChiMode:  標題文字為中文
% \IndexEngMode:  標題文字為英文
% 預設的目錄標題為: 目錄 (中文) / Table of Contents (英文)
% 預設的表格目錄標題為: 表格 (中文) / List of Tables (英文)
% 預設的圖片目錄標題為: 圖片 (中文) / List of Figures (英文)
% 預設使用\IndexEngMode

%\IndexChiMode
\IndexEngMode

% ----------------------

% 目錄標題文字 Text of title
% 如果預設文字不是你所希望的, 那可以使用這邊去個別設定你所希望的文字, 不分中英文.

% 設定目錄標題
%\SetIndexTitleText{Table of Contents / 目錄}

% 設定表格目錄標題
%\SetTablesIndexTitleText{List of Tables / 表格}

% 設定圖片目錄標題
%\SetFiguresIndexTitleText{List of Figures / 圖片}

% ----------------------------------------------------------------------------

% --- 圖片相關的設定 ---
% 預設上每一張圖的名字都是以 'Figure 2.1'
% 假如想使用自定的名字, 如 '圖 2.1'
% 則使用 \SetCustomFigureName{圖} 即可.

%\SetCustomFigureName{Figure}

% ----------------------------------------------------------------------------

% --- 表格相關的設定 ---
% 預設上每一張表的名字都是以 'Table 2.1'
% 假如想使用自定的名字, 如 '表 2.1'
% 則使用 \SetCustomTableName{表} 即可.

%\SetCustomTableName{Table}

% ----------------------------------------------------------------------------

% --- 參考文獻 Reference ---
%
% --- 使用方式 ---
%    \SetupReference{ < 設定 > }
%
% < 設定 >
%    Title (Reference的標題文字)
%    BibStyle (Reference引用時的格式)
%
% ----------------------
%
% < Title >
% 模版提供了一些預設的文字
%    \TextDefaultTitleReferenceChi: 參考文獻
%    \TextDefaultTitleReferenceEng: References
%    \TextDefaultTitleBibliographyEng: Bibliography
%
% 預設上是使用\TextDefaultTitleReferenceEng.
%
% 使用時:
%    Title={\TextDefaultTitleBibliographyEng}
%
% 或自定你的文字:
%    Title={我的參考文獻標題}
%
% ----------------------
%
% < BibStyle >
% 除非有特殊的格式要求, 否則這部份是不用管的.
%
% 使用的格式 | 作者名稱顯示的格式 | 引用時顯示的例子
%   abbrv | H. J. Simpson | [4]
%   plain | Homer Jay Simpson | [4]
%   alpha | Homer Jay Simpson | Sim95
%   apacite | Homer J. S. | Homer, 1995
%
% 模版提供了:
%   LaTex基本格式:
%       abbrv, acm, alpha, apalike, ieeetr, plain, siam, unsrt
%   可參考<https://www.sharelatex.com/learn/Bibtex_bibliography_styles>
%
%   額外的格式:
%      apacite
%   可參考<https://www.overleaf.com/latex/examples/package-example-apacite/jkssmfcpzwmy>
%
% 預設使用plain.
%
% 注意:
%   如果你要轉換使用格式時, 推薦在重新產生論文前, 先把所有除了thesis.tex外的所有
%   thesis開頭或以thesis為檔名的檔案全刪掉.
%   例如'thesis.bbl', 'thesis.aux', 'thesis.lof'等所有檔案.
%   否則有可能在產生論文時遇到錯誤, 如果遇到錯誤, 請不斷重新刪掉和重新產生論文,
%   直到解決問題為止.
%
% 已知:
%   由abbrv/plain轉去apacite必定需要刪除檔案才能進行.
% ----------------------
%
% --- 請在這邊設定你要的樣子 ---
%

%\SetupReference{%
%  Title = {\TextDefaultTitleReferenceEng},
%  BibStyle = {plain},
%} % End of \SetupReference{}

% ----------------------------------------------------------------------------

% --- 章節標題的設定 ---
% 除非對章節標題格式有任何要求, 否則這部份內容是不用管的.
%
% 模版的章節有一個預設的格式:
%
% 一般章節:
%    Chapter: Chapter 1
%    Section: 1.1
%    SubSection: 1.1.1
%    SubSubSection: (空白, 只有題目)
%
% 附錄章節:
%    Chapter: Appendix A
%    Section: A.1
%    SubSection: A.1.1
%    SubSubSection: (空白, 只有題目)
%
% 如對格式有什麼的要求, 請使用\SetNumberingFormat.

% --- 使用方式 ---
%    \SetNumberingFormat[ < 章節類型 > ]{ < 設定 > }

% ----------------------
%
% < 章節類型 >
% 針對每一種的章節都可自設自己需要的格式,
% 有8種類型提供, 包括一般章節和附錄章節.
%    Chapter (章)
%    Section (節)
%    SubSection (小節)
%    SubSubSection (小小節)
%    AppendixChapter (附錄中的章)
%    AppendixSection (附錄中的節)
%    AppendixSubSection (附錄中的小節)
%    AppendixSubSubSection (附錄中的小小節)
%
% ----------------------
%
% < 設定 >
% 以下的設定針對標題中不同內容的設定.
%    BeginText (章節號碼前面的文字)
%    EndText (章節號碼後面的文字)
%    TextAlign (標題文字的位置)
%    CNumStyle ('章' 的數字類型)
%    SNumStyle ('節' 的數字類型)
%    SSNumStyle ('小節' 的數字類型)
%    SSSNumStyle ('小小節' 的數字類型)
%    SepAtIndex (目錄中章節號碼跟章節題目中的分隔符號)
%    SepBetweenCnS ('章' 號碼跟 '節' 號碼中間的分隔符號)
%    SepBetweenSnSS ('節' 號碼跟 '小節' 號碼中間的分隔符號)
%    SepBetweenSSCnSSS ('小節' 號碼跟 '小小節' 號碼中間的分隔符號)
%
% 標題在不同位置使用的內容都不一樣:
%
% 內文:
%   <BeginText> @NUMBER@ <EndText>
%   例如: 第2章
%
%   @NUMBER@為第幾章節的那個數字
%       <CNumStyle> <SepBetweenCnS>
%         <SNumStyle> <SepBetweenSnSS>
%           <SSNumStyle> <SepBetweenSSCnSSS> <SSSNumStyle>
%   例如: 2.1, 3.1.2, A.2
%
% 目錄:
%   <BeginText> @NUMBER@ <EndText> <SepAtIndex> @TITLE@
%   例如: 第2章. 介紹
%
% 被引用時:
%   @NUMBER@
%   例如: 2.1, 3.1.2, A.2
%
% ----------------------
%
% 一個完整的 \SetNumberingFormat 的樣子:
% \SetNumberingFormat[ < 章節類型 > ]{%
%   BeginText = { @文字/符號@ }, EndText = { @文字/符號@ },%
%   TextAlign = { @Left/Center/Right@ },%
%   CNumStyle = { < 數字類型 > }, SNumStyle = { < 數字類型 > },%
%   SSNumStyle = { < 數字類型 > }, SSSNumStyle = { < 數字類型 > },%
%   SepAtIndex = { @文字/符號@ }, SepBetweenCnS = { @文字/符號@ },%
%   SepBetweenSnSS = { @文字/符號@ }, SepBetweenSSCnSSS = { @文字/符號@ },%
% } % End of \SetNumberingFormat{}
%
% ----------------------
%
% --- 數字類型 ---
%
% 模版提供以下的數字類型使用
%    ChiNum (使用 '中文數字' 方式, 如: 一二三)
%    Tiangan 使用 '天干' 方式, 如: 甲乙丙丁戊癸)
%    Arabic (使用 '阿拉伯數字' 方式, 如: 1 2 3 4 5 6)
%    LowerRoman (使用 '小寫羅馬數字' 方式, 如: i ii iii vi x)
%    UpperRoman (使用 '大寫羅馬數字' 方式, 如: I II III VI X)
%    LowerAlph (使用 '小寫英文字母' 方式, 如: a b c)
%    UpperAlph (使用 '大寫英文字母' 方式, 如: A B C)
%
% 選擇你想要的數字類型後, 在<設定>中的這些位置填寫你要的類型
%    CNumStyle
%    SNumStyle
%    SSNumStyle
%    SSSNumStyle
%
%   例如: CNumStyle={Arabic}
%
% ----------------------
%
% --- 標題文字位置 ---
%
% 模版提供以下的位置使用
%   Left: 左邊
%   Center: 置中
%   Right: 右邊
% 預設上所有章節都是Left.
%
% 例如: TextAlign={Center}
%
% ----------------------
%
% --- 例子 ---
%
% 如果 '章' 要由文字改使用為:
%        'Chapter 1' -> '第1章'
% 則使用
%   \SetNumberingFormat[Chapter]{%
%     BeginText = {第}, EndText = {章}%
%   }%
%
% -----------
%
% 如果 '附錄的章' 要由文字改使用為:
%        'Appendix A' -> '附錄 A'
% 則使用
%   \SetNumberingFormat[AppendixChapter]{%
%     BeginText = {附錄 }%
%   }%
%
% -----------
%
% 如果 '章' 要由數字改使用為:
%        'Chapter 1' -> 'Chapter -A-'
% 則使用
%   \SetNumberingFormat[Chapter]{%
%     BeginText = {Chapter -}, EndText = {-},%
%     CNumStyle = {UpperAlph},%
%    }%
%
% -----------
%
% 如果 '節' 要由數字改使用為:
%        '1.2' -> '一 -乙-'
% 則使用
%   \SetNumberingFormat[Section]{%
%     EndText = {-},%
%     CNumStyle = {ChiNum}, SNumStyle = {Tiangan},%
%     SepBetweenCnS = { -},%
%    }%
%
% -----------
%
% 如果 '節' 不想看到 '章' 的數字:
%        '1.2' -> '(2)'
% 則使用
%   \SetNumberingFormat[Section]{%
%     BeginText = {(}, EndText = {)},%
%     CNumStyle = {}, SNumStyle = {Arabic},%
%     SepBetweenCnS = {},%
%    }%
% 不提供 '章' 的數字類型跟中間的分隔符號
%
% ----------------------
%
% 目錄中章節號碼跟章節題目中的分隔符號
% 正常在目錄中會顯示 'Chapter 1. ABCDEF' 或 '第一章. ABCDEF'
% 但因個人喜好, 做法不一樣, 如 'Chapter 1: ABCDEF' 或 '第一章 ABCDEF'
% 故使用 SepAtIndex 可設定你想要的符號或不需要符號
%
% 如想換'章'的由'Chapter 1. ABCDEF'換成'Chapter 1: ABCDEF'
% 則使用
%   \SetNumberingFormat[Chapter]{%
%     SepAtIndex = {:},%
%    }%
%
% 如想換'章'的由'第一章. ABCDEF'換成'第一章 ABCDEF'
%   \SetNumberingFormat[Chapter]{%
%     BeginText = {第}, EndText = {章},%
%     CNumStyle = {ChiNum},%
%     SepAtIndex = {},%
%   }%
% ----------------------
%
% --- 請在這邊設定你要的樣子 ---
%

% Chapter (章)
%\SetNumberingFormat[Chapter]{%
%  BeginText = {Chapter }, EndText = {},
%  CNumStyle = {Arabic},
%  SepAtIndex = {.},
%} % End of \SetNumberingFormat{}

% Section (節)
% \SetNumberingFormat[Section]{%
%   BeginText = {}, EndText = {},
%   TextAlign = {Left},
%   CNumStyle = {Arabic}, SNumStyle = {Arabic},
%   SepAtIndex = {.}, SepBetweenCnS = {.},
% } % End of \SetNumberingFormat{}

% SubSection (小節)
% \SetNumberingFormat[SubSection]{%
%   BeginText = {}, EndText = {},
%   TextAlign = {Left},
%   CNumStyle = {}, SNumStyle = {}, SSNumStyle = {},
%   SepAtIndex = {.}, SepBetweenCnS = {}, SepBetweenSnSS = {},
% } % End of \SetNumberingFormat{}

% SubSubSection (小小節)
%\SetNumberingFormat[SubSubSection]{%
%  BeginText = {}, EndText = {},
%  TextAlign = {Left},
%  CNumStyle = {}, SNumStyle = {}, SSNumStyle = {}, SSSNumStyle = {},
%  SepAtIndex = {}, SepBetweenCnS = {},
%  SepBetweenSnSS = {}, SepBetweenSSCnSSS = {},
%} % End of \SetNumberingFormat{}

% AppendixChapter (附錄中的章)
%\SetNumberingFormat[AppendixChapter]{%
%  BeginText = {Appendix }, EndText = {},
%  CNumStyle = {UpperAlph},
%  SepAtIndex = {.},
%} % End of \SetNumberingFormat{}

% AppendixSection (附錄中的節)
%\SetNumberingFormat[AppendixSection]{%
%  BeginText = {}, EndText = {},
%  TextAlign = {Left},
%  CNumStyle = {UpperAlph}, SNumStyle = {Arabic},
%  SepAtIndex = {.}, SepBetweenCnS = {.},
%} % End of \SetNumberingFormat{}

% AppendixSubSection (附錄中的小節)
%\SetNumberingFormat[AppendixSubSection]{%
%  BeginText = {}, EndText = {},
%  TextAlign = {Left},
%  CNumStyle = {UpperAlph}, SNumStyle = {Arabic}, SSNumStyle = {Arabic},
%  SepAtIndex = {.}, SepBetweenCnS = {.}, SepBetweenSnSS = {.},
%} % End of \SetNumberingFormat{}

% AppendixSubSubSection (附錄中的小小節)
%\SetNumberingFormat[AppendixSubSubSection]{%
%  BeginText = {}, EndText = {},
%  TextAlign = {Left},
%  CNumStyle = {}, SNumStyle = {}, SSNumStyle = {}, SSSNumStyle = {},
%  SepAtIndex = {}, SepBetweenCnS = {},
%  SepBetweenSnSS = {}, SepBetweenSSCnSSS = {},
%} % End of \SetNumberingFormat{}

% ----------------------------------------------------------------------------

% --- Theorems的設定 ---
% 除非對Theorems格式有任何要求, 否則這部份內容是不用管的.
%
% 提供以下的Theorems的使用:
%
%    Definition       (定義)
%    Condition        (條件)
%    Theorem          (定理)
%    Lemma            (引理)
%    Example          (例子)
%    Corollary        (推論)
%    Proposition      (主張)
%    Proof            (證明)
%    Conjecture       (猜想)
%    Note             (附註)
%    Annotation       (註解)
%    Claim            (主張)
%    Case             (情況)
%    Acknowledgment   (確認)
%    Conclusion       (結論)
%    Criterion        (標準)
%    Assertion        (斷言)
%    Problem          (問題)
%    Question         (問題)
%    Hypothesis       (假設)
%    Summary          (總結)
%
% 如對格式有什麼的要求, 請使用\SetNumberingFormat.
% 而插入新內容的話則使用\InsertXXXX.
%
% --- 使用方式 ---
%   \SetTheoremFormat[ < Theorem類型 > ]{ < 設定 > }
%     和
%   \InsertXXXX[ < 設定 > ]{ < 內容 >} (XXXX為Theorem類型), 如
%     \InsertTheorem{ abc }
%     \InsertLemma{ abc }
%     \InsertProof{ abc }
%
% ----------------------
%
% 一個完整的 \SetTheoremFormat 的樣子:
%
% \SetTheoremFormat[ < Theorem類型 > ]{%
%   ShowText = { @文字/符號@ },
%   FollowCounter = { < Counter類型 > },%
% } % End of \SetTheoremFormat{}
%
% ----------------------
%
% --- ShowText ---
%
% ShowText是指所顯示在文章中的文字. 如Proof可修改成:
%    ShowText = {證明}
%
% ----------------------
%
% --- Counter類型 ---
%
% 模版提供以下的Counter類型使用:
%
%    Section
%    Definition
%    Condition
%    Theorem
%    Lemma
%    Example
%    Corollary
%    Proposition
%    Proof
%    Conjecture
%    Note
%    Annotation
%    Claim
%    Case
%    Acknowledgment
%    Conclusion
%    Criterion
%    Assertion
%    Problem
%    Question
%    Hypothesis
%    Summary
%
%  有一些Theorem是不需要Counter, 故那些是不會需要這設定. 如Proof/Note.
%  而需要的則全預設跟隨Section Counter.
%
%  以下為預設使用Counter的清單:
%
%    Definition
%    Condition
%    Theorem
%    Lemma
%    Example
%    Corollary
%    Proposition
%    Conjecture
%    Criterion
%    Assertion
%    Problem
%    Question
%    Hypothesis
%
% ----------------------
%
% --- 請在這邊設定你要的樣子 ---
%

%\SetTheoremFormat[Definition]{ShowText = {Definition}}%
%\SetTheoremFormat[Condition]{ShowText = {Condition}}%
%\SetTheoremFormat[Problem]{ShowText = {Problem}}%
%\SetTheoremFormat[Example]{ShowText = {Example}}%
%\SetTheoremFormat[Theorem]{ShowText = {Theorem}}%
%\SetTheoremFormat[Lemma]{ShowText = {Lemma}}%
%\SetTheoremFormat[Corollary]{ShowText = {Corollary}}%
%\SetTheoremFormat[Proposition]{ShowText = {Proposition}}%
%\SetTheoremFormat[Conjecture]{ShowText = {Conjecture}}%
%\SetTheoremFormat[Proof]{ShowText = {Proof}}%
%\SetTheoremFormat[Note]{ShowText = {Note}}%
%\SetTheoremFormat[Annotation]{ShowText = {Annotation}}%
%\SetTheoremFormat[Claim]{ShowText = {Claim}}%
%\SetTheoremFormat[Case]{ShowText = {Case}}%
%\SetTheoremFormat[Acknowledgment]{ShowText = {Acknowledgment}}%
%\SetTheoremFormat[Conclusion]{ShowText = {Conclusion}}%
%\SetTheoremFormat[Criterion]{ShowText = {Criterion}}%
%\SetTheoremFormat[Assertion]{ShowText = {Assertion}}%
%\SetTheoremFormat[Question]{ShowText = {Question}}%
%\SetTheoremFormat[Hypothesis]{ShowText = {Hypothesis}}%
%\SetTheoremFormat[Summary]{ShowText = {Summary}}%

% ----------------------------------------------------------------------------


% --------------------------

% 在 pdf 簡介欄裡填入相關資料
\FillInPDFData

% ------------------------------------------------

% 一些會受到conf.tex中設定而影響的package或排版的設定

% Makes all pages the height of the text on that page.
% No extra vertical space is added.
%\raggedbottom

% Setup all custom numbering format
\SetupNumberingFormat

% 當所有的package都include完後, 才真正設定我們要的字型,
% 以清掉所有由package影響到的設定.
\InitDefaultFontType

%\setlength{\parindent}{4em}
%\usepackage{indentfirst}

% Initinal all theorem formats
\InitTheoremFormats

% ------------------------------------------------
